\documentclass[a4paper]{article}

\usepackage{amsmath}
\usepackage{amssymb}
\usepackage{parskip}
\usepackage{fullpage}
\usepackage{hyperref}
\usepackage{xcolor}
\usepackage{stellar}
\usepackage{chronology}

\hypersetup{
    colorlinks=true,
    linkcolor=black,
    urlcolor=blue,
    pdftitle={Storia},
    pdfpagemode=FullScreen,
}

\title{Storia}
\author{Paolo Bettelini}
\date{}


\begin{document}

\maketitle
\tableofcontents
\pagebreak

% EAN 9788859300434

\section{Storia}

\sdefinition{Storiografia}{
  La \textit{storiografia} è la disciplina scientifica che si occupa di studiare la storia.
}

\section{Periodizzazione}

\sdefinition{Periodizzazione}{
  La \textit{periodizzazione} è l'operazione culturale volta a suddividere la linea temporale in vari intervalli,
  ciascuno con caratteristiche comuni.
}

Le prime periodizzazioni derivano dalle prime religioni monoteiste (Es. nascità di Gesù, calendario islamico).

Le periodizzazioni sono delle convenzioni.

\section{Fake news storiche}

% Da: Fascismo e fake news

Le fake news sono in genere effimere, ma quelle storiche sono persistenti e
pronfonde nelle persone.

\begin{itemize}
    \item Più una bugia viene ripetuta, più la si può scambiare per verità.
    \item Notizie di oggi viaggiano velocemente, è difficile bloccarle e smentirle.
    \item Comprendere il passato è un modo per comprendere il presente.
    \item Esistono fake news storiche, ancorate ad un argomento preciso.
    \item Bufale storiche vanno contrastate perché falsificano il passato (così come il ricordo e la memoria).
    \item Bufale storiche nascono da osservazioni o testimonianze inesatte, che poi si diffondono in una società pronta ad accoglierle.
    \item Bufale storiche servono ad alimentare emozioni e a rassicurare: credere in un passato positivo può portare la speranza e rischia di creare una prospettiva a cui tendere.
\end{itemize}

Effetti di scardinare le bufale:

\begin{itemize}
    \item Corregere le informazioni sul passato.
    \item Distruggere sicurezze, e ciò può creare incomunicabilità.
    \item Permette di limitare l'ambito di diffusione di queste notizie, che mistificano la memoria e la percezione del presente.
\end{itemize}

\pagebreak

\section{Linea temporale}

\begin{chronology}[250]{-1251}{2010}{\textwidth}
    \event{-1250}{Caduta di Troia}
    \event{-753}{Romolo Re di Roma}
    \event{1}{Nascita di Gesù}
    \event{622}{Egira}
    \event{800}{Carlo Magno Imperatore}
    % rinascimento
    %\event[1450]{1700}{Rinascimento}
    \event{1517}{Riforma protestante}
    % illuminismo
    \event{1789}{Rivoluzione francese}
    % romanticismo
    \event{1922}{Marcia su Roma}
\end{chronology}

\begin{chronology}*[500]{-3000}{2010}{\textwidth}
    \event{476}{Crollo Impero Romano d'Occidente}
    \event{1453}{Crollo Impero Romano d'Oriente}
    \event[-3000]{476}{Età antica}
    \event[476]{1492}{Medioevo}
    \event[1492]{1789}{Età moderna}
    \event[1789]{2010}{Età contemporanea}
\end{chronology}

% preistoria - fino a -3000

\section{Fonti}

Le fonti possono essere distinti in
\begin{itemize}
    \item \textbf{Fonti materiali:} oggetti e i reperti storici.
    \item \textbf{Fonti scritte:} scritto su carta o altri materiali storici.
    \item \textbf{Fonti figurate o iconografiche:} immagini che rappresentano eventi o scene del passato.
    \item \textbf{Fonti orali:} racconti delle persone presenti a un avvenimento.
\end{itemize}

% volontarie e involontarie, dirette indirette

\pagebreak

\section{Antico Regime}

\sdefinition{Antico Regime}{
    Una società dominata dalla disuguaglianza e dall'ingiustizia.
    Antico regime è il termine con il quale gli storici indicano l'insieme delle istituzioni
    politiche, giuridiche, economiche e sociali caratteristiche di gran parte dell'Europa tra 16°
    e 18° secolo. L'espressione ancien régime ("antico regime") fu introdotta dai rivoluzionari
    francesi del 1789 per contrapporre il vecchio regime prerivoluzionario al nuovo regime da
    loro creato in Francia con la Rivoluzione francese.
}

L'Antico regime era un tipo di società caratterizzata:
\begin{itemize}
    \item dall'autorità di un sovrano assoluto alleato con un una Chiesa
        intollerante;
    \item dal diritto fondato sulle disuguaglianze di nascita, che non
        riconosceva il valore del merito e della competenza;
    \item da un ordinamento oppressivo che imponeva ai contadini le
        servitù personali e che in generale schiacciava i sudditi sotto il
        peso delle tasse.
\end{itemize}

L'antico regime è difficile da periodizzare perché è composto da diverse
componenti di diverse epoche, anche di milleni di anni, ancora rigorosamente in vigore.

\subsection{Monarchia}

\sdefinition{Monarchia}{
    Forma di governo in cui i supremi poteri dello stato sono
    accentrati in una sola persona (re, sovrano, monarca), la cui carica non è elettiva e che può
    essere anche affiancata da altre istituzioni: m. \textit{Ereditaria}, \textit{non ereditaria}; m. \textit{Assoluta}, in
    cui il supremo governo statale è concentrato nel monarca; m. \textit{Limitata} o \textit{costituzionale},
    quando, accanto al monarca, vi sono altre istituzioni sovrane, quali il parlamento e il
    governo, che ne controllino il potere in base a una costituzione: si distingue
    la m. \textit{Costituzionale parlamentare} dalla m. \textit{Costituzionale pura} secondo che sia o no in
    vigore il principio parlamentare, ossia della necessità di un rapporto di fiducia fra
    esecutivo e legislativo.
}

Un uomo detenie quindi la sovranità, affidatagli generalmente da una divinità
per guidare il popolo verso la prosperità (legittimazione divina del potere).
La carica è ereditaria e a vita.
Nelle monarchie assolute il potete è indivisibile, è tutto nelle mani
della medesima persona.

\subsection{Repubblica}

\sdefinition{Repubblica}{
    Con riferimento all'età classica, al
    medioevo e alla prima età moderna, ogni stato non retto da un monarca o da un
    dittatore: la R. romana o di Roma, dal 509 al 31 a. C.; le r. oligarchiche della Grecia; le
    R. marinare italiane; la R. di Cromwell in Inghilterra (metà del sec. 17°), ecc. 
}

Una parte dei cittadini detiene la sovranità, che viene esercitata entro i limiti stabiliti dalle leggi.
Vi è una presenza di una pluralità di istituzioni.
La carica pubblica non è ereditaria e generalmente limitata nel tempo.\\
\textbf{\color{red}nota:} una repubblica non è necessariamente democratica.

\subsection{Impero}

\sdefinition{Impero}{
    Per impero si intende un organismo politico costituito da diversi paesi, popolazioni e Stati
collocati anche in zone non contigue, in molti casi caratterizzato dalla presenza di razze
diverse e culture e lingue non omogenee, ma sempre dotato di un centro politico e di un
nucleo nazionale dominante che esercita sull'insieme il comando e il potere supremo.
Nell'antichità e nel Medioevo a capo degli imperi vi erano i monarchi, mentre in età
moderna e contemporanea imperi sono state anche alcune repubbliche.[…]
Il maggiore e più durevole impero del mondo antico sorto in Occidente fu quello romano,
le cui origini vanno ricondotte all'opera dell'imperatore Augusto a partire dal 27 a.C.: egli
riordinò i grandi territori già conquistati da \href{http://www.treccani.it/enciclopedia/roma_(Enciclopedia_dei_ragazzi)/}{Roma} in età repubblicana, territori che
sarebbero stati ulteriormente accresciuti dai suoi successori in Europa, Asia e Africa. I
fondamenti della politica imperiale furono la superiorità militare dei Romani, una
crescente uniformità amministrativa, la diffusione della cultura greco-latina come cultura
egemone, l'allargamento della cittadinanza.\\
Data la sua estensione, l'Impero venne diviso tra il 3° e il 4° secolo in una parte occidentale
e in una parte orientale. Nel 4° secolo l'Impero divenne ufficialmente cristiano
e \href{http://www.treccani.it/enciclopedia/costantino-i-il-grande_(Enciclopedia_dei_ragazzi)/}{Costantino} spostò la capitale principale da Roma a Costantinopoli. Nel 476 l'Impero
d'Occidente crollò in seguito alle invasioni barbariche, mentre quello d'Oriente, l'\href{http://www.treccani.it/enciclopedia/impero-bizantino_(Enciclopedia_dei_ragazzi)/}{Impero bizantino},
sopravvisse fino al 1453, quando venne definitivamente abbattuto dai Turchi
ottomani.
}

\end{document}
