\documentclass[a4paper]{article}

\usepackage{amsmath}
\usepackage{amssymb}
\usepackage{parskip}
\usepackage{fullpage}
\usepackage{hyperref}
\usepackage{stellar}

\hypersetup{
    colorlinks=true,
    linkcolor=black,
    urlcolor=blue,
    pdftitle={Biology},
    pdfpagemode=FullScreen,
}

\title{Biology}
\author{Paolo Bettelini}
\date{}

\begin{document}

\maketitle
\tableofcontents
\pagebreak

% 978 88 6364 9437 (trovato)
% 978 88 6364 9635
% 978 88 6364 9659

\section{Sistemi}

\sdefinition{Sistema}{
    Un \textit{sistema} (vivente e non-vivente) è composto di parti differenti, specializzate e interdipendenti. 
    
    \begin{enumerate}
        \item Organizzazione della relazione fra le parti
        \item Struttura fisica, chimica etc. 
        \item Processo di riproduzione
    \end{enumerate}
}

\sdefinition{Emergenza Sistemica}{
    Una \textit{emergenza sistemica} è lo scopo che le diverse parti riescono ad raggiungere ed eseguire.
}

\subsection{Sistemi viventi}

Il sistema vivente presenta le medesima ma caratteristiche del sistema non-vivente,
ma possiede anche le seguenti componenti.

\subsubsection{Autopoiesi}

\sdefinition{Autopoiesi}{
    La capacità di ripararsi, modificarsi e riprodursi da solo, internamente ed in maniera autonoma.
}
I sistemi viventi sono organizzativamente chiusi, per cui hanno un confine.

\sexample{Sistema autopoietico - ciclo}{
    TODO: mettere foto
}

\sexample{Sistema autopoietico - cellula}{
    TODO: mettere foto
}

\subsubsection{Dissipazione}

\sdefinition{Dissipazione}{
    La necessità di consumare energia, materia ed informazioni dall'esterno.
}
I sistemi viventi sono metabolicamente aperti, per cui hanno degli scambi con l'esterno
e rinnovano il proprio materiale.

\subsubsection{Cognizione}

\sdefinition{Cognizione}{
    L'attiva conoscenza dell'ambiente, esterno ed interno, da parte del sistema.
}

\section{Che cos'è la vita}

\subsection{Visione sistemica}

\subsection{Visione meccanicistica}

\end{document}
