\documentclass[a4paper]{article}

\usepackage{amsmath}
\usepackage{amssymb}
\usepackage{parskip}
\usepackage{fullpage}
\usepackage{hyperref}
\usepackage{stellar}
\usepackage{makecell}
\usepackage{tikz}

\newcommand{\quotes}[1]{``#1''}

\hypersetup{
    colorlinks=true,
    linkcolor=black,
    urlcolor=blue,
    pdftitle={English},
    pdfpagemode=FullScreen,
}

\newcommand*\circled[1]{\tikz[baseline=(char.base)]{
            \node[shape=circle,draw,inner sep=2pt] (char) {#1};}}

\newcommand\hr{\par\vspace{-.5\ht\strutbox}\noindent\hrulefill\par}

\title{English}
\author{Paolo Bettelini}
\date{}

\begin{document}

\maketitle
\tableofcontents
\pagebreak

\section{Definitions}

\subsection{Narration}

The story is told by the narrator, who is not identical with the author. The narrator answers the
question \textit{Who speaks}?

There are essentially two types of narrators:
\begin{itemize}
    \item \textbf{first-person narrator:}
        a character in the story speaking as “I” or “we” is called a first-person
        narrator;
    \item \textbf{third-person narrator:}
        he/she is not a character in the story.
\end{itemize}

The third-person narrator knows the thoughts and emotions of all the characters; he is an
\textbf{omniscient narrator}. An omniscient narrator can move freely in time and space. He can shift from
character to character; reporting what he/she chooses of their speech, actions, thoughts, feelings
and emotions. He may give comments or decide to \quotes{show} the action without judgement.

If the narrator chooses to describe the thoughts and emotions of only one character, he is a \textbf{selective
narrator}: he presents the story through one character's eyes.

A first-person narrator is naturally \textbf{limited} in his perspective.

\subsection{Focalization}

Choosing a perspective or point of view is separate from determining whether the narrator is a
character in the story. The focalizer answers the question \textit{Who sees or perceives}?

There are three kinds of focalizers:
\begin{itemize}
    \item \textbf{zero focalization} corresponds to the omniscient narrator. Here the narrator knows more
    than the characters;
    \item \textbf{an internal focalizer}'s perception belongs to a character within the story. Internal focalizers
    are also called character-focalizers. Here the narrator knows and says only what a given
    character knows;
    \item \textbf{an external focalizer} is a POV character external to the story. An external focalizer is called
    a narrator-focalizer because perception belongs to the narrator. Here the character knows
    more than the narrator.
\end{itemize}

\pagebreak

\section{Romanticism}

%% file:///home/paolo/Downloads/Chapters%204-7.pdf
%% TODO: narration and focalization

\sdefinition{Romanticism}{
    \textit{Romanticism} is an artistic and intellectual movement that originated
    in Europe towards the end of the 18th century.
    The movement emphasized intense emotion as an authentic source of aesthetic experience.
    It granted a new importance to experiences of sympathy, awe, wonder, and terror,
    in part by naturalizing such emotions as responses to the "beautiful" and the "sublime".
}

The word \textit{sublime} embeds the idea of romanticism:

The passion caused by the great and sublime in nature [...], is Astonishment;
and astonishment is that state of the soul,
in which all its motions are suspended,
with some degree of horror.
No passion so effectually robs the mind of all its powers of acting and reasoning as fear.
For fear being an apprehension of pain or death,
it operates in a manner that resembles actual pain. [...] % https://www.taylorfrancis.com/chapters/edit/10.4324/9781315303673-98/sublime-edmund-burke

\pagebreak

\section{Atonement, Ian McEwan}

\subsection{Chapter 1}

\subsubsection{Summary}

\ssummary{Atonement, Ian McEwan - Chapter 1}{
    In the upper-class Tallis family house in England in 1935,
    Briony Tallis writes a play to perform with her visiting
    cousins in honor of her adored brother Leon's return home.
    Her mother, Emily Tallis, indulges Briony by complimenting her work,
    but her cousins Lola and twins Jackson and Pierrot Quincey, are really
    not interested in it. The cousings are staying with the Tallis family
    because their parents are divorcing, and Lola ruins
    Briony's plan to take the role of the play's main character, Arabella, by
    claiming the part for herself.
}

\subsubsection{Quotes}

\begin{minipage}[l]{0.05\textwidth}
    \circled{1}
\end{minipage}
\begin{minipage}[r]{0.95\textwidth}
    \textbf{Quote:} \textit{
        THE PLAY […] was written by her in a two-day tempest of
        composition, causing her to miss a breakfast and a lunch.
        (p. 3)
    }
    \\
    \textbf{Meaning:}
    Briony is very creative and likes to write stories.
    At times she is so focused on writing that she
    forgets about everything else, including her
    basic needs, such as eating.
\end{minipage}
\hr
\begin{minipage}[l]{0.05\textwidth}
    \circled{2}
\end{minipage}
\begin{minipage}[r]{0.95\textwidth}
    \textbf{Quote:} \textit{
        Mrs. Tallis read the seven pages of The Trials of Arabella
        in her bedroom, at her dressing table, with the author's
        arm around her shoulder the whole while. Briony studied
        her mother's face for every trace of shifting emotion, and
        Emily Tallis obliged […]. (p. 4)
    }
    \\
    \textbf{Meaning:} Briony is seeking her mother's approval.
\end{minipage}
\hr
\begin{minipage}[l]{0.05\textwidth}
    \circled{3}
\end{minipage}
\begin{minipage}[r]{0.95\textwidth}
    \textbf{Quote:} \textit{
        Briony was hardly to know it then, but this was the
        project's highest point of fulfillment. (p. 4)
    }
    \\
    \textbf{Meaning:} She will not be able to achieve her goal because this is the highest point,
    and everything can only go down from here. She puts much emphasis on what she wants,
    needing lots of attention and being very self-centered.
\end{minipage}
\hr
\begin{minipage}[l]{0.05\textwidth}
    \circled{4}
\end{minipage}
\begin{minipage}[r]{0.95\textwidth}
    \textbf{Quote:} \textit{
        Her play was […] for her brother, to celebrate his return,
        provoke his admiration and guide him away from his
        careless succession of girlfriends, toward the right form
        of wife, the one who would persuade him to return to the
        countryside, the one who would sweetly request Briony's
        services as a bridesmaid. (p. 4)
    }
    \\
    \textbf{Meaning:} She is selfish and a little bit self-centered but at the same time naïve.
    She thinks she knows better than her brother, even is she is only thirteen years old.
\end{minipage}
\hr
\begin{minipage}[l]{0.05\textwidth}
    \circled{5}
\end{minipage}
\begin{minipage}[r]{0.95\textwidth}
    \textbf{Quote:} \textit{
        Nothing in her life was sufficiently interesting or
        shameful to merit hiding […]. (p. 5)
    }
    \\
    \textbf{Meaning:} We can see the difference between what she writes in her stories and her life.
    By writing, she's projecting the fact that she would like have a more interesting life,
    and have something to hide, which she currenctly hasn't.
\end{minipage}
\hr
\begin{minipage}[l]{0.05\textwidth}
    \circled{6}
\end{minipage}
\begin{minipage}[r]{0.95\textwidth}
    \textbf{Quote:} \textit{
        […] she was discovering, as had many writers before her,
        that not all recognition is helpful. (p. 7)
    }
    \\
    \textbf{Meaning:} She compares herself to other accomplishes (real) writers, even thought she is
    young and does it as a hobby. She would like to be older and an actual writer.
    On the contrary of her mother, which gives her compliments, her sister is too extreme
    in this behavior, making it blatantly fake.
    This shows that the family tends to overprotect her.
\end{minipage}
\hr
\begin{minipage}[l]{0.05\textwidth}
    \circled{7}
\end{minipage}
\begin{minipage}[r]{0.95\textwidth}
    \textbf{Quote:} \textit{
        Her sandals revealed an ankle bracelet and toenails
        painted vermilion. The sight of these nails gave Briony a
        constricting sensation around her sternum, and she knew
        at once that she could not ask Lola to play the prince. (p.
        11)
    }
    \\
    \textbf{Meaning:} She has shown a sign of being a mature and caring about being aesthetically pleasing.
\end{minipage}
\hr
\begin{minipage}[l]{0.05\textwidth}
    \circled{8}
\end{minipage}
\begin{minipage}[r]{0.95\textwidth}
    \textbf{Quote:} \textit{
        In a generally pleasant and well-protected life, she had
        never really confronted anyone before. (p. 15)
    }
    \\
    \textbf{Meaning:}
    Now, for the first time, she has a confrontation with somebody who is a little bit outside of her
    circle. He understands that this is outside of her comfort zone given that she has never had
    to be confronted with anyone and had a really easy life.
    This is basically her first time noticing what the real life is: not everybody gives you what you want.
\end{minipage}

\subsection{Chapter 2-3}

\subsubsection{Exercises}

\sexercise{What do we learn about Cecilia Tallis in chapter 2? Focus on her character and her relationship
with Robbie.}{
    She's the oldest of the sisters, she's educated and went to Cambridge.
    She was able to afford to go to college because she is rich, even thought many women
    did not get an education back then, and thus is anticonventional.
    Cecilia is shown to have a strange dynamic with Robbie, she wants to go against convention
    by doing what is not expected of her (jumping into the fountain).
    She doesn't behave like a women from the upper-class in the 1930s should behave.
    She prefers to stay home instead of going out with her college friends.
    We can infer that Cecilia and Robbie have a lingering attraction, even thought she
    pretends to be annoyed by it because her emotions are forbidden since Robbie is from a lower-class.
}

\sexercise{Reread the incident at the fountain from Cecilia (ch. 2) and Briony's (ch. 3) point of view.}{
    \begin{itemize}
        \item \textit{How should the reader interpret the scene from Cecilia's point of view?}
            The reader should interpret sexual tension, that they are attracted but cannot
            oblige their feelings. She is enjoying the fact
            that this incident might make Robbie feel at fault.
            She wants to punish him, not in a negative way, but rather in a way to
            make him closer to her.
        \item \textit{How does Briony's perspective during this scene influence her interpretation of events?}
            She does not perceives any sexual attraction between them,
            but rather perceives it as if Robbie is attaccking her in some sort of way.
            Briony is too young and inexperienced to actually understand this kind of
            interactions. Briony is left confused by the event and uses
            her imagination to describe it using surreal adjectives, but is surprised
            when she realizes that the pieces of her made up story do not have a logical sequence.
    \end{itemize}
}

\sexercise{What is hinted at in the following lines at the end of chapter 3?}{
    \hspace{2cm}\makecell[l]{
        \textit{Six decades later she would describe how at the age of thirteen she had} \\
        \textit{written her way through a whole history of literature, beginning with stories} \\
        \textit{derived from the European tradition of folktales, through drama with simple moral} \\
        \textit{intent, to arrive at an impartial psychological realism which she had discovered} \\
        \textit{for herself, one special morning during a heat wave in 1935. (p. 38)}
    }
    This is a flashforward sixty years into the future where Briony talks about
    what she learned six decades prior.
}

\pagebreak

\subsection{Chapter 4-7}

\subsubsection{Quotes}

\begin{minipage}[l]{0.05\textwidth}
    \circled{1}
\end{minipage}
\begin{minipage}[r]{0.95\textwidth}
    \textbf{Quote:} \textit{
        IT WAS not until the late afternoon that
        Cecilia judged the vase repaired. It had
        baked all afternoon on a table by a south-
        facing window in the library, and now three
        fine meandering lines […] were all that
        showed. No one would ever know. (p. 40)
    }
    \\
    \textbf{Who speaks / Who sees?:}
    The narrator is third person but we see through Cecilia's eyes.
    Thus, she is the focalizer.
\end{minipage}
\hr
\begin{minipage}[l]{0.05\textwidth}
    \circled{2}
\end{minipage}
\begin{minipage}[r]{0.95\textwidth}
    \textbf{Quote:} \textit{
        [S]he didn't wish to dirty her cashmere by
        lying on the floor, and instead slumped in a
        chair, and the director could hardly object to
        that. The older girl entered so fully into the
        spirit of her own aloof compliance that she
        felt beyond reproach. (p. 52)
    }
    \\
    \textbf{Who speaks / Who sees?:}
    The action is being perceived and judged by Lola (internal focalizer),
    but the narrator is third person.
\end{minipage}
\hr
\begin{minipage}[l]{0.05\textwidth}
    \circled{3}
\end{minipage}
\begin{minipage}[r]{0.95\textwidth}
    \textbf{Quote:} \textit{
        NOT LONG after lunch, once she was assured
        that her sister's children and Briony had
        eaten sensibly and would keep their promise
        to stay away from the pool for at least two
        hours, Emily Tallis had withdrawn from the
        white glare of the afternoon's heat to a cool
        and darkened bedroom. (p. 60)
    }
    \\
    \textbf{Who speaks / Who sees?:}
    There is a third person narrator but the internal focalizer is
    internal (Emily).
\end{minipage}
\hr
\begin{minipage}[l]{0.05\textwidth}
    \circled{4}
\end{minipage}
\begin{minipage}[r]{0.95\textwidth}
    \textbf{Quote:} \textit{
        In a spirit of mutinous resistance, she
        climbed the steep grassy slope to the bridge,
        and when she stood on the driveway, she
        decided she would stay there and wait until
        something significant happened to her. (p.
        72)
    }
    \\
    \textbf{Who speaks / Who sees?:}
    There is a third person narrator but the internal focalizer is Briony.
\end{minipage}

All of these are third person narrator, but the internal focalizer
changes constantly between characters.

\subsubsection{Exercises}

\sexercise{For each item taken from this section of the novel, describe what it is and why it’s relevant to the
story}{
    \begin{itemize}
        \item \textbf{\quotes{the look}}
            It's about Ccilia and Leon, that can communicate by looking at eachother
            without others understanding them. They share a strong bond and in this case
            Leon wishes Cecilia and Paul to marry, but she's not really interested.
        \item \textbf{cigarettes}
            Cecilia smokes even thought is it not very lady-like.
            Cigarettes are sometimes related to feminism,
            indicating a rebellion against society's standards.
        \item \textbf{divorce}
            The parents of the twins and Lola are divorcing. Divorce was
            very rare and taboo and talking about it is quite forbidden.
            The children are spent to the Tallis family to spend time there
            for a while.
        \item \textbf{Hamlet}
            Hamlet is mentioned by Paul Marshall when talking to Lola.
            Both of them pretends to have seen it even thought they really haven't.
        \item \textbf{Army Amo}
            This is a pun because \textit{amo} is the latin word for love,
            but amo is also similar to ammo - this double meaning between
            ammunitions and love is the surname given to Paul's chocolate,
            as a slong for loving you country and fueling the soldier.
            Lola is told about this flirtatiously.
        \item \textbf{curtains}
            Emily goes in her room and closes the curtains - this is a sign
            to tell the world that she doesn't want be disturbed.
        \item \textbf{nettle branch}
            Nettle branch is a simple branch that Briony uses when she sees that the rehersals
            are not doing well, destroying the branch to show her frustration.
    \end{itemize}
}

\end{document}
