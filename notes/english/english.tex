\documentclass[a4paper]{article}

\usepackage{amsmath}
\usepackage{amssymb}
\usepackage{parskip}
\usepackage{fullpage}
\usepackage{hyperref}
\usepackage{stellar}

\hypersetup{
    colorlinks=true,
    linkcolor=black,
    urlcolor=blue,
    pdftitle={English},
    pdfpagemode=FullScreen,
}

\title{English}
\author{Paolo Bettelini}
\date{}

\begin{document}

\maketitle
\tableofcontents
\pagebreak

\section{Definition}

% Arthur Miller A view from the Bridge
% OR a view from the Bridge
% ISBN
% 978-0-141-18996-3 
% 978-1-350-24578-5 

\sdefinition{Romanticism}{
    \textit{Romanticism} is an artistic and intellectual movement that originated
    in Europe towards the end of the 18th century.
    The movement emphasized intense emotion as an authentic source of aesthetic experience.
    It granted a new importance to experiences of sympathy, awe, wonder, and terror,
    in part by naturalizing such emotions as responses to the "beautiful" and the "sublime".
}

The word \textit{sublime} embeds the idea of romanticism:

The passion caused by the great and sublime in nature [...], is Astonishment;
and astonishment is that state of the soul,
in which all its motions are suspended,
with some degree of horror.
No passion so effectually robs the mind of all its powers of acting and reasoning as fear.
For fear being an apprehension of pain or death,
it operates in a manner that resembles actual pain. [...] % https://www.taylorfrancis.com/chapters/edit/10.4324/9781315303673-98/sublime-edmund-burke


\end{document}
