\documentclass[a4paper]{article}

\usepackage{amsmath}
\usepackage{amssymb}
\usepackage{parskip}
\usepackage{fullpage}
\usepackage{hyperref}
\usepackage{stellar}
\usepackage{makecell}
\usepackage{soul}
\usepackage{tikz}
\usepackage{graphicx}
\usepackage{epigraph}

\newcommand*\circled[1]{\tikz[baseline=(char.base)]{
            \node[shape=circle,draw,inner sep=2pt] (char) {#1};}}

\usetikzlibrary{cd}

\hypersetup{
    colorlinks=true,
    linkcolor=black,
    urlcolor=blue,
    pdftitle={Italiano},
    pdfpagemode=FullScreen,
}

\title{Italiano}
\author{Paolo Bettelini}
\date{}

\newcommand{\quotes}[1]{``#1''}

\newcommand\hr{\par\vspace{-.5\ht\strutbox}\noindent\hrulefill\par\vspace{0.15cm}}

\begin{document}

\maketitle
\tableofcontents
\pagebreak

% Autori:
% Dante, Boccaccio, Petrarca
% Machiavelli (sagiistica politica)
% (sagiistica politica) e Ariosto
% Beccaria
% Leopardi
% Leopardi

% Esame scritto: 2 traccie nuove e sceglierne una
% commentare 4 ore di orologio

% Porta un dizionario
% Sufficientemente grande da avere le parole alma guiderdone spirto

% Verifiche:
% 11 ott, Dante 
% 6 dic, Boccaccio(?)

% TODO, spostare definizioni in un punto apposta



\part{Dante (1265-1321)}

\section{Biografia}

La biografia di Dante è molto offuscata e nessuno scritto originale è rimasto.
Questi fattori rendono difficile datare le sue varie opere e contestualizzarle. Inoltre,
è anche difficile validare in maniera precisa le parole esatte scritte dall'autore, siccome i testi che possediamo
sono frutto di trascrizioni.

Dante nasce a Firenze nel 1265 in una piccola nobilità cittadina.
A 12 anni diventa promesso sposo di a Gemma Donati.

A 18 anni incontra Beatrice, dopo averla vista per la prima volta a 9 anni.

Verso l'anno 1295 Dante, si avvicina alla politica.
Si iscrive all'Arte (Arte dei medici e degli speziali), questo è dato dal fatto che essere iscritti ad un Arte
fosse un requisito necessario per esercitare un'attività politica. Dante diventa piore, per cui a capo della cittadina. %%

Prima della nascita di Dante, i Ghibellini sostenevano il potere dell'imperatore, mentre i Guelfi sostenevano quello papa.
Le due parti erano in forte conflitto, e nella battaglia del 1266, muore il figlio dell'imperatore.
I Ghibellini escono quindi di scena quando Dante è appena nato.

Successivamente, i Guelfi si separano in Bianchi e Neri, con un conflitto ancora più forte di quello precedente.

La scena politica fiorentina era dominata dallo scontro fra i Bianchi e i Neri.
Dante, durante il suo priorato, manda in esilio in più violenti dei Neri, fino alla scaduta del suo priorato.
Papa Bonificio VIII manda le truppe di Carlo di Valois, le quali permettono ai Neri di prendere carica al governo.
I bianchi vengono quindi esiliati, fra cui Dante. 

% Boccaccio - primo grande studiso di dante

\pagebreak

\section{Vita Nova}

\subsection{Introduzione}

\sdefinition{Sonetto}{
    Il \textit{sonetto} è una poesia composto da 2 quartine e 2 terzine dove tutti i versi sono degli endecasillabi.
}

\sdefinition{Prosimetro}{
    Il \textit{prosimetro} è un testo ibrido, composto da un
    racconto (prosa) intervallato e poesie (versi).
}

La \textit{Vita nova} è il primo prosimetro. Esso racconta la storia d'amore da parte di Dante
nei confronti di Beatrice.
Questa vicenda diventa un modello per questa tipologia di narrativa.

Il significato del titolo indica come Dante consideri l'inizio della sua vita (nuova vita, rinnovata)
quando vide Beatrice per la prima volta.
Il primo contatto amoroso nella poesie è spesso caratterizzato da un innamoramento a prima vista.

% Il saluto nel medioevo da parte delle donne
Quando la voce dell'interesse di Dante nei confronti di Beatrice le giunge, lei gli nega il saluto.
\snote{Il saluto nel medioevo}{
    Il saluto nel medioevo ha un significato molto più profondo di quello odierno.
    %Infatti, 
}
Nonostante il rifiuto, Dante continua ad esprimere il suo amore verso Beatrice semplicemente
lodandola (scrivendo di lei), completamente senza ricambio di interesse.
Questa loda rappresenta la forma più pura di amore.

Questo libro introduce la simbologia del numero 9 associato a Beatrice.
Ciò è dato dal fatto che Dante l'abbia vista per la prima volta a 9 anni, rivista 9 anni dopo,
e altri motivi che vengono descritti. Il numero 9 è anche un simbolo biblico (3 volte la trinità).

\subsection{Oltre la spera}

Il seguente sonetto descrive il concetto di \textbf{intelligenza nova}
indotta nello spirito di Dante.

\begin{center}
    \textit{Oltre la spera che più larga gira,} \\
    \textit{passa 'l sospiro ch'esce del mio core:} \\
    \textit{intelligenza nova, che l'Amore} \\
    \textit{piangendo mette in lui, pur sù lo tira.}
\end{center}

Il primo verso è una perifrasi che indica \quotes{oltre il pianeta più lontano} (chiamato \textit{Il Primo Nobile}), ossia il paradiso
siccome la visione dell'universo era quella tolemaica e creazionista.

Il secondo verso ci indica che il sospiro del poeta esce dal suo cuore, mentre è vivo, dalla sua intimità più profonda,
e attraverso i cieli fino al paradiso.  

Ai versi 3-4 viene descritto ciò che permette questo percorso, ossia ciò che lo tira
verso l'alto. Questa forza è un'intelligenza nova, ossia una nuova sensibilità nel vedere le cose.
Questa nuova intelligenza deriva dall'amore, che permette all'autore di avere una nuova consapevolezza.
Questa esperienza amorosa è dolorosa ma porta ad una nuova capacità di intendimento.
Inoltre, la parola amore ha la maiuscola perché esso viene personificato.

\begin{center}
    \textit{Quand'elli è giunto là dove disira,} \\
    \textit{\textbf{vede} una donna, che riceve onore,} \\
    \textit{e \textbf{luce} sì, che per lo suo \textbf{splendore}} \\
    \textit{lo peregrino spirito la \textbf{mira}.}
\end{center}

La seconda quartina descrive il punto di arrivo.
Quando lo spirito arriva, vede una donna, la quale viene onorata dagli altri beati, Dio e la Madonna.
Viene anche detto che questa donna brilla.
A causa di questo grande splendore, lo spirito giunto in paradiso (pellegrino, in pellegrinaggio) la ammira.

\begin{center}
    \textit{\textbf{Vedela} tal, che quando 'l mi ridice,} \\
    \textit{io no lo \textbf{'ntendo}, sì \textbf{parla} sottile} \\
    \textit{al cor dolente, che lo fa \textbf{parlare}.}
\end{center}

Lo spirito ripercorre il medesimo tragitto verticale, ma al contrario, tornando da Dante.
Questo spirito cerca di spiegargli che cosa ha visto.
\quotes{La vede tale che quando me lo ridice, io non capisco}.
Dante non comprende quindi ciò che lo spirito gli riferisce, perché
\quotes{parla sottile}, ossia parla in maniera troppo difficile.
Il cuore dolente del poeta è ciò lo fa sì che lo spirito venga interrogato.
Infatti, lo spirito parla proprio al cuore \underline{e} a Dante (questo amplifica l'incomprensione della spiegazione).
Lo spirito parla in maniera troppo complessa perché il linguaggio non riesce
ad esprimere quello che si è provato (topos dell'ineffabilità, è ineffabile)
siccome l'esperienza lo tracende.

\begin{center}
    \textit{So io che \textbf{parla} di quella gentile,} \\
    \textit{però che spesso \textbf{ricorda} Beatrice,} \\
    \textit{\st{sì ch'io lo \textbf{'ntendo} ben, donne mie care.}}
\end{center}

\textbf{\color{red}Nota:} La parola \quotes{però} significa \quotes{per ciò}. \\
Questa è l'unica occorrenza dove Beatrice viene nominata direttamente in un testo poetico in \textit{Vita Nova}.\\
Nell'incomprensione fra Dante e lo spirito, Dante capisce che la donna vista era sicuramente
Beatrice. \\
Nella poesia antica, la parola \textit{gentile} è molto più profonda di quella odierna
e possiede un significato diverso. Essa ha un significato nobile di purezza (nobiltà d'animo).

L'ultimo verso è dato dal fatto che Dante si stesse riferendo a delle Donne nel testo.

\hr

Questo sonetto è diviso in due parti, dove vengono distinte le due verticalità del viaggio dello spirito
(avanti e indietro).

Molte parole della prima parte appartendono alla sfera visiva, poiché il paradiso
è fatto di luci, mentre molte parole della seconda fanno parte del parlare.
Questo è dato dal fatto che lo spirito può vedere, ma ha l'impossibilità di esprimersi.

Questa separazione è collegata dall'uso di due parole quasi uguali,
\textbf{mira} e \textbf{Vedela} (detto per anadiplosi).

% TODO atoni e tonici

\sdefinition{Legge di Tobler Mussafia}{
    È vietato iniziare un verso (poesie o prosa) o far seguire una congiunzione coordinante
    con un pronome atoni.
}

\pagebreak

\section{La Divina Commedia}

\subsection{Cosmo Dantesco}

\sdefinition{Sistema tolemaico}{
    Data la credenza creazionista, Dio ha creato l'uomo e l'ha collocato al centro.
    Per cui, la Terra risiede al centro del sistema solare, dove gli altri pianeti gli ruotano attorno.
}

Le colonne d'Ercole (Stretto di Gibilterra) e La foce del Gange
sono i due estremi della Terra. All'uomo non è concesso conoscere oltre questi confini
(fare ciò implicherebbe peccare di superbia).

Gerusalemme si trova al centro dell'emisfero. Sotto di esso, risiede l'inferno.

Lucifero era l'angelo prediletto di Dio.
Lucifero si ribella a Dio, e per punizione viene scagliato sulla Terra, la quale,
prova ribrezzo e si ritira formando la forma conica dell'inferno. Lucifero si trova nel punto
più profondo dell'inferno, ossia il centro della Terra, nonché il punto più lontano da Dio.

La creazione dell'inferno crea una montagna dall'altra parte del mondo, dove in cima ad esso
vi è il Giardino dell'Eden. Ciò marca anche la creazione del purgatorio.

\subsection{L'inferno}

L'inferno è composto da settori sempre più stretti. Più lo spazio diminuisce e più i peccati sono immorali
secondo Date.
Principalmente, l'inferno è suddiviso in 3 sezioni.
Dall'alto verso il basso, ci sono gli \textit{incontinenti}, \textit{violenti} e
i \textit{freudolenti}.

\sdefinition{Legge del contrappasso}{
    La \textit{legge del contrappasso} associa una pena
    legata alla colpa.
    Il nesso avviene o per \textit{analogia} o per \textit{opposizione}.
}

%\subsection{Struttura dell'inferno}
%\begin{figure}[h]
%    \centering
%    \includegraphics[width=0.5\textwidth]{./inferno.jpg}
%\end{figure}

\subsection{Struttura del testo}

% https://tikzcd.yichuanshen.de/#N4Igdg9gJgpgziAXAbVABwnAlgFyxMJZABgBoBGAXVJADcBDAGwFcYkQAdDnGADx2ABjemDyCAFjAC+IKaXSZc+QinIVqdJq3Zce-XMAAKzAE4Bzejggn8MuQux4CRNcQ0MWbRJ2588AgEkwADMYE0g7eRAMR2UXUgAmdy0vHz1-I3oTeigsOAhIhyVnFATSNxoPbW9dP2ByAGoAZiaAAmFRLELoxScVZDKqSpSdX35gFvaRPG6Y4v6ypOHPUfSJto6Z2Q0YKDN4IlBgkwgAWyQykCskABYaSRz2HAB3CAeoBHsQY7OkNSuIEgAKz3GCPbwvN5gj6yKI-c6IJo0a6IABsoPBV1e70+cJOCLIAL+X3hSEJKMuACMYGAoEgmsQSfiycjAYiaNTaUgALQMqSUKRAA
\begin{center}
\begin{tikzcd}
    & \textit{Inferno} \arrow[r, two heads]    & \text{1+33 canti} \\
    \text{Cantiche} \arrow[r] \arrow[ru, bend left] \arrow[rd, bend right] & \textit{Purgatorio} \arrow[r, two heads] & \text{33 canti}   \\
    & \textit{Paradiso} \arrow[r, two heads]   & \text{33 canti}  
\end{tikzcd}
\end{center}

\subsection{Lucifero}

Lucifero viene rappresentato come un grande orrenda cretura con 3 bocche.
In ogni bocca mastica per l'eternità i 3 peccatori più grandi.
Al centro Giuda, mentre ai lati Bruto e Cassio.

\pagebreak

\subsection{Inferno}

\subsubsection{Inferno, Canto I}

Il primo canto dell'inferno fa da proemio a tutta la Commedia.

\begin{center}
    \textit{Nel mezzo del cammin di nostra vita} \\
    \textit{mi ritrovai per una selva oscura,} \\
    \textit{ché la diritta via era smarrita.}
\end{center}

la vita media è di 70 anni. Questo è un dato biblico e non il valore della vita media.
Dante è nato nel 1265, per cui 1265+35=1300. Il percorso inizia infatti esattamente nel 1300.
Questo dato viene anche confermato in \textit{Inferno XXI, 112-114}.

\sdefinition{Giubileo}{
    Il \textit{giubileo} è un anno di assoluzione collettica di peccati.
}
Questa data è quella del primo Giubileo, indetto dal Papa Bonifacio VIII.

\sdefinition{Allegoria}{
    Figura retorica per mezzo della quale l'autore esprime e il lettore ravvisa un significato riposto,
    diverso da quello letterale.
}
La selva oscura, dove non vi è luce, rappresenta una condizione di peccato.
La \textit{diritta via} è quella che conduce a Dio.
Essa è smarrita, ma può essere appunto ritrovata.
\\
Dante non specifica il tipo di peccato, questo è dato dal fatto che Dante rappresenta allegoricamente
l'interno dell'umanità (nel 1300), per cui il peccato di tutti gli uomini in quel periodo.

\begin{center}
    \textit{Ahi quanto a dir qual era è cosa dura} \\
    \textit{esta selva \textbf{selvaggia} e \textbf{aspra} e \textbf{forte}} \\
    \textit{che nel pensier rinova la paura!}
\end{center}

I tre aggettivi; selvaggia (disumano), aspra (fitta) e forte (da cui è difficile uscire)
sono disposti a climax.

\begin{center}
    \textit{Tant' è amara che poco è più morte;} \\
    \textit{ma per trattar del ben ch'i' vi trovai,} \\
    \textit{dirò de l'altre cose ch'i' v'ho scorte.}
\end{center}

La morte, che è la cosa più terribile che ci sia, lo è solamente poco più della selva.
\\
I due verbi sui quali si chiude \textit{Vita Nova},
\textbf{dire} e \textbf{trattare}, si ritrovano all'inizio della \textit{Commedia}.
In \textit{Vita Nova} questi verbi si riferiscono all'io poetico, mentre all'inizio della \textit{Commedia}
sono riferiti al \textbf{bene} e ad \textbf{altre cose}.
Il bene si riferisce alla salvezza (Dio), mentre altre cose si riferisce a tutto ciò che trovò durante il viaggio.
La parola \textit{vi} si riferisce probabilmente a tutto il viaggio compiuto da Dante.

\begin{center}
    \textit{Io non so ben ridir com' i' v'intrai,} \\
    \textit{tant' era pien di sonno a quel punto} \\
    \textit{che la verace via abbandonai.}
\end{center}

Il sonno rappresenta in senso allegorico il sonno della coscienza, che porta al peccato, ossia la selva.

\snote{Funzioni di Dante}{
    Date ha diverse funzioni che si intrecciano nel racconto.
    \begin{itemize}
        \item Dante personaggio, pellegrino che compie il viaggio
        \item Dante allegoria per tutta l'umanità
        \item Dante poeta fiorentino
    \end{itemize}
}

\begin{center}
    \textit{Ma poi ch'i' fui al piè d'un colle giunto,} \\
    \textit{là dove terminava quella valle} \\
    \textit{che m'avea di paura il cor compunto,}
\end{center}

\begin{center}
    \textit{guardai in alto e vidi le sue spalle} \\
    \textit{vestite già de' raggi del pianeta} \\
    \textit{che mena dritto altrui per ogne calle.}
\end{center}

Le spalle del colle sono il punto in cui la collina si piega.
Questa perifrasi indica semplicemente che la collina era illuminata dalla luce solare.
In alto vi è la luce divina, mentre in basso c'è il buio del peccato.
Il collo rappresenta infatti il percorso difficile; è molto più facile
cadere all'inferno che giungere a Dio. \\
Il gesto di guardare in alto indica un progressivo distaccarsi dal peccato.

\begin{center}
    \textit{Allor fu la paura un poco queta,} \\
    \textit{che nel lago del cor m'era durata} \\
    \textit{la notte ch'i' passai con tanta pieta.}
\end{center}

\begin{center}
    \textit{E come quei che con \textbf{lena affannata},} \\
    \textit{uscito fuor del pelago a la riva,} \\
    \textit{si volge a l'acqua perigliosa e guata,}
\end{center}

% pelago = mare
Il verbo \textbf{guatare} significa guardare con partecipazione, spesso paura.
Questa similitudine mette in confronto un naufrago che scampa il pericolo dell'acqua,
che come Dante scampa dalla selva e si gira a guardarla, con un sentimento di sollievo.
Il corpo di Dante è fermo, ma il suo animo è ancora spaventato e vorrebbe continuare a scappare.

Questo indica anche che indugiare nel peccato, come indugiare nella selva o nelle acque, porta alla morte.

\begin{center}
    \textit{così l'animo mio, ch'ancor fuggiva,} \\
    \textit{si volse a retro a rimirar lo passo} \\
    \textit{che non lasciò già mai persona viva.}
\end{center}

\begin{center}
    \textit{Poi ch'èi posato un poco il corpo lasso,} \\
    \textit{ripresi via per la piaggia diserta,} \\
    \textit{sì che 'l piè fermo sempre era 'l più basso.}
\end{center}

La piaggia è un leggero pendio che non è ancora l'effettiva salita.
\\
Il terzo verso, dove un piede è sempre più basso dell'altro,
implica che ci sia ancora una zavorra che lo mantenga vicino al peccato (alla selva).

\begin{center}
    \textit{Ed ecco, quasi al cominciar de l'erta,} \\
    \textit{una lonza leggiera e presta molto,} \\
    \textit{che di pel macolato era coverta;}
\end{center}

Dante incontra la prima delle tre fiere, la lonza.
\sdefinition{Lussuria}{
    La lussuria è un vizio inteso come l'abbandono alle proprie passioni o anche a divertimenti di natura generica, senza il controllo da parte della nostra ragione e della nostra morale.
}
La lonza è leggera (veloce, agile). Essa rappresenta infatti la lussuria.

\begin{center}
    \textit{e non mi si partia dinanzi al volto,} \\
    \textit{anzi 'mpediva tanto il mio cammino,} \\
    \textit{ch'i' fui per ritornar più volte vòlto.}
\end{center}

\begin{center}
    \textit{Temp' era dal principio del mattino,} \\
    \textit{e 'l sol montava 'n sù con quelle stelle} \\
    \textit{ch'eran con lui quando l'amor divino}
\end{center}

Il tempo è il mattino, e la stazione è la primavera.
Secondo la \textit{Genesi} il mondo è stato creato di primavera.

\begin{center}
    \textit{mosse di prima quelle cose belle;} \\
    \textit{sì ch'a bene sperar m'era cagione} \\
    \textit{di quella fiera a la gaetta pelle}
\end{center}

\begin{center}
    \textit{l'ora del tempo e la dolce stagione;} \\
    \textit{ma non sì che paura non mi desse} \\
    \textit{la vista che m'apparve d'un leone.}
\end{center}

Dante ritrova speranza pensando che il mattino di primavera sia un momento propizio di inizio.
\\
Dante incontro la seconda delle fiere, il leone.

\begin{center}
    \textit{Questi parea che contra me venisse} \\
    \textit{con la test' alta e con rabbiosa fame,} \\
    \textit{sì che parea che l'aere ne tremesse.}
\end{center}

\sdefinition{Superbia}{
    Radicata convinzione della propria superiorità (reale o presunta) che si traduce in atteggiamenti di orgoglioso distacco o anche di ostentato disprezzo verso gli altr
}
Il leone rappresenta la superbia.

\begin{center}
    \textit{Ed una lupa, che di tutte brame} \\
    \textit{sembiava carca ne la sua magrezza,} \\
    \textit{e molte genti fé già viver grame,}
\end{center}

Immediatamente Date incontro anche la terza fiera, la lupa.
\sdefinition{Avarizia (antica)}{
    La brama di possessi materialistici.
}
La lupa è molto magra. Essa rappresenta l'avariazia, la fame insaziabile e la brama di possessi materiali.

\begin{center}
    \textit{questa mi porse tanto di gravezza} \\
    \textit{con la paura ch'uscia di sua vista,} \\
    \textit{ch'io perdei la speranza de l'altezza.}
\end{center}

Questa gravezza (peso) simboleggia il ritorno verso la sede, perdendo la speranza di salire.

\snote{Alternarsi della speranza}{
    Dante viene attraversato da un continuo alternarsi fra speranza e disperazione.
}

\begin{center}
    \textit{E qual è quei che volontieri acquista,} \\
    \textit{e giugne 'l tempo che perder lo face,} \\
    \textit{che 'n tutti suoi pensier piange e s'attrista;}
\end{center}

Questa similitudine si riferisce agli avari (oppure potrebbe riferisci ai giocatori d'azzardo).
Coloro che affidano la propria felicità ai beni materiali, e si disperano quando perdono tuto.


\begin{center}
    \textit{tal mi fece la bestia sanza pace,} \\
    \textit{che, venendomi 'ncontro, a poco a poco} \\
    \textit{mi ripigneva là dove 'l sol tace.}
\end{center}

La lupa faceva provare a Dante la stessa sensazione dell'ultima terzina,
riportandolo verso la selva (dove il sole non splende).

È presente una sinestesia (dove 'l sol tace).

\begin{center}
    \textit{Mentre ch'i' rovinava in basso loco,} \\
    \textit{dinanzi a li occhi mi si fu offerto} \\
    \textit{chi per lungo silenzio parea fioco.}
\end{center}

Mentre Dante rotolava verso il basso, incontra \textbf{Virgilio}.
La sua voce era bassa perché non aveva parlato per molto tempo.
Questo indica anche che la sua parola non veniva ascoltata da molto,
esso rappresenta infatti la ragione umana.

\sproposition{Virgilio}{
    \textit{Virgilio} fu un poeta
    vissuto tra il 70 a.C e il 19 a.C.
}

\begin{center}
    \textit{Quando vidi costui nel gran diserto,} \\
    \textit{«Miserere di me», gridai a lui,} \\
    \textit{«qual che tu sii, od ombra od omo certo!».}
\end{center}

Questa è la prima volta che qualcuno parla.

Dante chiede \quotes{Abbi pietà di me. Chiunque tu sia, anima o uomo}.

\begin{center}
    \textit{Rispuosemi: «Non omo, omo già fui,} \\
    \textit{e li parenti miei furon lombardi,} \\
    \textit{mantoani per patrïa ambedui.}
\end{center}

Nel medioevo le persone si presentavano con la loro provenienza geografica.
Ciò indica il nome della propria famiglia e l'appartenenza politica.

La Lombardia era tutta l'Italia del Nord.
I genitori erano mantovani.

\begin{center}
    \textit{Nacqui sub Iulio, ancor che fosse tardi,} \\
    \textit{e vissi a Roma sotto 'l buono Augusto} \\
    \textit{nel tempo de li dèi falsi e bugiardi.}
\end{center}

L'anima è nata durante il periodo di Giulio Cesare, ma visse
a Roma sotto Augusto, a seguito della morte di Cesare nel 44 a.C.

Virgilio ha vissuto in un periodo di Dei pagani (siccome Cristo non era ancora nato).

\begin{center}
    \textit{Poeta fui, e cantai di quel giusto} \\
    \textit{figliuol d'Anchise che venne di Troia,} \\
    \textit{poi che 'l superbo Ilïón fu combusto.}
\end{center}

Virgilio celebrò di Enea (Eneide) dopo che la fortezza fu bruciata.
Qui termina la presentazione.

È importante notare che la salvezza di Dante deriva da un poeta.
La poesia era cruciale nel mondo medievale.

\begin{center}
    \textit{Ma tu perché ritorni a tanta noia?} \\
    \textit{perché non sali il dilettoso monte} \\
    \textit{ch'è principio e cagion di tutta gioia?».}
\end{center}

In italiano antico la noia indica tormento.

\hr

Nonostante ci si aspetterebbe la risposta di Dante, esso
riconosce Virgilio e lo elogia con le seguenti 3 terzine:

\begin{center}
    \textit{«Or se' tu quel Virgilio e quella fonte} \\
    \textit{che spandi di parlar sì largo fiume?»,} \\
    \textit{rispuos' io lui con \textbf{vergognosa} fronte.}
\end{center}

Dante si rivolge a Virgilio con vergogna, sentimento di deferenza e rispetto.

\begin{center}
    \textit{«O de li altri poeti onore e lume,} \\
    \textit{vagliami 'l lungo studio e 'l grande amore} \\
    \textit{che m'ha fatto cercar lo tuo volume.}
\end{center}

Dante dice di avere studiato la sua opera, e dichiara un debito poetico.
L'amore poetico di Dante l'ha portato a studiare (a memoria) l'Eneide.
Infatti, molte espressioni nella Commedia sono riprese dall'Eneide.

\begin{center}
    \textit{Tu se' lo mio maestro e 'l mio autore,} \\
    \textit{tu se' solo colui da cu' io tolsi} \\
    \textit{lo bello stilo che m'ha fatto onore.}
\end{center}

\hr

\begin{center}
    \textit{Vedi la bestia per cu' io mi volsi;} \\
    \textit{aiutami da lei, famoso saggio,} \\
    \textit{ch'ella mi fa tremar le vene e i polsi».}
\end{center}

La lonza e il leone non vengono nemmeno più nominati, \quotes{la bestia} è quella più difficile.

\begin{center}
    \textit{«A te convien tenere altro vïaggio»,} \\
    \textit{rispuose, poi che lagrimar mi vide,} \\
    \textit{«se vuo' campar d'esto loco selvaggio;}
\end{center}

Il primo verso di questa terzina è quello più importante di tutto il canto.
Virgilio risponde indicando un altro percorso da compiere

\begin{center}
    \textit{ché questa bestia, per la qual tu gride,} \\
    \textit{non lascia altrui passar per la sua via,} \\
    \textit{ma tanto lo 'mpedisce che l'uccide;}
\end{center}

la lupa non lascia passare \textit{nessuno}.
Indugiare qui significa morire.

\begin{center}
    \textit{e ha natura sì malvagia e ria,} \\
    \textit{che mai non empie la bramosa voglia,} \\
    \textit{e dopo 'l pasto ha più fame che pria.}
\end{center}

Viriglio descrive ulteriorment la lupa.
La lupa ha ancora più fame dopo aver mangiato, questa è la cupidigia.

\begin{center}
    \textit{Molti son li animali a cui s'ammoglia,} \\
    \textit{e più saranno ancora, infin che 'l veltro} \\
    \textit{verrà, che la farà morir con doglia.}
\end{center}

Molte sono le vittime di questo peccato, ma Virgilio profetizza che il veltro sia l'unico
a poterla superare.
Possiamo capire che questa sia una profezia dal tempo futuro e linguaggio enigmatico.

\begin{center}
    \textit{Questi non ciberà terra né peltro,} \\
    \textit{ma sapïenza, amore e virtute,} \\
    \textit{e sua nazion sarà tra feltro e feltro.}
\end{center}

%  TODO Dieresi - spezza le due vocali

Il veltro non si ciberà nè di terra nè di peltro (lega metallica delle monete).
Non avrà quindi fame di ricchezza materiale.
Invece, si ciberà di sapienza, amore e virtù (i 3 attributi della trinità).

Questo personaggio nascerà fra \textit{feltro} e \textit{feltro} (un panno, tessuto).
la prima interpretazione è quella di interpretare il feltro come un panno vile.

\sdefinition{Veltro}{
    Il \textit{veltro} è un cane da caccia.
    Nella letteratura italiana, rappresenta un'azione di riforma,
    evidentemente promossa da Dio, che perseguiti la cupidigia in tutte le sue forme
    ristabilendo in tutto il mondo ordine e giustizia. 
}

La lupa potrebbe essere sconfitta da un uomo di quella chiesa (probabilmente dei Francescana),
che si occupa dei malati ed è umile.

Un'altra interpretazione vede il \textit{feltro} come il cielo,
mentre un'altra lo collega alla collocazione geografia di Verona (la quale si
situa tra Feltre e Montefeltro), per cui nascerà a Verona.

Un'ulteriore interpretazione, quella più accreditata, dice che esso nascerà da un'elezione imperiale
(probabilmente Arrigo VIII), siccome l'urna veniva foderata di feltro all'interno.

\begin{center}
    \textit{Di quella umile Italia fia salute} \\
    \textit{per cui morì la vergine Cammilla,} \\
    \textit{Eurialo e Turno e Niso di ferute.}
\end{center}

\textbf{\color{red}nota:} fia = sarà. \\
Virgilio sta dicendo chde il veltro sarà la salvezza dell'Italia.

\begin{center}
    \textit{Questi la caccerà per ogne villa,} \\
    \textit{fin che l'avrà rimessa ne lo 'nferno,} \\
    \textit{là onde 'nvidia prima dipartilla.}
\end{center}

Questa terzina termina la profezia.
Il veltro sconfiggerà la lupa cacciandola ovunque fino all'inferno.
L'ultimo verso sembrerebbe indicare il primo momento in cui la lupa è stata scagliata
fra gli uomini, a seguito dell'invidia del demonio nei confronti di Dio.

\begin{center}
    \textit{Ond' io per lo tuo me' penso e discerno} \\
    \textit{che tu mi segui, e io sarò tua guida,} \\
    \textit{e trarrotti di qui per loco etterno;}
\end{center}

Virgilio dice che per il meglio di Dante, è auspicabile che lui lo segua.
Questa è infatti la saggezza di Virgilio.
Dante verrà salvato e portato via attraverso un luogo eterno (l'inferno).

\begin{center}
    \textit{ove udirai le disperate strida,} \\
    \textit{vedrai li antichi spiriti \textbf{dolenti},} \\
    \textit{ch'a la seconda morte ciascun grida;}
\end{center}

Queste terzine descrivono quindi l'inferno, dove Dante passerà.
Esso viene rappresentato come pieno di anime dannate.

La prima morta è quella fisica, mentre la seconda morte è quella dell'anima che
diventa dannata.

\begin{center}
    \textit{e vederai color che son \textbf{contenti}} \\
    \textit{nel foco, perché speran di venire} \\
    \textit{quando che sia a le beate genti.}
\end{center}

Le anime dannate sono distrutte nel dolore perché sanno che quella è la loro
fine eterna, mentre le anime nel purgatorio sono contente di scontare la propria pena,
perché sanno che essa avrà una fine (fino all'Apocalisse).

Questo può essere sintetizzato dalla rima \quotes{dolenti}:\quotes{contenti}.

\begin{center}
    \textit{A le quai poi se tu vorrai salire,} \\
    \textit{anima fia a ciò più di me degna:} \\
    \textit{con lei ti lascerò nel mio partire;}
\end{center}

Se Dante vorrà salire fra le genti beate (paradiso), non potrà farlo con Virgilio
ma con un'altra persona (Beatrice).

\begin{center}
    \textit{A le quai poi \textbf{se} tu vorrai salire,} \\
    \textit{anima fia a ciò più di me degna:} \\
    \textit{con lei ti lascerò nel mio partire;}
\end{center}

Dante avrà la scelta di percorrere anche i cieli. Non è infatti necessario
farlo per essere salvi. Dante sarà salvo dopo aver percorso il purgatorio.

\begin{center}
    \textit{ché quello imperador che là sù regna,} \\
    \textit{perch' i' fu' ribellante a la sua legge,} \\
    \textit{non vuol che 'n sua città per me si vegna.}
\end{center}

Il motivo per cui Dante dovrà essere seguire da un'altra anima
è perché Virgilio è un dannato (non è battezzato). 
Virgilio non può nominare direttamente Dio essendo tale.

% virtù cardinali e teologate

\begin{center}
    \textit{In tutte parti impera e quivi regge;} \\
    \textit{quivi è la sua città e l'alto seggio:} \\
    \textit{oh felice colui cu' ivi elegge!».}
\end{center}

Beato colui che può accedere in paradiso.

\begin{center}
    \textit{E io a lui: «Poeta, io ti richeggio} \\
    \textit{per quello Dio che tu non conoscesti,} \\
    \textit{a ciò ch'io fugga questo male e peggio,}
\end{center}

A differenza di Virgilio, Dante può pronunciare il nome di Dio.
Nel nome del Dio che purtroppo Virgilio non potrà conoscere, Dante accetta.

\begin{center}
    \textit{che tu mi meni là dov' or dicesti,} \\
    \textit{sì ch'io veggia la porta di san Pietro} \\
    \textit{e color cui tu fai cotanto mesti».}
\end{center}

Dante ripassa i posti che traverserà con Virgilio, ma in ordine opposto.
La porta di san Pietro rappresenta il purgatorio, mentre le anime dannate
sono nell'inferno.

\begin{center}
    \textit{Allor si mosse, e io li tenni dietro.}
\end{center}

Come il canto comincia con un cammino (metaforico), viene terminato
con un cammino (fisico).

\pagebreak

\subsubsection{Inferno, Canto II}

Il secondo canto serve da proemio al libro dell'Inferno.
Tutto il secondo canto si svolge nel medesimo punto dove il primo canto termina.
Infatti, Dante e Virgilio non si muovono.

\sdefinition{Invocazione alle Muse}{
    Nel ricevere e far suo il tradizionale uso retorico d'invocare le Muse, quando più arduo si presenti l'impegno dell'arte.
}
Dante invoca le Muse, ingegno e memoria per quello che sta per scrivere.

A dante giungono dei dubbi, ossia quale sia lo scopo del suo viaggio
(la risposta verrà data in Paradiso XVII) e chi gli permetta di fare tale.
Perché Dante ha il permesso di compiere questo viaggio?
\\
Lo stesso viaggio è stato compiuto solamente da San Paolo (Nuovo Testamento)
e Enea (Eneide), uno ai cieli e uno agli inferi.

Virgilio risponde alla seconda domanda mediante le seguenti affermazioni.
È stata Beatrice ad avvisarlo che Dante fosse in difficoltà.
La Madonna ha visto la difficoltà di Dante, l'ha detto a Santa Lucia, che va da Beatrice,
la quale scende da Virgilio.
Per cui, il viaggio è permesso da 3 donne benedette.
\begin{center}
    \begin{tikzcd}
        \text{Madonna} \arrow[r] & \text{Santa Lucia} \arrow[r] & \text{Beatrice} \arrow[d] \\
        & \text{Dante}                 & \text{Virgilio} \arrow[l]
    \end{tikzcd}
\end{center}
Il motivo per cui Beatrice ha scelto Virgilio è per la sua ragione e uso della parola.

Beatrice si sente infatti in Debito con Dante per la sua gratitudine (\textit{Vita Nova}).

\pagebreak

\subsubsection{Inferno, Canto III}

\hr

Le seguenti 3 terzine sono le incisioni sulla porta dell'inferno.

\begin{center}
    \textit{"Per me si va ne la città \textbf{dolente},} \\
    \textit{per me si va ne l'etterno \textbf{dolore},} \\
    \textit{per me si va tra la \textbf{perduta gente}.}
\end{center}

La porta parla al singolare con un'anafora di \quotes{Per me si va}.
La prima terzina esprime il dolore che è presente nel posto.

\begin{center}
    \textit{Giustizia mosse il mio alto fattore;} \\
    \textit{fecemi la divina podestate,} \\
    \textit{la somma sapïenza e 'l primo amore.}
\end{center}

Questi sono i tre attributi della trinità (Io, inferno, sono fatto da Dio).
Tutto quello che si vedrà è quindi un luogo giusto, perché Dio è giusto per definizione.
Nell'inferno manca quindi arbitrio.

\begin{center}
    \textit{Dinanzi a me non fuor cose create} \\
    \textit{se non etterne, e io etterno duro.} \\
    \textit{Lasciate ogne speranza, voi ch'intrate".}
\end{center}

L'inferno è sempiterno.

\hr

\begin{center}
    \textit{Queste parole di colore oscuro} \\
    \textit{vid'ïo scritte al sommo d'una porta;} \\
    \textit{per ch'io: "Maestro, il senso lor m'è duro".}
\end{center}

Le scritte sono di colore nero, scure e hanno la funzione opposte delle scritte
che c'erano nei portoni delle chiese medievali, le quali invitavano i fedeli ad entrare.

\begin{center}
    \textit{Ed elli a me, come persona accorta:} \\
    \textit{"Qui si convien lasciare ogne sospetto;} \\
    \textit{ogne viltà convien che qui sia morta.}
\end{center}

Virgilio dice a Dante che deve lasciare ogni sospetto (esitazione).
L'animità non ha più il tempo di esitare.

Dante è un uomo con le sue debolezze ma che necessita umanamente di un uomo che lo aiuti con la sua paura.

\begin{center}
    \textit{Noi siam venuti al loco ov'i' t' ho detto} \\
    \textit{che tu vedrai le genti dolorose} \\
    \textit{c' hanno perduto il ben de l'intelletto". }
\end{center}

\begin{center}
    \textit{E poi che la sua mano a la mia puose} \\
    \textit{con lieto volto, ond'io mi confortai,} \\
    \textit{mi mise dentro a le segrete cose. }
\end{center}

Questo ultimo verso è l'ultimo alla luce del sole.

\begin{center}
    \textit{Quivi \textbf{sospiri}, \textbf{pianti} e \textbf{alti guai}} \\
    \textit{risonavan per l'aere sanza stelle,} \\
    \textit{per ch'io al cominciar ne lagrimai. }
\end{center}

Le prime percezioni sono acustiche siccome Dante non vede quasi nulla (area senza stelle).
Per tutto l'inferno Dante proverà compassione dei dolenti.

Climax: \textbf{sospiri}, \textbf{pianti} e \textbf{alti guai}.

\begin{center}
    \textit{Diverse \textbf{lingue}, orribili \textbf{favelle},} \\
    \textit{\textbf{parole} di dolore, \textbf{accenti} d'ira,} \\
    \textit{voci alte e fioche, e suon di man con elle }
\end{center}

Anti-climax di dettaglio: \textbf{lingue}, \textbf{favelle}, \textbf{parole}, \textbf{accenti} (suoni).

\begin{center}
    \textit{facevano un tumulto, il qual s'aggira} \\
    \textit{sempre in quell'aura sanza tempo tinta,} \\
    \textit{come la rena quando turbo spira.}
\end{center}

Tutti i suoini giungono a Dante come se fossero un vortice (tromba d'aria) di sabbia.

\begin{center}
    \textit{E io ch'avea d'error la testa cinta,} \\
    \textit{dissi: "Maestro, che è quel ch'i' odo?} \\
    \textit{e che gent'è che par nel duol sì vinta?". }
\end{center}

\begin{center}
    \textit{Ed elli a me: "Questo misero modo} \\
    \textit{tegnon l'anime triste di coloro} \\
    \textit{che visser \textbf{sanza 'nfamia e sanza lodo}. }
\end{center}

\sdefinition{I pusillanimi}{
    I \textit{pusillanimi} sono coloro che rifiutano la loro identità di uomo,
    sono timidi e non hanno coraggio e determinazione, per cui non esercitano il libero arbitrio.
}

Virgilio risponde che questo misero modo di lamentarsi appartiene
alle anime dei pusillanimi (ignavi), ossia coloro che non hanno preso decisioni
nella loro vita, non hanno commesso nè bene (lodo) nè male (infamia).

La loro colpa è quella di non aver esercitato il libero arbitrio di Dio,
per cui rinunciare alla identità più profonda di uomo e vivere come un animale.

\begin{center}
    \textit{Mischiate sono a quel cattivo coro} \\
    \textit{de li angeli che \textbf{non furon ribelli}} \\
    \textit{\textbf{né fur fedeli} a Dio, ma per sé fuoro.}
\end{center}

Vi è un gruppo di angeli che sta a sè (non sta nè con Dio nè con il Demonio), angeli che non si schierarono.

Questa categoria non è riconosciuta dalle Scrittura ma riconosciuta dalla cultura popolare.

\begin{center}
    \textit{\textbf{Caccianli i ciel} per non esser men belli,} \\
    \textit{\textbf{né lo profondo inferno li riceve},} \\
    \textit{ch'alcuna gloria i rei avrebber d'elli».}
\end{center}

Virgilio smette di parlare dicendo che queste anime sono cacciate dai cieli
ma nemmeno nell'inferno profondo. Ribadisce 3 volte che non sono nè da un lato nè dall'altro.

\begin{center}
    \textit{E io: «Maestro, che è tanto greve} \\
    \textit{a lor che lamentar li fa sì forte?».} \\
    \textit{Rispuose: «Dicerolti molto breve.}
\end{center}

\begin{center}
    \textit{Questi non hanno speranza di morte,} \\
    \textit{e la lor cieca vita è tanto bassa,} \\
    \textit{che 'nvidïosi son d'ogne altra sorte.}
\end{center}

Le anime del profondo inferno, potrebbero vantarsi con i pusillani perché loro non hanno un peccato.
Un'anima dannata che vede qualcuno che non ha commesso il male nella sua stessa posizione,
potrebbe sminuire i propri peccati e fare meno importanza alla propria colpa.
Questo è il motivo per cui i pusillani sono espulsi.

Virgilio ritorna una spiegazione molto breve, come se loro non si meritassero nemmeno
più parole. Tutte le anime sperano di essere annichilite, ma questi
hanno una vita cieca tanto bassa e piccola che sono invidiosi di tutti.

\pagebreak

\begin{center}
    \textit{Fama di loro il mondo esser non lassa;} \\
    \textit{misericordia e giustizia li sdegna:} \\
    \textit{non ragioniam di lor, ma guarda e passa».}
\end{center}

Queste anime non mancheranno al mondo e non verranno ricordati.
Dante e Virgilio hanno un grosso disprezzo verso questi spiriti.
Non meritano nè parola nè tempo.

\begin{center}
    \textit{E io, che riguardai, vidi una 'nsegna} \\
    \textit{che girando correva tanto ratta,} \\
    \textit{che d'ogne posa mi parea indegna;}
\end{center}

Dante vede una bandiera che corre molto rapida, la quale viene seguita
da una massa di gente. È sorpreso che ci siano tanti pusillanimi.

\begin{center}
    \textit{che girando correva tanto ratta,} \\
    \textit{di gente, ch'i' non averei creduto} \\
    \textit{che morte tanta n'avesse disfatta.}
\end{center}

Viene quindi mostrata la prima relazione fra pena e colpa per contrappasso,
in questo caso, per opposizione. Infatti, le anime devono sepre seguire il vessillo (simbolo dello schieramento).
Così come in vita non si sono mai schierati, ora devono schierarsi per sempre.

\begin{center}
    \textit{Poscia ch'io v'ebbi alcun riconosciuto,} \\
    \textit{vidi e conobbi l'ombra di colui} \\
    \textit{che fece per viltade il gran rifiuto.}
\end{center}

Dante ne riconosce alcuni, in particolare l'anima di colui
che per pusillanimità fece un gran rifiuto.

Si riferisce probabilmente al papa Celestino V, il quale si è dimesso
dopo qualche mese. Le sue dimissioni fecero eleggere Bonifacio VIII, il quale manderà Dante in esilio.
A supporto di questa tesi vi è la frase "vidi e conobbi", il personaggio è quindi un contemporaneo
di Dante. Inoltre, già tutti i commentatori antichi, i quali erano vinici a quest'epoca,
hanno riferito questo papa.

Dante avrebbe potuto scrivere il canto prima che il papa diventasse santo
perché forse non avrebbe messo un santo all'inferno.
D'altro canto non sarebbe un eresia farlo, in quanto è una scelta della chiesa.

\begin{center}
    \textit{Incontanente intesi e certo fui} \\
    \textit{che questa era la setta d'i cattivi,} \\
    \textit{a Dio spiacenti e a' nemici sui.}
\end{center}

\begin{center}
    \textit{Questi sciaurati, che mai non fur vivi,} \\
    \textit{erano ignudi e stimolati molto} \\
    \textit{da mosconi e da vespe ch'eran ivi.}
\end{center}

Questa è la seconda parte della pena: erano nudi (come tutte le anime)
e punti da vespe e mosconi.

\begin{center}
    \textit{Elle rigavan lor di sangue il volto,} \\
    \textit{che, mischiato di lagrime, a' lor piedi} \\
    \textit{da \textbf{fastidiosi} vermi era ricolto.}
\end{center}

Questo è un contrappasso per analogia. Così come in vita sono stati insignificanti,
ora sono tormentati piccoli insetti insegnificanti come loro (mosconi vespe e vermi).

Lacrime e sangue indicano uno sforzo, quello sforzo che loro non hanno mai compiuto.

\begin{center}
    \textit{E poi ch'a riguardar oltre mi diedi,} \\
    \textit{vidi genti a la riva d'un gran fiume;} \\
    \textit{per ch'io dissi: «Maestro, or mi concedi}
\end{center}

\begin{center}
    \textit{ch'i' sappia quali sono, e qual costume} \\
    \textit{le fa di trapassar parer sì pronte,} \\
    \textit{com' i' discerno per lo fioco lume».}
\end{center}

Guardando in un'altra direzione, oltre un fiume, vede molte genti e chiede a Virgilio
chi sono e perché sembrino (dalla poca luca) desiderose di attraversare la riva.

\begin{center}
    \textit{Ed elli a me: «Le cose ti fier conte} \\
    \textit{quando noi fermerem li nostri passi} \\
    \textit{su la trista riviera d'Acheronte».}
\end{center}

Virgilio indica che gli darà la risposta quando arriveranno là.
Il primo fiume infernale si chiama Acheronte.

\begin{center}
    \textit{Allor con li occhi vergognosi e bassi,} \\
    \textit{temendo no 'l mio dir li fosse grave,} \\
    \textit{infino al fiume del parlar mi trassi.}
\end{center}

Dante, temendo di essere stato inopportuno, fa silenzio sino all'arrivo al fiume.

\begin{center}
    \textit{Ed ecco verso noi venir per nave} \\
    \textit{un vecchio, bianco per antico pelo,} \\
    \textit{gridando: «Guai a voi, anime prave!}
\end{center}

Giunge verso Dante un vecchio con barba e capelli bianchi su una nave che grida.
Esso è Caronte ed è il traghettatore di anime.

\begin{center}
    \textit{\textbf{Non isperate} mai veder lo cielo:} \\
    \textit{i' vegno per menarvi a l'altra riva} \\
    \textit{ne le tenebre \textbf{etterne}, in caldo e 'n gelo.}
\end{center}

Le parole di Caronte replicano l'iscrizione della porta dell'inferno.
Il personaggio è già presente con il medesimo ruolo nell'Eneide.

\begin{center}
    \textit{E tu che se' costì, anima viva,} \\
    \textit{pàrtiti da cotesti che son morti».} \\
    \textit{Ma poi che vide ch'io non mi partiva,}
\end{center}

Caronte si ritrova davanti un vivo (Dante) e gli dice che lui non è un dannato, e che quindi se ne deve andare.

\begin{center}
    \textit{disse: «Per altra via, per altri porti} \\
    \textit{verrai a piaggia, non qui, per passare:} \\
    \textit{più lieve legno convien che ti porti».}
\end{center}

Una volta visto che Dante non si muoveva, gli dice
"[Quando sarà il tuo momento], farai un altro percorso su una barca più leggera".
Per cui, non è un'anima dell'inferno.

\begin{center}
    \textit{E 'l duca lui: «Caron, non ti crucciare:} \\
    \textit{vuolsi così colà dove si puote} \\
    \textit{ciò che si vuole, e più non dimandare».}
\end{center}

Virgilio interviene e dice a Caronte di far passare Dante
siccome il suo viaggio è voluto da Dio, e per cui Caronte non deve interferire.

\begin{center}
    \textit{Quinci fuor quete le lanose gote} \\
    \textit{al nocchier de la livida palude,} \\
    \textit{che 'ntorno a li occhi avea di fiamme rote.}
\end{center}

% nocchier = tragetatore
Caronte ha delle fiamme attorno agli occhi. Il suo ritratto temrina qui, e lui si tranquillizza.

\begin{center}
    \textit{Ma quell' anime, ch'eran lasse e nude,} \\
    \textit{cangiar colore e dibattero i denti,} \\
    \textit{ratto che 'nteser le parole crude.}
\end{center}

A differenza di Dante, le anime dannate sanno che quelle parole erano rivolte a loro.

\begin{center}
    \textit{Bestemmiavano \textbf{Dio} e lor \textbf{parenti},} \\
    \textit{\textbf{l'umana spezie} e 'l \textbf{loco} e 'l \textbf{tempo} e 'l \textbf{seme}} \\
    \textit{di lor semenza e di lor nascimenti.}
\end{center}

Anti-climax: \textbf{Dio}, \textbf{parenti}, \textbf{l'umana spezie}, \textbf{tempo}
e \textbf{seme}. \\
Le anime maledicono il giorno della loro nascita.

\begin{center}
    \textit{Poi si ritrasser tutte quante insieme,} \\
    \textit{forte piangendo, a la riva malvagia} \\
    \textit{ch'attende ciascun uom che Dio non teme.}
\end{center}

I pusillanimi si trovano nell'antinferno, uno spazio fra la porta e il fiume.
L'inferno vero comincia infatti dopo il fiume.


\begin{center}
    \textit{Caron dimonio, con occhi di bragia} \\
    \textit{loro accennando, tutte le raccoglie;} \\
    \textit{batte col remo qualunque s'adagia.}
\end{center}

Caronte viene anche definito come un demonio con occhi di brace.
Con un cenno raduna tutte le anime e partono in barca.
Caronte picchia con il remo chi anche solo minimanete provi ad adagiarsi.

\begin{center}
    \textit{Come d'autunno si levan le foglie} \\
    \textit{l'una appresso de l'altra, fin che 'l ramo} \\
    \textit{vede a la terra tutte le sue spoglie,}
\end{center}

Come d'autunno un albero vede tutte le sue foglie cadere,

\begin{center}
    \textit{similemente il mal seme d'Adamo} \\
    \textit{gittansi di quel lito ad una ad una,} \\
    \textit{per cenni come augel per suo richiamo.}
\end{center}

similmente, ad una a una, le anime salgono sulla nave di Caronte.
La similitudine presenta anche un segno di rassegnazione.

\begin{center}
    \textit{Così sen vanno su per l'onda bruna,} \\
    \textit{e avanti che sien di là discese,} \\
    \textit{anche di qua nuova schiera s'auna.}
\end{center}

Caronte non fa a tempo a portare le anime dall'altra parte,
che un nuovo gruppo di anime si raggruppa nuovamente.

\hr

Ecco finalmente le due risposte di Virgilio.

\begin{center}
    \textit{«Figliuol mio», disse 'l maestro cortese,} \\
    \textit{«quelli che muoion ne l'ira di Dio} \\
    \textit{tutti convegnon qui d'ogne paese;}
\end{center}

\begin{center}
    \textit{e pronti sono a trapassar lo rio,} \\
    \textit{ché la divina giustizia li sprona,} \\
    \textit{sì che la tema si volve in disio.}
\end{center}

Ogni uomo, pur dannato e peccatore (eccetto i pusillanimi),
conserva la coscienza, e ritiene la pena giusta in quanto riesce a distinguere fra bene e male,
nonostante la prorpia paura.

\begin{center}
    \textit{Quinci non passa mai anima buona;} \\
    \textit{e però, se Caron di te si lagna,} \\
    \textit{ben puoi sapere omai che 'l suo dir suona».}
\end{center}

Il fatto che Caronte non voglia che Dante sia lì, indica che Dante non ci passerà dopo la morte.

\hr

\begin{center}
    \textit{Finito questo, la buia campagna} \\
    \textit{tremò sì forte, che de lo spavento} \\
    \textit{la mente di sudore ancor mi bagna.}
\end{center}

Vi è un terremoto molto spaventoso, che spaventa ancora Dante scrittore.

\begin{center}
    \textit{La terra lagrimosa diede vento,} \\
    \textit{che balenò una luce vermiglia} \\
    \textit{la qual mi vinse ciascun sentimento;}
\end{center}

\begin{center}
    \textit{e caddi come l'uom cui sonno piglia.} \\
\end{center}

Dante sviene dalla grandissima luce intensa dopo il terremoto.

\subsubsection{Inferno, Canto IV}

Attraversato il fiume vi è il Limbo, ossia la zona di Virgilio.

\pagebreak

\subsection{Caronte Virgiliano e Dantesco}

\sproposition{Caronte}{
    \textit{Caronte} è un personaggio che ha il ruolo di tragettare le anime
    oltre il primo fiume dell'inferno.
}

Vi sono delle differenza fra la descrizione di Caronte nell'\textit{Eneide}
e la \textit{Commedia}.

In ambo i casi, Caronte è un vecchio tragettatore che viene descritto come un personaggio con
delle caratteristiche comuni.
Entrambi sono caratterizzati da dei capelli binachi (rr. 299-300, r. 83) e posseggono
degli occhi di fiamme. Dante indica tuttavia quest'ultima caratteristica 2 volte.
Il ruolo del personaggio è il medesimo; Caronte ha ruolo di ostacolo, ferma chiunque voglia passare
(rr. 338-339, rr. 88-89).

Nella \textit{Commedia}, la barca e Caronte stesso è descritto con più precisione
rispetto all'\textit{Eneide}.
\snote{Descrizioni di Dante}{
    Dante usa più verbi che aggettivi nelle varie descrizioni.
}
Dante descrive Caronte principalmente per quello che fa piuttosto che come è fatto.

Il Caronte Dantesto è più aggressivo (urla), sintatticamente le sue frasi sono
più brevi e nette e viene descritto come indemoniato. Infatti, anche la sua entrata in scena
è improvvisa e violenta. Questa caratteristica non è presente nel Caronte virgiliano,
il quale è più pacato, e descritto come un Dio ma meno importante.

La differenza più importante è che il Caronte dell'\textit{Eneide} fa parte di un mondo pagano.
Non vi è ancora la concezione di salvezza dell'anima, la quale deriva dal cristianesimo.

\pagebreak

\part{Il Duecento e il Trecento}

\subsection{Passagio fra il Duecento e Trecento}

Dante (1265-1321) e Boccaccio (1313 - 1375)
si distinguono per le caratteristiche del cambio di secolo.

\begin{itemize}
    \item Fino ai primi anni del secolo, si assiste ad un aumento della popolazione, favorito dall'accresciuto benesseere e dalle migliorate condizioni sanitarie.
    \item I comuni si espandono sempre di più, grazie alla migliori condizioni economiche e all'immigrazione dal contado. La città diviene il vero e proprio centro della vita: civile, sociale, economica, culturale.
    \item L'esempio di Firenze e l'ascesa della borghesia mercantile.
\end{itemize}

Firenze aveva i fiorini ed era economicamente forte.
Molti poeti e scrittori sono appunto provenienti da paesi del genere.

A differenza di Dante, Boccaccia e Petrarca si distaccano dai loro comuni di appartenzenza,
mentre Dante centralizza Firenze nei suoi temi.

La società comincia lentamente a diventare parzialmente laica.
Il mondo passa dalla trascendenza (visione verticale, tutto è subordinato a Dio)
ad un mondo di immanenza (ciò che è sulla terra).
La teologia nelle università rimane la materia primaria e fondamentale. Medicina è l'unica materia autonoma, ma non è possibile sezionare i cadaveri (corpo sacro).

\subsection{Crisi economica e demografica}

La produzione agricola entra progressivamente in crisi man mano che ci si inoltra nel Trecento.
E una decisa inversione di tendenza rispetto al secolo precedente.\\
In questa situazione già deteriorata, la diffusione in tutta Europa dellla peste nera (1347-1350) provocò
un tracollo economico e una vera e propria crisi demografia (la popolazione si riduce di almeno un terzo).

% Todo le altre cose

\pagebreak

\part{Boccaccio (1313 - 1375)}

\section{Decameron}

\subsection{Introduzione}

L'opera è ambientata durante l'epidemia di peste del 1348 a Firenze
e segue un gruppo di dieci giovani aristocratici (sette donne e tre uomini)
che cercano rifugio nella campagna toscana per sfuggire alla peste.
Per passare il tempo, ciascuno di loro racconta una novella ogni giorno,
per un totale di cento novelle in dieci giorni (14 giorni totali).
Queste novelle spaziano in temi e argomenti, toccando la vita, l'amore, l'umorismo,
l'ingiustizia sociale, la morale, la religione e molti altri aspetti della società medievale.

L'opera è strutturata nella seguente maniera:
\begin{itemize}
    \item \textbf{Introduzione e Proemio:} Inizia con una breve introduzione che spiega le circostanze della fuga dei giovani dalla città e il loro soggiorno in campagna. Il proemio presenta anche i temi principali e lo scopo dell'opera.
        Boccaccio parla della propria sofferenza amorosa, e scrive il Decameron come omaggio alle donne (borghesi, sensibili, che non posso distrarsi essendo confinate in casa).
    \item \textbf{Dieci Giornate:} Ogni giornata rappresenta una sezione dell'opera in cui i personaggi raccontano le novelle. Ogni giornata è caratterizzata da un tema generale o un'idea predominante scelta da un re o una regina che viene eletto ogni giorno.
    \item \textbf{Novelle:} Ogni giornata contiene dieci novelle narrate dai personaggi, ciascuna di esse seguendo il tema giornaliero. Le novelle sono scritte in prosa ed esprimono la varietà delle esperienze umane.
    \item \textbf{Conclusione:} L'opera si conclude con una breve conclusione in cui Boccaccio parla dell'importanza dell'amicizia, dell'amore e della sorte.
\end{itemize}

\sdefinition{Metanarrazione}{
    La \textit{metanarrazione} è una tecnica narrativa,
    che consiste nell'intervento diretto dell'autore all'interno dello stesso testo che va componendo
    Si verifica così una narrazione che assume come proprio oggetto l'atto stesso del raccontare,
    così da sviluppare un romanzo nel romanzo.
}

\subsection{Analisi}

I primi 7 paragrafi [1-7] sono una scusa rivolte alle donne. Viene delineata la necessità di 
avere una salita di sofferenza prima del piacere.
Boccaccio spiega che l'introduzione (sofferente) è necessaria per capire come si arriva alle novelle. \\
La storia vera e propria comincia all'ottavo paragrafo [8] mediante l'arrivo della peste.
Questa peste, deriva geograficamente dall'Oriente prima di giungere a Firenze
(è stata trasportata dai topi sulle barche mercantili).
Secondo Boccaccio, è possibile che la pesta sia una punizione divina inflitta sugli uomini.
Viene descritto come la peste influenzasse il corpo, come lo facesse gonfiare e come i sintomi evolvessero per poi giungere alla morte.
Successicamente [13] si parla dello scopo dei medici; nessuna medicina riusciva a contrastare questa piaga, la quale è contagiosa [13]
anche fra uomini e animali [17].
Dopo aver descritto in dettaglio le dinamiche della malattia, vengono descritte
le reazioni delle persone [19]: 1. Forse il modo migliore è quello di vivere con parsimonia ma con cibi e vini di qualità
2. Fare festa e concedersi gli eccessi.
Le leggi divine e leggi umane sono finite, in quanto tutti sono affetti dal caos [23].
\\
Nei paragrafi [53-65] vi è l'inizio del lungo discorso di Pampinea, la quale arriva a proporre alle altre 6 conoscenti
di uscire da Firenze. Essa insiste molto che ciò non è una scelta di disimpegno,
al massimo sono loro ad essere il frutto dell'abbandono.
La proposta è di uscita, che tuttavia non è fuga nel divertimento, bensì di vita onesta nella campagna (contado),
onestà che non c'è più nella città. Lo scopo è quello di fare festa, provare piacere ed essere allegri,
ma \quotes{senza trapassare in alcun atto il segno della ragione}.
\sdefinition{Locus amoenus}{
    Un luogo naturale, verdeggiante, paradisiaco come edenico.
}
Nei seguenti paragrafi [-67] viene descritto un locus amoenus.
Queata descrizione è data come opposto al luogo di Firenze.
Questa non è una proposta di abbandona o fuga definitiva della città [71],
ma vedranno che cosa gli risparmia il cielo se rimarranno vivi.
Ma Filomena [74], la quale era molto discreta, ricorda che un gruppo di donne non può gestirsi
senza un uomo che le guidi, quanto le donne sono litigiose, sospettose, pusillanimi e paurosa.
Filomena le da ragione, ma Elissa fa notare che il progetto dell'onestà cadrebbe dall'inizio se
prendessero degli uomini non legati a loro (vivere sotto il medesimo tetto sarebbe uno scandalo).
Per pura coincidenza [78] 3 uomini entrano nella chiesa. Gli uomini sono chiamati Panfilo, Filostrato e Dioneo
e sono belli e di belle maniere, come lo sono le donne.
Questi 3 non sono completi estranei perché sono innamorati di alcune delle ragazze.
Neifile [81], l'amata di una dei ragazzi, parla bene dei ragazzi, ma teme
promiscuità siccome i ragazzi sono innamorati di loro e non sono i loro mariti.
La risposta a questa preoccupazione viene da Filomena, la quale dice che ciò non importa,
essendo il piano onesto. \\
Pampinea approccia i ragazzi con un sorriso e cominciano ad organizzare il trasferimento [88].
Usciti dalla città di misero in via. Il luogo scelto è solo a due miglia di distanza dalla città [89].
Nei successivi paragrafi viene descritto il luogo: un palazzo con sale e camere affrescate, cortile, loggie,
giardini meravigliosi con pozzi d'acqua freschissimi e cantine di vini raffinati.
Le prime parole sono di Dioneo, il quale dice si aver lasciato i propri pensieri luttuosi e tristi nella città.
Il suo programma consiste in una dicotomia: o le donne cantano e si divertono con lui
(sempre nei limiti dell'onestà), o lui se ne torna in città.
Pampinea propone di avere un campo per tenere la comunità entro i propri limiti e in armonia,
che si preoccupi di far star bene gli altri 9.
Affinché tutti abbiano ambo i ruoli, Pampinea indica una carica a rotazione giornaliera,
dove il primo capo è scelto collettivamente e i successivi vengono scelti dai predecessori.

\pagebreak

\subsection{Andreuccio da Perugia}

\subsubsection{Analisi}

Il turno di parola viene dato immediatamente alla fine della novella precedente,
il pezzo di cornice fra le due novelle è quindi assente. 

\textbf{Rubrica:} Andreuccio da Perugia, venuto a Napoli a comperar cavalli, in una notte
da tre gravi accidenti soprapreso, da tutti scampato con un rubino si torna a
casa sua.

% mercato, 500 fiorini d'oro, li mostra
% giovane (prostituta) e vecchia siciliana | la vecchia riconosce adreuccio e spiega alla giovane ocme sono imparentati
% la giovane torna a casa e mette e fa occupare la vecchia, mentre trama la trappola ad andreuccio
% Ella manda qualcuno a chiamare andreuccio, dicendo che una donna gli vuole parlare ed è apparecchiato.
% Il vanitoso andreuccio crede che lei sia innamorata di lui, e si reca in dimora.
% Questa donna siciliana lo aspetta in cima alla scala per abbracciarlo calorosamente.
% Piangendo di gioia, gli da il benvenuto e lo porta nella sua camera, la quale è stata preparata per essere ricca piuttosto che di una prostituta.
% Gli comunica di essere sua sorella spiegando che il padre, Pietro, avrebbe vissuto da giovane a Palermo,
% e che anreuccio sarebbe nato a Perugia da un'altra donna (sono fratellastri), per poi essere abbandonati dal padre.
% - re carlo
% Ella gli fa domande estensive circa i suoi parenti, rafforzando  la credibilità di tutta la questione

% Lo invita a cena, e si offre di avvisare l'albero di Andreuccio di avvisare della sua assenza per restare da lei.
% Lei fa durare tanto la cena, in maniera tale da far rimanere andreuccio a dormire
% Da una camera ad Andreuccio con un fanticello a sua disposizione.
% Si spoglia del farsello (tipico abito maschile) e dei pantaloni, appoggiandoli affianco al letto.
% chiede al fanciullo dove fosse il bagno, e lui gli mostra l'uscio.
% L'asse sul quale appoggia il piede è tagliato e casca nella latrina.
% La donna si affretta a controllare se avesse lasciato i suoi averi in camera.
% ANdreuccio si ritrova bloccato in strada, in mutande e imbrattato di feci.
% pronta a rientrare bussando violentemente contro il portone, ma viene ignorato e scacciato dai vicini.
% Giunge il rozzo e assonnato padrone della prostituta, che scaccia Andreuccio minacciandolo prima che possa finire di spiegarsi.

% Andreuccio perde la speranza di recuperare i suoi fiorini e cerca di tornare al suo albergo
% ma nella strada svolta per _Runa Catalana_ per andare a lavarsi.
% Vede due signori con degli arnesi, ma non riesce a nascondersi per il proprio tanfo.
% Andreuccio gli racconta tutta la storia, e i due capiscono da che casa provenisse.
% Tuttavia, i due lo rassicurano perché se non fosse caduto sarebbe morto nel sonno.
% I due gli propongono di andare con loro e ottenere più di quanto noa bbia perso.
% Sicome disperato, Andreuccio accetta prima di sapere di cosa si tratti.

% Il lavoro è il saccheggiamento della tomba di un vescovo morto recentemente
% il quale possiede un preziosissimo rubino.
% Prime di andare, si decidono di lavare Andreuccio in un pozzo.
% Giunti al pozzo, il secchio manca e quindi legano andreuccio e lo calano nel pozzo.
% Una volta lavato, la fune si rompe.
% Arrivano delle due guardie nel pozzo per bere, le quali tirano sù andreuccio pensando fosse il secchio
% Appena notano le mani di Andreuccio si spaventano e lasciano cadere la fune.
% Andreuccio incontra nuovamente i due ladri per strada e si dirigono alla chiesa per saccheggiare la tomba.
% Il coperchio è compsoto da una grande lastra di marmo.
% Una volta sollevata la lastra, i due vogliono far entrare Andreuccio, il quale si rende conto dei loro intenti.
% Alchè, una volta entrato, Andreuccio prende immediatamente l'anello e se lo nasconde
% prima di passare ai ladri tutti gli altri beni sul vescovo.
% I due ladri tolgono il supporto, facendo rimanere Andreuccio bloccato nella tomba.

% dopo un po' di tempo giungono sul posto altri ladri, i quali aprono la lastra.
% Andreuccio prende per le gambe il prete che stava entrando. I ladri fuggono lascindo la lastra aperta.
% Andreuccio se ne va con il rubino molto prezioso.

La vicenda può essere grossolanamente separata in tre avventure:
quella con la donna siciliana nel chiassetto (§§1-55), quella del pozzo (§§56-70)
ed infine quella della chiesa (§§71-89).
Le tre disavventure sono ambientate in luoghi distinti e stretti,
ma in tutti i posti vi è l'azione di cadere/scendere (latrina, pozzo, arca).
Ognuna di queste caduta rappresenta una crescita personale per Andreuccio, infatti,
ogni caduta è progressivamente più difficile da superare. Questo percorso rappresenta il suo
sviluppo personale, dove impara ad essere più scaltro e astuto.
Il sistema dei personaggi è costruito con Andreuccio al centro e tutti gli altri come antagonisti.
Andreuccio è molto ingenuo e privo di esperienza (§§3, §§16) ma anche vanitoso (§§11).
La prima azione da lui fatta mediante un ragionamento critico è quando si intasca l'anello per fregare i due ladri (§§77),
questo suo genio è caratterizato dai verbi \quotes{pensò} e \quotes{s'avisò}.
Prima di allora, Andreuccio aveva sempre agito in maniera completamente passiva o solo per disperazione (§§64).
\\
A differenza di Andreuccio, la prostituta è molto intelligente e possiede la capacità di agire in maniera critica e astuta.
La prima dimostrazione di ciò è quando nota la borsa di Andreuccio senza essere vista (§§4),
ma la dimostrazione della sua scaltrezza è data dalla sua recita dove convince Andreuccio di essere la sua sorellastra.
\\
La novella di Andreuccio ricorda ua fiaba ed un romanzo di formazione
Questo romanzo di formazione porta Andreuccio a crescere, acquisendo una maggiore astuzia e anche più fortuna di
quanta non ne avesse all'inizio, per cui Andreuccio acquista il pensiero di un mercante.
A lui non interessa diventare più saggio o colto, diventa più furbo e usa questa qualità per ottenere un guadagno materiale
Sia la prostituta che i ladri e Andreuccio guadagno qualcosa di materiale.
Tutti questi personaggi si arricchiscono in maniera immorale, senza nessun giudizio negativo dalla parte di Boccaccio.

Gli spazi della novella sono posti reali, la storia del protagonista comincia a Perugia,
si svolge a Napoli per poi tornare nuovamente a Perugia.
I posti descritti sono fattuali, come l'albergo e la casa siciliana, le vie di napoli citate
e il mercato dei cavalli.
Oltre ai posti, anche i riferimenti storici sono veri: Napoli non era infatti una città sicura.
Il vescovo e la sua morte sono reali, anche se di un'altro tempo. I personaggi presenti nella Novella
sono anch'essi probabilmente esistiti.

\subsubsection{Inferno, Canto XXIV}

\href{https://en.wikipedia.org/wiki/Vanni_Fucci}{Vanni Fucci}
incontra Dante nel canto XXIV. Fucci ha commesso il crimine di
rubare da una chiesa, per questo soffro in eterno nell'ottavo cerchio dell'Inferno.
Allo stesso modo, Andreuccio ruba in chiesa e se ne torna a casa con più soldi.
La differenza è che a Boccaccio non interessano le implicazioni morali delle azioni
delle persone, bensì considera solo la loro vita terrena.
Infatti, i personaggi che commettono peccati non vengono necessariamente visti
in maniera negativa.
Contrariamente, Dante ritiene che la vita sia solo un piccolo segmento di tempo
per prepararsi all'oltretomba, e si concentra proprio questo aspetto.

\pagebreak

\subsection{Lo stalliere del re Agilulfo}

\textbf{Rubrica:} Un pallafreniere giace con la moglie d'Agilulf re, di che Agilulf tacitamente s'accorge; truovalo e tondalo; il tonduto tutti gli altri tonde, e così scampa della mala ventura.

\sdefinition{Industria}{
    Con \textit{industria} si intende la capacità di chi acquista o recupera una cosa desiderata.
    Molto spesso la cosa desiderata è di tipo erotico e sessuale.
}

Le rr. 4-8 della novella dichiarano la morale ideologica del quale la novella sarà un esempio.
Questo concetto è quello che non sempre è auspicabile pretendere di sapere tutto,
perché a volte sapere certe cose porta allo svergognarsi e si ritorcono contro di noi.
La novella è quindi una dimosrazione di questo fenomeno.

La regina è vedova e sfortunata in amore (rr. 11),
mentre il re è bravo, saggio e capace governare.
Lo stalliere appartiene al ceto più basso, ma è intelligente e fisicamente simile al re.
La dinamica dei personaggi crea un triangolo amoroso, ma nonostante
i vertici del triangolo siano molto distanti da un punto di vista sociale, possono comunque battersi ad armi pari.
Gli ambiti dove si possono battere ad armi pari sono l'ingegno e la prestazione amorosa (rr. 58-59).
La donna cantanta è sempre posta ad un livello più alto e l'amore da parte dello stalliere
è di tipo platonico (rr. X).
Questo amore rimane segreto e arde come il fuoco (rr. 19), si innamora in maniera nobile e
cortese. Nonostante esso sia nobile d'animo, vuole anche avere delle effusioni carnali con la regina.

Il testo può essere principalmente diviso in due parti: nella prima lo stalliere
ha uno scopo erotico, e vuole andare a letto con la regina. Nella seconda,
vuole eludere la vendetta del re non facendosi scoprire.
In ambo le parti l'obiettivo è raggiunto mediante la furbizia, con una doppia beffa.
Per due volte vi è uno scambio di persona. Inizialmente, si scambia l'identità con il re, mentre
nella seconda diluisce la sua identità fra quelle dei suoi compagni.
Possiamo notare come tutti guadagnino qualcosa: la regina guadagna una notte più appagante,
il re guadagna reputazione con la regina e lo stalliere ci va a letto.

Lo stalliere fa finta di essere arrabiato per non parlare con la ragione, e
possiede un senso del limite, siccome si ferma prima per non rischiare di essere scoperto (rr. 54) (e non sfiderà più la fortuna).
Possiede la capacità di autocontrollo poiché sta fermo quando il re gli posa la mano
sul petto. Inoltre, compie la seconda beffa tagliando i capelli a tutti gli altri (rr. 95-96).
Le sue azioni rimarranno per sempre un segreto.
Questi sono i punti dove esso dimostra una grande furbizia e scaltrezza.
Similarmente, il re presente anche una grande intelligenza
perché riesce a contenersi quando scope che sua moglie è stata a letto con un altro,
riesce ad autocontrollarsi, quando un altro invece avrebbe fatto una scenata (rr. X).
Il re presume che il colpevole sia qualcuno di vicino (rr. 75) e ascolta i battiti
cardiaci degli stallieri. Anche quando scopre il colpevole, riesce a contenersi e gli taglia
i capelli per riconoscerlo il giorno successivo.
Il re riconosce inoltre il merito dell'avversario, dopo aver combattuto sullo stesso
piano nonostante essendo di ceti sociali completamente opposti (rr. 102-103).

Ci sono quindi degli ambiti, in questo caso, prestazione erotica e ignegno,
dove vengono completamente rimosse le condizioni sociali.

\pagebreak

\subsection{Lisabetta da Messina}

\textbf{Rubrica:} I fratelli d'Elisabetta uccidon
l'amante di lei: egli l'apparisce
in sogno e mostrale dove sia sotterato;
ella occultamente disotterra la testa e mettela in un testo di bassilico,
e quivi sù piagnendo ogni dì per una grande ora, i fratelli gliele tolgono,
e ella se ne muore di dolor poco appresso.

\snote{}{
    Nonostante ci siano 3 uomini, chi narra si rivolge spesso alle donne.
    Ciò è dato dalla lode di Boccaccio verso le donne.
}

\begin{enumerate}
    \item \textbf{Premessa (§3) e antefatto (§§4-5)}
    \item \textbf{Svolgimento dell'azione (§§6-23)}
    \begin{enumerate}
        \item Protagonisti: fratelli (§§6-11)
        \begin{itemize}
            \item Scoperta della trasca (§§6-11)
            \item Omicidio (§§8-9)
            \item Domande di Lisabetta (§§10-11)
        \end{itemize}
        \item Protagonista: Lisabetta (§§11-18)
        \begin{itemize}
            \item Sogno (§§12-13)
            \item Scoperta del cadavere e recupero della testa (§§14-16)
            \item Culto per il vaso (§§17-18)
        \end{itemize}
        \item Protagonisti: fratelli (§§19-22): sottrazione del vaso
        \item Protagonista: Lisabetta (§23): morte
    \end{enumerate}
    \item \textbf{Conclusione che rivela l'origine della novella (§§23-24)}
\end{enumerate}

% Struttura
% Sistema dei personaggi
% Interpretazioni

I protagonisti, essendo alternati, mostrano di non comunicare fra di loro.
Non vi sono dialoghi fra i fratelli e le sorelle vi sono solo domande che non vengono nemmeno risposte.
Piuttosto che essere verbali, i fratelli compiono azioni concrete. Sono uomini pratici.
Al contrario, la sorella è più verbale, comunica con le sue domande (§10, §11, §13, §20) e il pianto (§11, §12, §14, §17, §18, §20, §23).
L'unico punto in cui Elisabbeta sospende il pianto è quando trova il corpo deceduto e ne taglia la testa.

I personaggi sono divisi in due, da un lato i 3 fratelli mentre dall'altra Lisabetta, dall'altro Lisabetta, Fante e Lorenzo.
Il rapporto non è equilibrato, i fratelli sono gerarchicamente superiori per ragioni sociali.
I loro valori sono completamente diversi, mettono il denaro in cima a tutto. Tuttavia, ciò non viene detto direttamente,
bensì viene implicato dal loro interesse per la reputazione della loro famiglia ed attività.
Se gli interessi della sorella si dovessero scoprire, la loro reputazione, assieme ai loro affari, crollerebbero (§7, §22).

Quando uno dei fratelli scopre la sorella a letto con il garzone, non agisce di impulso ma ne parla con gli altri fratelli.
La vicenda viene trattata come se fosse un affare di famiglia; ragionano in maniera fredda e come mercanti, bilanciando cosa fare in funzione della resa economica.
La novella è tragica, nessuno ritrae un valore dalla vicenda.

Boccaccio accusa la logica mercantile (non il mondo dei mercanti) con i suoi eccessi.
Questa novella è in contrapposizione a quella dello Stalliere, dove viene posto un limite, mentre i fratelli non si fermano, non riuscendo ad equilibrare furbizia e sentimenti.

% Da un punto di vista psicoanaliti, a differenza di quello sociale, i ruoli si ivnertono:
% i 3 fratelli non hanno nome, indistinguibili e privi di identità.
% Sono 3 attori, ma solo un attante
% Non sanno amare, non sanno comunicare
% Di lorenzo si dice che tutti lor fatti guidava, per cui i fratelli sono anche incapaci senza di lui.
% Contrariamente, Lisabetta sa amare, sa comunicare (con le domande), e sa agire (prende l'iniziativa con Lorenzo, nonostante il divieto)
% È come se i fratelli vedessero la loro inferiorià nella sessualità della propria sorella.
% I fratelli uccidono Lorenzo 3vs1, ingannandolo e colpendolo alle spalle come dei vigliacchi.

% Quando Lisabetta sogna Lorenzo, le sue parole non sono nè di amore nè di enfasi verso ciò che è successo.
% Quando Lisabetta gli stacca la testa, la mette in grembo alla fante come se fosse un figlio.
% Lisabetta trasforma quindi il trauma portando tutte le sue cure alla testa, come se fosse un bambino.
% la pianta del basicili, mitologicamente, è legata alla fertilità. Ciô è legato alla cura e la crescita del figlio.
% Il basilico è bellissimo e odorifero molto §19 che è sintatticamente la stessa descrizione di Lorenzo, §5

\pagebreak

\subsection{Nastagio degli Onesti}

\textbf{Rubrica:} Nastagio degli Onesti, amando una de'Traversari,
spende le sue ricchezze senza essere amato.
Vassene, pregato da'suoi, a Chiassi; quivi vede cacciare ad un
cavaliere una giovane e ucciderla e divorarla da due cani.
Invita i parenti suoi e quella donna amata da lui ad un desinare,
la quale vede questa medesima giovane sbranare; e temendo di simile avvenimento
prende per marito Nastagio.

% 1 riga di cornice

% Personaggi 
I personaggi vengono descritti poco. Lei è \underline{molto} nobile (§4), e siccome sa di essere tale e bella
è molto cruda nei confronti di Nastagio (sdegnosa). Nastagio è nobile, giovane, molto ricco (dalla morte del padre),
spende senza misura e si toglierebbe la vita per quanto la ami. Il suo vizio di spendere senza ottenre nulla lo sta rovinando.

% I due racconti
La storia di Nastagio potrebbe essere quella di Guido, ed è identica rispetto alla sua.
Infatti, i loro nomi di famiglia coincidono. Entrambi sono gentiluomini di Ravenna
che amavano non essendo essi amati.
Le due donne non hanno nome, e il loro comportamento è simile (§6, §21).
La storia di Guido è tuttavia andata più avanti dal momento che lui si è effettivamente ucciso (§21), 
mentre l'altro ne ha solo avuto la tentazione (§6).
Inoltre, Nastagio si sforza di odiarla (§7) e Guido la odia e la uccide (§26).
Guido è quindi l'evoluzione possiible della storia di Nastagio.
La donna di Nastagia non lo odia, bensì ne è solamente indifferente (§6), mentre l'altra
non solo è indifferente, ma gode anche della sua morte (§22).

Nastagio vede la proiezione della sua storia in quella di Guido, e cambia il corso degli eventi per far sì che
la sua prenda un cammino diverso.
Dopo la sua visione, sfrutta la situazione di caccia che gli si presenta (§32) mediantre la propria furbizia.

Rovesciamento di quello che la religione indica come un obbliga, che qui viene espresso come un vizio: \\
La novella di Nastagio è una parodia di exemplum (storia di modello morale. E.g. fiaba, con fine didascalico).
Agire contro le forze di natura è sempre sbagliato. In questo caso, l'attrazione, è giusto che trovi sfogo.
Questa è una nuova morale insegnata da Boccaccio.
Il cambiamento inverosimile è che tutte le donne di Ravenna si concedono agli uomini (§44)
e X in una sera tramuta l'odio in amore.
È Dio che vuole che la crudeltà sia da condannare e che la pietà sia da lodare.
Tuttavia, viene successivamente indicata la crudeltà come il fatto di resistere alle forze della natura, per cui
vi è un rovescimaneto di ciò che insegna la religione.
Al contrario di Dante, finisce all'inferno chi resiste all'amore.

% La parodia
% L'industria di Nestagio
% L'aldilà
% Le fonti della novella

\snote{Rovesciamento Dante \(\rightarrow\) Boccaccio}{
    Dante scrive 100 canti nell'oltretomba e ne ambienta 2 sulla terra, mentre Boccaccio scrive 100 novelle di cui solamente
    due sono ambientate nell'oltretomba (tra l'altro, Inferno sulla terra).
    Il centro della vita non è più la preparazione all'oltretomba, bensì la vita stessa.
}

Nell'(anti-)exemplum di Passavanti la morale viene esplicitata alla fine: resistere è un valore,
mentre cedere è un disvalore.

% FONTI. Chiedi a qualcuno perché stavo studiando bio

\pagebreak

\subsection{Frate Cipolla}

\snote{}{
    Le novelle sono solitamente corte siccome la morale è centrata attorno una singola battuta.
    La novella di Frate Cipolla ne fa eccezione, compensando la brevità delle prime nove.
}

\textbf{Rubrica:} Frate Cipolla promette a certi contadini di mostrar loro la penna dell'agnolo Gabriello, in lugoo della quale trovando carboni, quegli dice esser di quegli che arostiron san Lorenzo.

Frate Cipolla è un frate di santo Antonio conosciuto per essere un bravo e scaltro oratore.
A messa terminata parla di una reliquia che avrebbe presentato, una piuma che fu una delle penne dell'Arcangelo Gabriele.
Due suoi amici nella chiesa si decidono di fargli una beffa e di sottrargliela.
Per eludere il fante ingaggiano una donna molto formosa per sedurlo, il quale viene distratto ma fallisce nel sedurre lei.
I due rubano la penna e la rimpiazzano con del barcone.
Il giorno dopo, Frate Cipolla si trova molti fedeli lì per vedere la piuma, ma quando ne apre la scatola
ci trova il carbone. Allora comincia a parlare del viaggio che ha compiuto per trovarla,
alla fine se la cava dicendo di aver preso la scatoletta per sbaglio contenente il carbone di San Lorenzo,
e quindi era un miracolo che Dio gli avesse fatto prendere il carbone proprio due giorni prima del giorno santo.

% Personaggi
% Gli elementi comici
%  - Il discordo di Guccio Imbrato
%  - Il discorso di Frate Cipolla
% Il pubblico 
% La polemica religiosa

La novella presente delle descrizioni eccezionalmente lunghe, in particolare addirittura tre.
Frate cipolla è piccolo, lievo nel viso, con capelli rossi (tratto da imbroglione) ma
soprattutto è bravo a parlare nonostante non abbia studiato (§7).
Guccio Imbratta, il fante di Frate Cipolla, viene presentato da Frate Cipolla,
mentre Frate Cipolla viene presentato dal narratore Dioneo.
Guccio viene presentato (§17) con nove aggettivi in rime da tre:
sugliardo e bugiardo; negligente, disubidente e maldicente; trascutato, smemorato e scostumato.
Il terzo personaggio (§21) è la serva Nuta, piccolo, brutta, sudata, unta, con un paio di poppe che sembravano due
cestoni di letame.
Le descrizioni di personaggi fatte da altri personaggi danno informazioni circa i loro gusti.

Il discoso di Guccio Imbratta, per cercare di conquistare la Nuta, è più corto del discorso di Frate Cipolla.
Questo discorso è come un'anticipazione del discorso che farà il amestro Frate.
La strategia di Frate Cipolla è un fiume di parole che in realtà hanno l'obiettivo di confondere il suo interlocutore.
Anche Guccio è abbastanza scaltro, ma lui fallisce nel sedurre la serva.
Frate cipolla dice di aver (§37) compiuto un viaggio dove sorge il sole (verso Oriente)
che in realtà è una frase ambigua perché il sole sorge quasi ovunque sul pianeta terra.
Oltre alla grande tecnica di spacciare per fantastiche delle cose che in realtà sono normalissime,
si inventa anche delle parole.

Il luogo della novella è la città di Boccaccio, Certaldo, per cui è come se l'autore prendesse in giro i suoi compaesani.
La novella è narrata da Dioneo.
All'interno della novella, Frate Cipolla racconta agli abitanti di Certaldo e ai suoi due amici (pubblico nascosto).
Il fatto che gli abitanti siano di Certaldo è rivolto agli altri nove novellatori.

Anche in questa novella il giudizio morale verso il protagonista è assente.
Nonostante Frate Cipolla sia un ingannatore, ne esce positivamente come scaltro.

\snote{Dante, Paradiso, XXIV, 124-126 (Beatrice)}{
    L'ordine di sant'Antonio era particolarmente avido e scaltro con gli imbrogli,
    come per esempio le indulgenze.
    % TODO metti i versi qua
}

\pagebreak

\subsection{Gianni Lotteringhi}

\textbf{Rubrica:} Gianni Lotteringhi ode di notte toccar l'uscio suo; desta la moglie, ed ella gli fa accredere che egli è la fantasima; vanno ad incantare con una orazione, ed il picchiare si rimane.

% Gianni ha successo nel suo lavoro ma è scarso altrove
% I frati di un convento nominano Capo Gianni (per sfruttarlo un po' perché ricco)
% Lui fornisce tessuti per i loro abiti in cambio di insegnamenti (preghiere)

% personaggi
La novella è caratterizzata da un triangolo amoroso fra Tessa, Gianni e Federigo.
Gianni ha avuto molta fortuna con il suo lavoro, ma altrove è piuttosto scarso.
Esso è una persona molto ingenua, e per quanto riguarda la religione è molto
superstizioso e bigotto (§§4-5).
Federigo è bello, giovane e fresco (§6).
Rispetto a Gianni, Federgio è più intelligente e allegro.
I due sono infatti molto distanti.
% differenze
Il marito Gianni si mangia la carna salata, mentre l'amante si gode la cena completa (§§12-13).
La moglie con il marito dorme e basta, mentre con l'amante si sfoga in effusioni sessuali (§8 e §20).
Inoltre, un altro elemento di differenza è quello delle preghiere. Infatti, Gianni intende le preghiere in maniera letteraria,
mentre Federigo riesce ad interderne il significato (§8 e §20).
%
Tessa è bella, intelligente e saggia (§6). Rispetto è marito è molto più astuta, ma è innamorata di Federigo.
Conosce bene il marito e sa di poterlo inganare (§7).
Il marito viene ingannato dicendogli che la preghiera può ora essere detto dal momento che sono
entrambi presenti, facendolo sentire importante.


% non è importante che ci siano "2 versioni"
% Il consiglio di Emilia alle altre novellatrici è letterale; la beffa è utile per il tradimento

\pagebreak

\part{Petrarca (1304 - 1374)}

%\section{Biografia}

%TODO

\section{Rerum vulgarium fragmenta (Il Canzoniere)}

Il testo è composto da 366 poemi e parla del suo amore tormentato per una donna di nome Laura.
Tuttavia, il protagonista del libro non è Laura, bensì Petrarca che ne parla.

L'opera è separata in due parti: vi è un figlio bianco fra il Rvf 263 e 264,
ma Laura muore dal 267, lasciando un cuscinetto di 3 testi.

\subsection{Rvf 1: Voi ch'ascoltate in rime sparse il suono}

La prima poesia è un sonetto con schema delle rime ABBA, ABBA, CDE, CDE.

\begin{center}
    \textit{\textbf{Voi ch'ascoltate} in \textbf{rime sparse} il suono} \\
    \textit{di quei sospiri ond'io nudriva 'l core} \\
    \textit{in sul mio primo giovenile errore} \\
    \textit{quand'era in parte altr'uom da quel ch'i' sono,}
\end{center}
\begin{center}
    \textit{del vario stile in ch'io piango et ragiono} \\
    \textit{fra le vane speranze e 'l van dolore,} \\
    \textit{ove sia chi per prova intenda amore,} \\
    \textit{\textbf{spero trovar} pietà, nonché perdono.}
\end{center}
\begin{center}
    \textit{Ma ben veggio or sì come al popol tutto} \\
    \textit{favola fui gran tempo, onde sovente} \\
    \textit{di me medesmo meco mi vergogno;}
\end{center}
\begin{center}
    \textit{et del mio vaneggiar vergogna è 'l frutto,} \\
    \textit{e 'l pentersi, e 'l conoscer chiaramente} \\
    \textit{che quanto piace al mondo è breve sogno.}
\end{center}

Questo primo sonetto fa da prologo, ma possiede anche degli elementi di bilancio,
cioè degli elementi che leggono l'esperienza del libro retroaspettivamente,
e quindi fa anche da epilogo.
Inizialmente l'autore si rivolge al lettore che ascolta (v. 1), come spesso di tradizione (es. Il Paradiso II).
le prime due quartine presentano un errore di sintassi; nonostante si rivolga inizialmente all'autore,
il verbo principale dopo le varie subordinate è \quotes{spero} (v. 8), causando un incoerenza fra soggetto e verbo.
Questo è un elemento che fa da prologo, ossia la forma del suo libro è data da \quotes{rime sparse} (v. 1).
Nonostante ciò, vi è un significato che percorre le poesie nel loro ordine.
Dopo aver annunciato la forma del libro, viene anche annunciato il tema, ossia
un tema di una natura amorosa (v. 2).
Il (v. 3) fa invece da epilogo, perché l'amore, ormai passato, viene definito come errore.
L'\textit{errore} per questa donna viene giudicato tale perché esso risiede nell'atto stesso di amarla,
ossia il fatto che si tratti di un amore nei confronti di una donna terrena, mentre l'unico amore
vero può essere rivolto solo verso Dio. L'errore è appunto quello di essersi dedicati a qualcosa di fugace,
uno sviamento rispetto alla via verso Dio ed il suo amore eterno.
Tuttavia, questo errore non è ancora completamente superato poiché Petrarca
era \textit{in parte} un uomo diverso da ciò che è oggi (v. 4), e quindi il cambiamento non è ancora
completamente compiuto.
In fondo, possiamo notare un conflitto intrinseco fra l'amore sacro e l'amore terreno.
Questo sonetto è stato quindi scritto dopo tutto il resto, e potrebbe anche essere posizionato
logicamente al termine.
\\
Nella seconda quartina il poeta dichiarare di sperare di trovare pietà e perdono
presso un pubblico più ristretto rispetto a quello della prima quartina,
ossia presso coloro che nell'amore hanno fatto esperienza (v. 7).
Il \textit{vario stile} (v .5) si riferisce all'intercambiarsi continuo fra
razionalità e luciditià e pianto e dolore. Infatti, vi è chiasmo con le parole \textit{piango},
\textit{ragion}, \textit{speranze} e \textit{dolore}, dove all'interno vi è la razionalità e all'esterno
parole negative di dolore.
L'errore è quello di essersi attaccati ad un qualcosa di \textit{vano}, ma non inutile, bensì effimera, e quindi
indegna di un amore che dovrebbe essere dedicata solo alle cose Celesti.

TODO la seconda parte

\pagebreak

\subsection{Rvf 3: Movesi il vecchierel canuto et biancho}

La terza poesie è un sonetto con schema delle rime ABBA ABBA CDE CDE.

\begin{center}
    \textit{Era il giorno ch'al sol si scoloraro} \\
    \textit{per la pietà del suo factore i rai,} \\
    \textit{quando i' \textbf{fui preso}, \textbf{et non me ne guardai},} \\
    \textit{ché i be' vostr'occhi, donna, \textbf{mi legaro}.}
\end{center}
\begin{center}
    \textit{Tempo non mi parea da far riparo} \\
    \textit{contra colpi d'Amor: però m'andai} \\
    \textit{secur, senza sospetto; onde i miei guai} \\
    \textit{nel commune dolor s'incominciaro.}
\end{center}
\begin{center}
    \textit{Trovommi Amor del tutto \textbf{disarmato}} \\
    \textit{et aperta la via per gli occhi al core,} \\
    \textit{che di lagrime son fatti uscio et varco:}
\end{center}
\begin{center}
    \textit{però al mio parer non li fu honore} \\
    \textit{ferir me de \textbf{saetta} in quello stato,} \\
    \textit{a voi \textbf{armata} non mostrar pur l'\textbf{arco}.}
\end{center}

La prima quartina è bipartita siccome i primi due versi danno l'indicazione del tempo,
mentre gli altri due indicano cosa accade al poeta proprio in quel giorno.
Era il giorno a cui al sole, per pietà nei confronti di Chi l'ha creato, si scolorirono i raggi.
Questa è una parafrasi riguardante l'eclissi, e per il cui il giorno è il Venerdì santo, quando
Cristo morì prima di risuscitare (il giorno della Passione di Cristo).
Più nello specifico, la data di questo giorno di innamoramento è il 6 aprile 1327.
% data presa da una nota abituaria | nella chiesa di santa chiara la vide per la prima volta
Questa data è tuttavia forzata e non era un venerdì santo.
Una peculiarità di Petrarca è quella di far coincidere le date della propria biografia
con una storia superiore. Infatti, il 6 aprile 1348 è il giorno della morte di Laura.
Per Petrarca il numero 6 è un numero centrale.
L'innamoramento viene sempre attraverso lo sguardo, come il tipico innamoramento cortese,
un legame che imprigiona le due anime assieme.
Petrarca, con i secondi due versi, parla si come si sia innamorato durante il giorno della morte di Cristo.
Infatti, i due verbi usati, presero e legarono, sono gli stessi due verbi utilizzati nella descrizione
del Venerdì santo nel Vangelo per indicare l'arresto di Cristo.
Il suo innamoramento è quindi fortemente legato alla passione di Cristo.
I pensieri dell'autore erano orientati altrove, quando avvenne l'innamoramento, e non si aspettava
di innamorarsi, proprio quando i suoi pensieri erano orientati verso la morte di Cristo.
Questo concetto sarà ripreso varie volte.
Inoltre, la prima prima e l'ultima della quartina sono in contrasto, perché il verbo scolorare
indica una forza che si affievolisce, mentre legare indica una forza che aumenta.
In oltre parole, come il sole scompare, Laura appare. \\
% per cui i miei guai cominciavano nel comune dolore
% non pensavo che in quel momento dovessi ripararmi/proteggermi
Nella terza strofa l'autore si trova totalmente disarmato e trovò aperta la via
che dagli occhi scende al cuore, occhi che sono uscio e varco elle lacrime.
Il medesimo concetto di impreparatezza ritorna con il verbo disarmato, come d'altronde viene
nuovamente espresso dalla seconda strofa.
Nell'ultima terzina viene espresso come
l'innamoramento colpisce soltando lui piuttosto che anche Laura.
Non fu molto onorevole essere capito in un momento di preparazione, mentre
Laura non vide nemmeno l'arco.

Alcuni elementi della poesie cortese sono l'amore personificato come un guerriero
senza onore e il passaggio dagli occhi al cuore.
Vi sono anche degli elementi caratteristici della prospettiva religiosa e cristiana.
Questi sono i primi due versi che si riferiscono al Vangelo, il Tempo (del Venerdì santo),
e il comun dolore sempre riferitosi alla medesima vicenda.
Tutto il canzoniere è incentrato su questo oscillamento fra amore sacro e prospettiva religiosa.
Nel momento di massimo dolore per la cristianità non c'è una solo verso in cui Petrarca si accusi di
aver orientato i propri pensieri altrove.

\pagebreak

\subsection{Rvf 16: Movesi il vecchierel canuto et biancho}

La poesia è un sonetto con schema delle rime ABBA ABBA CDE CDE.

\begin{center}
    \textit{Movesi il vecchierel \textbf{canuto et biancho}} \\
    \textit{del dolce loco ov'à sua età \textbf{fornita}} \\
    \textit{et da la famigliuola sbigottita} \\
    \textit{che vede il caro padre \textbf{venir manco};}
\end{center}
\begin{center}
    \textit{indi trahendo poi l'antiquo fianco} \\
    \textit{per lvextreme giornate di sua vita,} \\
    \textit{quanto piú pò, col buon voler s'aita,} \\
    \textit{rotto dagli anni, et dal camino stanco;}
\end{center}
\begin{center}
    \textit{et viene a Roma, seguendo 'l desio,} \\
    \textit{per mirar la sembianza di colui} \\
    \textit{ch'ancor lassú nel ciel vedere spera:}
\end{center}
\begin{center}
    \textit{\textbf{così}, lasso, talor vo cerchand'io,} \\
    \textit{donna, quanto è possibile, in altrui} \\
    \textit{la disïata vostra forma vera.}
\end{center}

% caduta e biancho sono una ditologia sinonimica (sono due sinonimi)
Vi è una sproporzionata similitudine fra i primi undici versi e gli ultimi tre.
Questa similitudine delinea quindi una separazione del testo.

La prima parte può essere letta come una storia: un vecchietto si allontana
dalla sua città dove è cresciuto e dalla sua famiglia, la quale ne rimane stupida.
Il termine fornita può implicare che la sua vita sia ormai compiuta.
Infatti, la famiglia è sbigottita perché è preoccupata che non torni più (venir manco).
Questa partenza è connotata da un sentimento di affetto (vezzeggiativi vecchierel e famigliuola, dolce, caro).
Infatti, dolce e caro vengono valorizzati dall'inversione data dal fatto che siano prima del nome.

Da quel punto, il vecchio si trascina a fatica nelle sue ultime giornate,
spinto dalla sua volontà, sfaticato dall'età e dal cammino.
Il viaggio sta arrivando alla fine come la sua vita sta terminando.\\
Vi è un chiasmo con rotto: \textbf{rotto}, \textbf{anni}, \textbf{cammino}, \textbf{stanco},
dove gli aggettivi della fatica sono agli estremi.

Seguendo il desiderio, viene a Roma per ammirare la sembianza di Cristo (del Velo della Veronica),
che spera di poter vedere di nuovo in cielo, dopo la morte.

\sdefinition{Il Velo della Veronica}{
    Il \textit{Velo della Veronica} è un panno usato da Gesù per asciugarsi le lacrime e il sudore
    durante la Via Crucis.
    Si dice che il panno abbia impressa la faccia di Gesù. Il panno prende il nome della donna che lo aiutò ad alzarsi.
}

% lasso = ahimè
Allo stesso modo, io (scrittore) come il vecchietto, cerco nelle altre donne la vostra forma perfetta
(la forma di Laura).
Questa similitudine compara le sembianze di Laura e quelle di Cristo, e le donne con Santa Veronica.
Come il vecchierello è stanco, anche Petrarca è sfiatato (stanco - lasso).
Ciò che entrambi hanno in comune è l'idea di \textbf{ricerca} e di \textbf{lontananza}.

Gli elementi che hanno fatto discutere sono
\begin{itemize}
    \item la similitudine quasi blasfema, dove per rigor di logica Laura viene paragonata a Dio;
    \item la fortissima sproporzione del sonetto.
\end{itemize}

% Dante, Par. XXXI, vv. 103-111

\pagebreak

\subsection{Rvf 22: A qualunque animale alberga in terra}

\begin{center}
    \textit{A qualunque animale alberga in terra,} \\
    \textit{se non se alquanti ch'anno in odio il sole,} \\
    \textit{tempo da travagliare e quanto e 'l giorno;} \\
    \textit{\textbf{ma} poi che 'l ciel accende le sue stelle,} \\
    \textit{qual torna a casa et qual s'anida in selva} \\
    \textit{per aver posa almeno infin a l'alba.}
\end{center}
\begin{center}
    \textit{\textbf{Et} io, da che comincia la bella alba} \\
    \textit{a scuoter l'ombra intorno de la terra} \\
    \textit{svegliando gli animali in ogni selva,} \\
    \textit{non o mai triegua di sospir' col sole;} \\
    \textit{pur quand'io veggio fiammeggiar le stelle} \\
    \textit{vo lagrimando, et disiando il giorno.}
\end{center}
\begin{center}
    \textit{Quando la sera scaccia il chiaro giorno,} \\
    \textit{et le tenebre nostre altrui fanno alba,} \\
    \textit{miro pensoso le crudeli stelle,} \\
    \textit{che m'anno facto di sensibil terra;} \\
    \textit{et maledico il di ch'i' vidi 'l sole,} \\
    \textit{e che mi fa in vista un huom nudrito in selva.}
\end{center}
\begin{center}
    \textit{Non credo che pascesse mai per selva} \\
    \textit{si aspra fera, o di nocte o di giorno,} \\
    \textit{come costei ch'i 'piango a l'ombra e al sole;} \\
    \textit{et non mi stancha primo sonno od alba:} \\
    \textit{che, bench'i' sia mortal corpo di terra,} \\
    \textit{lo mi fermo \textbf{desir} vien da le stelle.}
\end{center}
\begin{center}
    \textit{Prima ch'i' torni a voi, lucenti stelle,} \\
    \textit{o torni giu ne l'\textbf{amorosa selva},} \\
    \textit{lassando il corpo che fia trita terra,} \\
    \textit{vedess'io in lei pieta, che 'n un sol giorno} \\
    \textit{puo ristorar molt'anni, e 'nanzi l'alba} \\
    \textit{puommi arichir dal tramontar del sole.}
\end{center}
\begin{center}
    \textit{Con lei foss'io da che si parte il sole,} \\
    \textit{et non ci vedess'altri che le stelle,} \\
    \textit{sol una nocte, et mai non fosse l'alba;} \\
    \textit{et non se transformasse in verde selva} \\
    \textit{per uscirmi di braccia, come il giorno} \\
    \textit{ch'Apollo la seguia qua giu per terra.}
\end{center}
\begin{center}
    \textit{Ma io saro sotterra in secca selva} \\
    \textit{e 'l giorno andra pien di minute stelle} \\
    \textit{prima ch'a si dolce alba arrivi il sole. }
\end{center}

\pagebreak
La metrica è caratterizzata da sestine con parole-rime: \\
\textbf{A:} \quotes{terra}, \textbf{B:} \quotes{sole},
\textbf{C:} \quotes{giorno}, \textbf{D:} \quotes{stelle},
\textbf{E:} \quotes{selva}, \textbf{F:} \quotes{alba}
in retrogradatio cruciata e congedo (A)E(C)D(F)B. \\
Fra queste parole vi è un rapporto di contenimento di sineddoche (alga-giorno, sole-stelle e selva-terra).

\sdefinition{Retrogradatio Cruciata}{
    Con \textit{retrogradatio cruciata} (retrogradazione incrociata)
    si indica un principio di rotazione per il quale le sei parole-rima
    della prima stanza ritornano sempre e obbligatoriamente nelle cinque stanze successive.
}

\hr

\circled{1} L'autore si rivolge inizialmente verso tutti gli animali della terra,
i quali, eccetto quelli notturni, faticano e soffrono durante il giorno.
Tuttavia, quando il cielo si oscura e le stelle appaiono,
gli animali tornano alle loro tane o ai loro rifugi per trovare
riposo almeno fino all'alba.
Questa descrizione riflette il ciclo naturale di attività e riposo,
e potrebbe essere interpretata anche come una metafora della condizione umana,
con le fasi di lavoro e di riposo nella vita di ognuno di noi. \\
Questa sestina può essere divisa da un \quotes{ma} avversativo che separa il giorno e la notte.
L'opposizione non è solamente luce contro buio, ma anche un'opposizione fra occupazione/sofferenza e riposo.

\circled{2} Questa introduzione è molto generale e l'autore non parla di sè stesso.
Nella seconda sestina, invece, Descrive i suoi sentimenti contrastanti nei confronti
del giorno e della notte.
Inizia dicendo che quando inizia l'alba e il sole inizia a dissipare
le ombre intorno alla terra, facendo svegliare gli animali nelle foreste,
lui non smette mai di sospirare finché c'è luce solare.
Quando vede brillare le stelle di notte, piange e desidera che sia giorno. \\
Questa sestina può essere divisa fra i primi quattro versi e gli ultimi due:
la prima rivolga al giorno e la seconda verso la notte.
A differenza della prima sestina, questa è interamente incentrata sull'io poetico.
La parola \quotes{Et} del primo verso lo distingue da tutto il resto del mondo.
Questa sua incessante sofferenza nel tempo, senza un momento di tregua,
si confonde fra il giorno e la notte.

\circled{3} Contrariamente a prima, la notte non scaccia più il sole, bensì
le stesse gli portano il sentimento del dolore poiché uomo (\quotes{sensibil terra},
come Dio creò l'uomo dal fango).
Inoltre, Petrarca maledice il giorno in cui è nato, il cui Sole lo rende come un uomo selvaggio cresciuto nei boschi.
\\
Questi versi riflettono un senso di inquietudine,
disgusto o disincanto nei confronti della vita umana e delle sue circostanze,
e trasmettono una sensazione di scontento e disillusione verso il proprio destino.
La maledizione del giorno della nascita è paragonata al giorno in cui Petrarca vide Laura (Rvf. 3).

\circled{4} Alla quarta strofa troviamo, per la prima volta, la causa delle sofferenze ininterrotta dell'autore,
ossia Laura. Ne esalta la bellezza e la purezza ed egli afferma che nemmeno una bestia selvaggia,
né di notte né di giorno, si sarebbe mai nutrita in una foresta così severa
e selvaggia come la donna di cui lui piange all'ombra e sotto il sole.
La sua amata è così straordinaria che il sonno o
l'alba non possono allontanare il suo desiderio per lei.
Anche se egli è un essere mortale fatto di materia terrena,
il suo desiderio per lei sembra provenire direttamente dalle stelle,
elevandolo al di là dell'umano e conferendo alla sua passione un carattere
quasi divino o trascendente. \\
Questi versi si possono direttamente collegare alla filosofia di Platone, il quale
diceva che tutte le idee erano nelle stesse, e le cose sulla terra ne erano le implementazioni
imperfette.

\circled{5} Questi versi esprimono il desiderio di Petrarca di poter vivere abbastanza
per tornare a contemplare le stelle prima di morire.
Egli spera di tornare a guardarle prima di essere ridotto in terra,
cioè prima della sua morte, chiedendo che la sua amata possa mostrargli pietà.
Petrarca crede che un solo giorno trascorso con lei possa rigenerare e rivitalizzare
molti anni di sofferenza, sottolineando il potere e l'importanza del suo amore per lui.
La speranza di arricchirsi dall'alba al tramonto del sole sottolinea quanto l'amore possa
influenzare e arricchire la vita di una persona anche nel breve spazio di un giorno. \\
Con \quotes{amorosa selva} non si intende l'inferno dantesco, bensì quello pagano in cui
si trova chi muore da innamorato.

\pagebreak

\circled{6} Nell'ultima strofa viene descritta in maniera più approfondita l'ipotetica notte con Laura.
Una notte da solo con lei, senza testimoni se non le stesse, senza l'alba che li sopraggiunga.
Viene fatto un riferimento ad un mito dove Apollo, il dio del Sole, segue Dafne.
Dafne, chiedendo aiuto agli Dei, viene trasformata in un'albero per sfuggire ad Apollo.
Viene quindi espressa la paura che Laura, come Dafne, diventi una pianta e gli sfuggisca dalle braccia.
La pianta di questo mito è l'alloro, il quale viene anche chiamato \textit{lauro}.

\circled{7} Tuttavia, questo desiderio è destinato a infrangersi perché Petrarca sarà già morto
e sepolto, e quel giorno sarà pieno di stelle, come per dire che non avverrà mai siccome
il giorno non può essere pieno di stelle (adynaton).

\sdefinition{Adynaton}{
    L'\textit{adynaton} è una figura retorica per la quale qualcosa succederà all'avvenire
    di qualcosa di sicuramente irrealizzabile, e di conseguenza non succederà mai.
}

La zona dell'inferno dedicata agli spiriti amanti possiede solo piante sempreverdi.
Tuttavia, la selva viene definita come \quotes{secca}.
Questo indica che Petrarca smetterà quindi di amarla prima di morire.

\pagebreak

\subsection{Rvf 90: Erano i capei d'oro a l'aura sparsi}

\begin{center}
    \textit{\textbf{Erano} i capei d'oro a l'aura sparsi} \\
    \textit{che 'n mille dolci nodi gli avolgea,} \\
    \textit{e 'l vago lume oltra misura ardea} \\
    \textit{di quei begli occhi, ch'or ne son sì scarsi;}
\end{center}
\begin{center}
    \textit{e 'l viso di pietosi color' farsi,} \\
    \textit{non so se vero o falso, mi parea:} \\
    \textit{i' che l'esca amorosa al petto avea,} \\
    \textit{qual meraviglia se di sùbito \textbf{arsi}?} 
\end{center}
\begin{center}
    \textit{\textbf{Non era} l'andar suo cosa mortale,} \\
    \textit{ma d'angelica forma; e le parole} \\
    \textit{sonavan altro, che pur voce humana.}
\end{center}
\begin{center}
    \textit{Uno spirto celeste, un vivo sole} \\
    \textit{fu quel ch'i' vidi: e se non fosse or tale,} \\
    \textit{piagha per allentar d'arco non sana.}
\end{center}

La metrica è un sonetto con schema di rima CDE DCE.

\sdefinition{Distico}{
    Il \textit{distico} è una strofa formata da una coppia di versi, solitamente legati da una rima.
}

\circled{1} Il primo distico è dedicato ai capelli di Laura, mentre il secondo è
dedicato ai suoi occhi. Il nome Laura non è scritto letteralmente,
ma \quotes{l'aura} è un allusione fonetica al suo nome (non vi erano ancora gli apostrofi,
e quindi viene scritto nell'originale come \quotes{laura}).
Questa strategia di illudere ad un nome in modo nascosto è un senhal.

\sdefinition{Senhal}{
    Il \textit{senhal} è una figura retorica e consiste in un
    appellativo riservato generalmente alla donna amata ma anche ad amici o
    altri personaggi.
}

I capelli biondi rappresentano la bellezza nella poesia antica perché:
\begin{enumerate}
    \item è il colore della divinità:
    \item è il colore degli Dei;
    \item gli angeli sono biondi;
    \item i capelli biondi sono più rari, e quindi più preziosi (nel Mediterrano).
\end{enumerate}
Nelle fiabe popolari i capelli neri rappresentano il cattivo,
mentre i capelli rossi rappresentano il diavolo.
Invece, nelle poesie nordiche i capelli neri sono belli.

La prima parola del primo verso, \quotes{Erano}, indica che queste sono caratteristiche
passate di Laura. Infatti, si sta riferendo al momento del suo innamoramento.
Nel secondo distico viene imposta una differenza fra la luminosità dei suoi occhi
fra il momento dell'innamoramento ed il momento della scrittura.
La grande novità di questo sonetto è che Laura è invecchiata, e ora i suoi occhi sono scarsi di luce.

% Si pensava che il mondo fosse costruire attorno al numero quattro. Quattro cardinali, quattro stagioni, quattro colori puri in natura
% Siccome la donna è quasi l'emblema della perfezione, essa deve portare i quattro colori.
% bianco (pelle) nero (capelli, ciglia, occhi) rosso (gote, labbra) giallo (deve per forza essere nei capelli)

\circled{2} Nella seconda strofa abbiamo la descrizione del viso.
Durante l'innamoramento gli sembravano che il suo viso si formasse dei colori della (pietà) compassione,
cioè che anche lei ricambiasse il sentimento, ma l'autore non sa se fosse vero o meno.
L'esca è un materiale infiammabile, e l'autore che la possedeva (ossia una predisposizione ad un potenziale innamoramento),
qual'è la meraviglia se lui subito si innamorò? Il suo sguardo fu sufficiente per innescarne la scintilla.

\pagebreak

\circled{3} Questi versi descrivono la straordinaria grazia e l'elevatezza dell'amata Laura.
Egli sottolinea come il suo modo di muoversi non sembrasse terreno o umano, ma piuttosto celestiale,
quasi angelico. Le parole che usava non sembravano essere solamente suoni umani,
ma avevano un tono o una qualità che andavano oltre la semplice voce umana, forse evocando
un senso di purezza o trascendenza.
Questa descrizione enfatizza la visione ideale e quasi divina che Petrarca aveva dell'amata Laura.

\circled{4} E se anche ora non fosse più così, la ferita non si rimargina nonostante l'arco sia allenato (cioè la frecca è già stata scoccata tempo fa).
Anche se Laura non è più splendente come in quel momento, e anche se l'arco non è più teso, la ferita dell'autore non si rimargina.
L'amore non è quindi finito, ed esso dura nel tempo. Questa è una dichiarazione di fedeltà dell'amore nonostante il tempo passi.

Il protagonista di questo sonetto è Petrarca che ricorda Laura.
Gli elementi tipicamente cortesi del testo sono:
\begin{enumerate}
    \item l'innamoramento con gli occhi;
    \item la freccia scagliata da Dio che crea un incendio;
    \item il potenziale di innamoramento dell'uomo che la donna accende;
    \item tratti tipici della donna: voce, occhi, viso, capelli.
    \item la visione della donna come angelica.
\end{enumerate}

I verbi al passato remoto \quotes{arsi}, \quotes{vidi} e \quotes{fu},
esprimono il fatto che il momento dell'innamoramento è irripetibile.
i verbi al presente sono \quotes{son}, \quotes{se fosse} (ora) e \quotes{sana},
con quella bellezza ormai sfiorita.
Tutti gli altri sono all'imperfetto, ossia che quelle azioni che hanno una durata fino al presente.

Questo sonetto potrebbe essere diviso a metà siccome i tratti di laura Laura
della sua bellezza terreno (viso, occhi etc.) sono descritti nella quartine,
mentre nelle terzine danno dei tratti più angelicati.
Infatti, a differenza delle quartine, le terzine descrivono alcuni tratti con
negazioni. Sopra, la bellezza terrena è esplicita, mentre quella angelica è indicibile,
e viene quindi descritta con cosa non è.

Il modello dietro a questo tipo di descrizione deriva da Enea nell'\textit{Eneide}:
prima i tratti terreni e poi quelli divini. Enea pensa che una donna sia una cacciatrice
prima di una dea.

% passato remoto = momento chiuso nel passato, non tanto temppo fa, ma chiuso nel passato

% tratti stillnovistici
% gli elementi cortesi possono essere dimostrati con i riferimenti:
% - a Guido guinizzelli "io voglio del ver la mia donna laudare",
% - Dante (vita nova, XXVI) "tanto gentile e tanto onesta pare
% più liminosa della stella diana (la srella più luminosa)

\pagebreak

\subsection{Rvf 267: Oimè il bel viso, oimè il soave sguardo,}

\begin{center}
    \textit{\textbf{Oimè} il bel viso, \textbf{oimè} il soave sguardo,} \\
    \textit{\textbf{oimè} il leggiadro portamento altero;} \\
    \textit{\textbf{oimè} il parlar ch'ogni aspro ingegno et fero} \\
    \textit{facevi humile, ed ogni huom vil gagliardo!}
\end{center}
\begin{center}
    \textit{et \textbf{oimè} il dolce riso, onde uscìo 'l dardo} \\
    \textit{di che morte, altro bene omai non spero:} \\
    \textit{alma real, dignissima d'impero,} \\
    \textit{se non fossi fra noi scesa sì tardo!}
\end{center}
\begin{center}
    \textit{Per voi conven ch'io arda, e 'n voi respire,} \\
    \textit{ch'i' pur \textbf{fui} vostro; et se di voi son privo,} \\
    \textit{via men d'ogni sventura altra mi dole.} % non vi può essere una sventura peggiore
\end{center}
\begin{center}
    \textit{Di speranza m'empieste et di desire,} \\
    \textit{quand'io partì' dal sommo piacer vivo;} \\
    \textit{ma 'l vento ne portava le parole.}
\end{center}

Questo è il primo sonetto in cui Petrarca parla (implicitamente) di Laura morta.
I primi elementi che fanno percepire il lutto sono i lamenti \quotes{oimè}.
Il \quotes{fui vostro} è al passato remoto e indica la fine di questa situazione,

L'autore lascia Laura con la sicurezza che l'avrebbe rivista, ma ciò non succede.

\pagebreak

\subsection{Rvf 267: XXX}

\pagebreak

\subsection{Rvf 310:  Zephiro torna, e 'l bel tempo rimena,}

\begin{center}
    \textit{Zephiro torna, e 'l bel tempo rimena,} \\
    \textit{e i fiori et l'erbe, sua dolce famiglia,} \\
    \textit{et garrir Progne et pianger Philomena,} \\
    \textit{et primavera candida et vermiglia.}
\end{center}
\begin{center}
    \textit{Ridono i prati, e 'l ciel si rasserena;} \\
    \textit{Giove s'allegra di mirar sua figlia;} \\
    \textit{l'aria et l'acqua et la terra è d'amor piena;} \\
    \textit{ogni animal d'amar si riconsiglia.}
\end{center}
\begin{center}
    \textit{Ma per me, lasso, tornano i piú gravi} \\
    \textit{sospiri, che del cor profondo tragge} \\
    \textit{quella ch'al ciel se ne portò le chiavi;}
\end{center}
\begin{center}
    \textit{et cantar augelletti, et fiorir piagge,} \\
    \textit{e 'n belle donne honeste atti soavi} \\
    \textit{sono un deserto, et fere aspre et selvagge.}
\end{center}

La metrica è un sonetto con schema di rima ABAB ABAB CDC DCD.

Zephiro è un vento caldo, tipico tiepito vento primaverile,
che torna riportando il bel tempo, fiori ed erbe che sono la sua dolce famiglia.
Inoltre, riporta anche il canto della rondine (Progne) e il pianto (canto malinconico)
dell'usignolo (Philomena).

Viene fatto un riferimento ad un mito dove Progne e Philomena erano sorelle.
Progne è sposara con Tereo. Tereo, violenta la cognata, sorella della moglie Philomena,
e le mozza la lingua per evitare che parli. Progne ne viene comunque a sapere,
e per punirlo uccide il figlio che ha avuto con lui, dandoglielo in pasto.
Le due donne chiedono agliuto agli Dei, ed essi per salvarle dalla violenza dell'uomo
le trasformano in rondine ed usignolo.

Anche la seconda strofa è descrittiva: i prati tornano ad essere rigogliosi, il cielo
si rasserena.
Nel medieovo si pensava che tutto il creato fosse composto da terra
acqua aria e fuoco in forma di amore.
Vi è una ciclicità in questa situazione.
TODO

Ma per l'autore, tornano invece i ricordi di Laura (morta).
Le chiavi del cuore che l'amata porta con sè, sono dall'altra parte, e di conseguenza
non l'amore è definitivamente chiuso e non può tornare.
Il canto degli ucelli e le pianure fiorite appaiono all'autore con un deserto.
Mentre vede la bellezza delle donne sono aspre e selvaggie.

\pagebreak

\subsection{Conclusione}

\begin{itemize}
    \item \textbf{Rvf 1} prologo o epilogo;
    \item \textbf{Rvf 3} innamoramento;
    \item \textbf{Rvf 16} tema dell'allontanaza e della ricerca;
    \item \textbf{Rvf 22} sofferenza nel tempo;
    \item \textbf{Rvf 90} descrizione di Laura;
    \item \textit{foglio vuoto}
    \item \textbf{Rvf 264} (canzone, non sonetto) fine amore;
    \item \textbf{Rvf 267} annuncio implicito della morte di Laura;
    \item \textbf{Rvf 268} annuncio esplicito della morte.
\end{itemize}

In \textbf{Rvf 1}, l'amore inizia il 6 aprile (Venerdì Santo, morte di Cristo),
mentre l'amore si chiude il 25 dicembre (Natale, nascita di Cristo).
Al tempo di Petrarca questo giorno era il primo dell'anno. 

La discrepanza fra la posizione del foglio bianco e la morte di Laura rappresenta il
fatto che il distacco dell'autore dalla prima parte
avviene autonomamente prima della morte di Laura.

\pagebreak

\part{Machiavelli}

\section{Introduzione}

\sdefinition{La trattatistica}{
    Un \textit{trattato} è un libro, solitamente scritto in latino,
    che descrive delle indicazioni comportamentali.
    Dietro ogni trattato vi è un'idea umanistica per perfezionare l'umano. \\
    Vige l'idea di demunicipalizzazione, ossia quella di distaccare
    il trattato dal municipio e corte locale, rendendoli più universali.
}

Un trattato può fissare uno dei seguenti modelli:
\begin{enumerate}
    \item modelli di comportamento;
    \begin{enumerate}
        \item Machiavelli, \textit{Il Principe} (il perfetto principe);
        \item Castiglione, \textit{Il Cortigiano} (il perfetto uomo e donna di corte);
        \item Giovanni della Casa, \textit{Il galateo} (il perfetto uomo civile.)
    \end{enumerate}
    \item modelli artistico-letterari;
    \begin{enumerate}
        \item Castelvetro;
    \end{enumerate}
    \item modelli linguistici;
    \begin{enumerate}
        \item Pietro Bembo, \textit{Prose della 'volgar liunga},
            dove vengono indicati Boccaccio e Petrarca come modelli per i rispettivi tipi di testo.
            Dante nel 1500 viene inece considerato come un rozzo a livello linguistico;
        \item Machiavelli: di scrivere con la lingua odierna;
        \item Castiglione: la lingua perfetta nascerebbe dal
            meglio delle parlate di tutte le corti d'Italia (secondo la bellezza fonetica delle parole).
    \end{enumerate}
\end{enumerate}

Con Machiavelli nasce l'idea odierna di politica.
I modelli scientifici sono trattati dai \textit{saggi} piuttosto che dai trattati.

\pagebreak

\section{Biografia}

Certamente! Niccolò Machiavelli è stato un influente pensatore, diplomatico e scrittore fiorentino nato nel 1469 e deceduto nel 1527. È noto soprattutto per il suo capolavoro "Il Principe", un trattato politico che esplora il potere, la politica e la leadership.

Nato a Firenze in una famiglia di modeste origini, Machiavelli ricevette un'educazione umanistica e divenne coinvolto nella politica fiorentina. La sua carriera diplomatica lo portò a interagire con figure di spicco del suo tempo, inclusi i Medici e alcuni degli esponenti più importanti della politica europea del Rinascimento.

Dopo la caduta dei Medici e l'instaurazione della Repubblica a Firenze, Machiavelli subì l'esilio e si ritirò a vita privata, dedicandosi alla scrittura. Ha continuato a scrivere e a studiare politica fino alla sua morte nel 1527.

La sua influenza si è estesa attraverso i secoli, e il suo nome è diventato sinonimo di astuzia politica e realismo nel campo della politica. La sua opera continua a essere studiata e discussa in ambito accademico e politico, lasciando un'impronta duratura nella storia del pensiero politico occidentale.

\section{Lettera a Francesco Vettori}

% Machiavelli segretario delfiglio di Papa Leone Xla repubblica, cade la repubblica e salgono i medici
% cacciano machiavelli e scrive da questo albergaccio dove si trova.
% Riesci a farsi reintegrare dai medici, e poi cambia di nuovo il potere e viene cacciato di nuovo

Niccolò Machiavelli scrisse diverse lettere a vari destinatari durante la sua vita. La lettera a Francesco Vettori, datata 10 dicembre 1513, è una delle sue più famose. In questa lettera, Machiavelli discute degli eventi politici dell'epoca, in particolare il suo recente licenziamento da parte dei Medici dopo la caduta della Repubblica di Firenze.
Nella lettera, Machiavelli esprime la sua delusione per essere stato escluso dalla vita politica e riflette sulle difficoltà e le instabilità della politica italiana. Egli condivide le sue preoccupazioni sulla situazione politica del tempo e sulla necessità di un principe forte e capace per riportare l'ordine e la stabilità nella regione.
Machiavelli discute anche della natura umana, sottolineando la necessità per un governante di adattarsi alle circostanze e di essere disposto a prendere decisioni impopolari per il bene comune. La lettera riflette il suo pensiero politico, caratterizzato dalla realpolitik e dalla convinzione che il fine giustifichi i mezzi.

L'esilio del 1512 denota uno spartiacque.
L'anno prima incontrava il re di Francia, mentre ora
le sue giornate sono monotone e non sa cosa fare.
Tuttavia, da un certo punto Machiavelli comincia a tenere il proprio cervello vivo
studiando (leggendo): questo è il senso della sua vita,
rialzando la sua scrittura.

\epigraph{``Mi pasco di quel cibo che solum è mio e ch'io nacqui per lui.''}{\textit{Niccolò Machiavelli}}

I luoghi della sua giornata sono essenzialmente tre:
\begin{itemize}
    \item \textbf{naturali} (luoghi aperti, bosco, etc.).
        Questi luoghi sono rappresentanti delle occupazioni pratiche (caccia di uccelli, commercio, taglio legna etc.);
    \item \textbf{la strada} (luogo di riflessione e di incontro con le persone dei paesi vicini);
    \item \textbf{osteria, scrittoio} (luoghi chiusi). Questi due posti sono quasi antitetici per la loro natura.
\end{itemize}

Un ragionamento analogo può esseere svolgo circa i tempi della sua giornata:
\begin{itemize}
    \item \textbf{mattini} (ocupazioni pratiche);
    \item \textbf{pomeriggio} (vita sociale);
    \item \textbf{sera e di notte} (tempo per sè, tempo dello studio).
\end{itemize}

La lettera è attraversata dal dialogo come tema dominante.
In primis, tutta la lettera è un dialogo con Vettori.
Nei contenuti, vi sono i dialoghi legati alle merci, legati al gioco/osteria,
quando chiede notizie e, metaforicamente, dialoga con gli scrittori che legge.



Attorno a questa lettera vi sono diversi problemi di inconsistenza.
Per cominciare, la frase
\begin{center}
    \textit{e} [ho] \textit{composto uno opuscolo De principatibus;}
\end{center}
sembra implicare che il trattato sia già finito alla scrittura della lettera.
Tuttavia, quando viene indicato di cosa parla, vengono indicati solamente
i principati come argomento, quando il trattato finale
tratta:
\begin{enumerate}
    \item Dedica;
    \item (Cap. 1-11) Principati;
    \item (Cap. 12-14) Armi;
    \item (Cap. 19-23) Il principe;
    \item (Cap. 24-26) \quotes{Italia contemporanea}.
\end{enumerate}
Il libro è quindi finito o no?

Nella lettera Machiavelli indica che il dedicatario sarà Giuliano de' Medici,
mentre dopo essere uscito la dedicata sarà a Lorenzo il giovane. % Non il magnifico (?)

Machiavelli prima descrive il libro come un ghiribizzo, quasi come un piccolo gioco,
ma successivamente dice che può essere usato da un principe nuovo.

\sdefinition{Stile dilemmatico}{
    Lo stile dilemmatico consiste nel creare un albero decisionale.
}
Il dubbio dilemmatico è quello di consegnare il libro.
E se darlo, darglielo personalmente o per altri mezzi?

La lettera è quindi piena di dissirio e molto problematica.

\section{Il Principe}

\subsection{La dedica}

Per entrare nelle grazie di un sovrano è necessario offrirgli qualcosa di prezioso o gradito.
Ciò che Machiavelli può offrire è
\begin{center}
    \textit{la cognizione delle azioni degli uomini grandi}
\end{center}
La parola grande non ha nessuna connotazione morale, bensì indica solamente
persone di potere.
Questa conoscenza è stata acquisita dalla sua lunga esperienza delle cose moderne
e dalla lettura dei testi antichi.
Il suo modo per offrire questa conoscenza è quello di includerle in un piccolo volume (\textit{Il Principe}).
Il vantaggio di un dono simile è che un libro è una sintesi di migliaia di pagine
lette e quindici anni di esperienza nella repubblica.

\snote{}{
    Fra i testi di Machiavelli vi è sempre una opposizione fra modestia e orgoglio.
    L'autore dona una accezione di poco conto alle sue opere, mentre
    offre anche una connotazione della loro grandezza e importanza (topos della modestia).
}

\begin{center}
    \textit{
        la quale opera io non ho ornata nè ripiena di clausule ampie, o di parole ampollose o magnifiche, o di qualunque altro lenocinio o ornamento estrinseco, con li quali molti sogliono le lor cose discrivere ed ornare; perchè io ho voluto o che veruna cosa la onori, o che solamente la verità della materia, e la gravità del soggetto la faccia grata.
    }
\end{center}

La forma del libro viene descritta come esteticamente poco attraente,
il motivo è quello che l'autore vuole che il suo testo venga onorato
solo per il suo contenuto effettivo, oppure per null'altro (piuttosto prefesce che non venga apprezza).
L'assenza di ornamenti è quindi quella di non far apprezzare il libro per qualcosa che non sia lo stretto contenuto.

\begin{center}
    \textit{
        Nè voglio sia riputata presunzione, se uno uomo di basso ed infimo stato ardisce discorrere e regolare i governi de' Principi; perchè così come coloro che disegnano i paesi, si pongono bassi nel piano a considerare la natura de' monti e de' luoghi alti, e per considerare quella de' bassi si pongono alti sopra i monti; similmente, a cognoscer bene la natura de' popoli bisogna esser Principe, ed a cognoscer bene quella de' Principi conviene essere popolare.
    }
\end{center}

Come i pittori/cartografi devono essere in basso per disegnare le altitudini, e viceversa, per parlare
del popolo bisogna essere principi, e per parlare di principi bisogna essere facenti parte del popolo.
Questa è la motivazione di Machiavelli per poter scrivere il principato.

Infinite l'autore si rivolge a Lorenzo de' Medici, augurandogli il meglio nel suo potere, e facendogli
notare la sua posizione sfortunata.

\subsection{Capitolo I}

\begin{center}
    \textit{
        Tutti gli Stati, tutti i dominii che hanno avuto, e hanno imperio sopra gli uomini, sono stati e sono o Repubbliche o Principati.
    }
\end{center}

Un Principato si differenzia dalla repubblica in quanto la repubblica è più democratica e partecipativa,
mentre il principato è incentrato su una singola persona o gruppo di perosone. \\
Come descritto dall'autore: I principali sono o nuovi (perché lo si conquista) o ereditari (lunga discendenza).
I principati nuovi, a loro volta, hanno anch'essi due diramazioni:
o sono completamente nuovi (\textit{come fu Milano a Francesco Sforza}),
o sono come pezzi aggiunti ad una conquista
(\textit{come è il Regno di Napoli al Re di Spagna}). \\
Per quest'ultima diramazione, a differenza degli altri, Machiavelli cita degli esempi storici.

\snote{}{
    La parola \textit{acquista} è un termine tecnico (termine generico, ma in questo caso indica conquistare).
}

O si conquista un popolo che è abituato ad essere libero, o che è abituato ad
essere sotto il dominio di un principe.
Inoltre, essi si conquistano o con le armi d'altri (mercenari, prestate o comprate),
oppure con il proprio esercito di milizia.
La terza ulteriore distinzione è quella più importante: si può conquistare per fortuna (tutto ciò che sfugge al calcolo umano),
oppure per virtù.
\\
Questo concetto di virtù è riferito alle capacità tecniche del principe (non la connotazione odierna morale).
Il principe virtuoso non è quindi necessariamente saggio, onesto etc., ma è colui che
possiede le capacità tecniche.

\pagebreak

\subsection{Capitolo VI}

\begin{center}
    \textit{
        De' Principati nuovi, che con le proprie armi e virtù si acquistano. 
    }
\end{center}

Il testo parte con il concetto di imitazione, per il quale uno scrittore
debba necessariamente ispirarsi a qualcuno.
In quanto, Machiavelli indica il suo volere di fornire dei modelli altissimi
ai quali i principi dovrebbero ispirarsi.
Questi modelli sono irraggiungibili ma, secondo Machiavelli, è necessario ispirarsi a loro.
Per motivare ciò viene data l'analogia dell'arciere che mira più in alto per raggiungere il suo bersagio
(mira più in alto così sei sicuo di raggiungere i tuoi obiettivi, che dovrebbero essere chiaramente
più modesti rispetto a quelli degli imperi antichi, che sono appunti, irraggiungibili, come l'Impero Romano).

\snote{}{
    Chiaramente, nell'ambito \underline{politico}, il fatto di avere avuto successo in passato
    non garantisce quello di avere successo nel presente.
    Questo motivo denota un inevitabile fallimento in questa ideologia.
}

I modelli da imitare indicati sono i seguenti:
\sperson{Moisè}{
    \textit{Mois} è un personaggio biblico, legislatore e fondatore del popolo ebraico di israele.
}
\sperson{Ciro}{
    \textit{Ciro II} è il fondatore della monarchia di Persia.
}
\sperson{Romolo}{
    \textit{Romolo} è un personaggio legato al mito della fondazione di Roma.
}
\sperson{Teseo}{
    \textit{Tese} è un personaggio mitologico, re di Atene.
}

Tutti e quattro sono dei fondatori di regno o di repubbliche.
Tutti e quattro sono inoltre riusciti nei loro intenti grazie
alle loro capacità, e non certo grazia alla fortuna.
In oltre, il loro tratto comune è quello di avere avuto un difficile riconoscimento iniziale,
e nessuna agevolazione.

Tuttavia, l'unico personaggio realmente esistito è Ciro.
Nonostante ciò, tutti sono scritti sulla stessa riga: all'autore non importa
questa distinzione, bensì solo l'importanza dell'esempio.
L'unica distinzione fatta è quella su Moisè, ossia l'ironia circa il fatto che lui
avesse il privilegio di parlare direttamente con Dio. 

Andare al potere con virtù significa fatica per raggiungere la carica,
ma successivamente mantenerla in maniera semplice.
Mentre raggiungere il potere con la fortuna, significa guadagnare la carica semplicemente,
ma fare fatica a mantenerla.

Un elemento intermedio fra la virtù e la fortuna è l'occasione.
Se il principe non ha virtù, l'occasione non è nè riconosciuta nè sfruttata.
Un minimo di fortuna è necessaria, ma l'assenza di fortuna viene compensata dalla grande virtù.
Infatti, questi elementi di fortuna sono descritti per ognuno dei quattro modelli:
\begin{itemize}
    \item per Moisè, trovare il popolo di Israele in Egitto schiavo e oppressi;
    \item per Ciro, trovare i persi malcontenti;
    \item per Romolo, essere allontano dal suo paese natale;
    \item per Teseo, trovare una popolazione dispersa per poi riunirla.
\end{itemize}

Quando un principe domina un nuovo popolo, spodestando il vecchio regime,
il gruppo che sostiene il vecchio principe è più agguerrito, perché lotta per delle \textbf{certezze},
mentre quello che sostiene quello nuovo è più tiepido, siccome lotta per un'\textbf{idea}.
Gli uomini lottano debolmente per un ideale se non hanno certezza di un risultato.

Un altro fattore chiave è quello dell'autosufficienza:
nel caso in cui il principe debba pregare per il sostegno da altri, cade inevitabilmente,
ma quando il principe è autonomo, è raro che incorra in pericoli.
Da questo deriva il fatto che solo chi è armato può vincere, mentre chi
non possiede forza è destinato ad andare in rovina.
Infatti, Per quanto i quattro sopracitati fossero virtuosi, ad un certo punto hanno avuto la necessità
di usare le armi.

Inoltre, le convinzioni del popolo sono importanti: è facile convincere il popolo di qualcosa,
ma è difficile fargli cambiare idea e fermarli.
Nel caso in cui la popolazione sia convinta fermamente di qualcosa, è necessario
fargli cambiare idea con le armi.

% ovvero = oppure

Viene infine dato un altro esempio (minore), ossia quello di Ierone (Gerone) Siracusano.
Questo esempio riassume tutto il capitolo: diventò principe di Siracusa
senza nessuna esperienza di potere e avendo dalla fortuna \textit{solo} l'occasione
(essendo I siracusani oppressi, lo elessero come principe).
Egli eliminò il vecchio esercito per averne uno proprio e fedele,
cambia completamente le alleanze, e su tale edificio (fondamenta faticate) potè
governare facilmente.

\pagebreak

\subsection{Capitolo VII}

Questo capitolo tratta l'opposto di quello precedente, ossia quando si giunge al potere
con la fortuna.
È possibile mantenere il potere una volta conquistato senza sforzo, si può
anche avere successo a condizione che la virtù la si dimostri dopo.

\sperson{Cesare Borgia}{
    \textit{Cesare Borgia} è stato un generale, cardinale, nobile e politico italiano.
}

L'esempio di Cesare Borgia è uno di quelli di chi ha fallito.
Tuttavia, Machiavelli lo propone comunque come modello poiché
il motivo per il quale Cesare Borgia ha fallito è dato da una sfortuna immensa,
una sfortuna quasi impossiible da riavere.

\snote{}{ % grazie al papa si ritrova la romagna
    Guidò l'esercito francese alla conquista del Ducato di Milano e,
    con l'appoggio del papa, incominciò la riconquista dei territori della Romagna,
    battendo i vari signorotti locali, fra cui la celebre Caterina Sforza,
    ricevendo in seguito dal padre il titolo di Duca di Romagna.
    Successivamente invase il Regno di Napoli guidando le truppe francesi.
    Nel 1502, raggiunto rapidamente un grande potere politico,
    riuscì a difendersi dalla congiura della Magione,
    traendo in inganno i traditori e facendoli strangolare a Senigallia.
    Questa vendetta colpì molto l'opinione pubblica.
}

Machiavelli si prepara alla morte futura del padre, per cui a quando
non avrà più il sostegno del padre e del papa.
La sfortuna risiede nel fatto che in concomitanza con la morte del padre,
Cesare Borgia si ammala e muore per la debolezza.

\pagebreak

\subsection{Capitolo XV}

Il capitolo XV risponde a come un principe si deve comportare
con i sudditi e con gli amici (alleati).
L'autore è consapevole di non essere il primo a parlare
delle caratteristiche del principe (trattatistica politica).
Infatti, in molti ne hanno già scritto e Machiavelli si
aggiunge alla tradizione.
Qui l'autore dichiara una novità: nonostante si inserisca in una tradizione,
porta qualcosa di diverso.
Il motivo di questa novità è che Machiavelli si distacca
dagli orfini (categorie mentali, principi) altrui.
Il suo scopo è quello di essere utile, ossia
affinché i suoi printicipi abbiano una utilità pratica e applicata
nella realtà.
Per proporre il ritratto del principe perfetto, viene utilizzata
la verà effettiva piuttosto che l'immaginazione.
Secondo Machiavelli gli altri si sono sempre basati sull'immaginazione,
ossia una idealizzazione del principe.
Questi principi utopici erano sempre quelli che incorporavano tutte le virtù.
Una sorta di catalogo astratto di virtù.
Vi è talmente tanta differenza fra come si vive realmente (effettiva realtà)
e come sarebbe bello vivere (immaginazione utopica) che
individuo (in particolare, un partice), se si comporta in modo moralmente giusto
perde il potere. La tesi è quindi che chi si comporta bene
viene stroncato dagli altri, perché gli altri non si comportano bene.

Uno è tenuto liberlae (generoso), l'altro misero (avaro),
l'uno effemminato e pusillanime (debole) e l'altro
feroce ed animoso. L'uno umano l'altro superbo, l'uno lascivo (che si lascia andare alle voglie) e l'altro casto.
L'uno intero (onesto), l'altro furbo (astuto).
\\
Tutte queste copie sono composte da parole antitetiche
secondo un criterio morale. Di conseguenza abbiamo delle virtù associate a dei vizi.
Secondo Machiavelli, sarebbe bellissimo se un principo avesse tutte queste virtù,
ma dal momento che non si possono rispettare sempre, date le condizioni umane,
va bene ignorarle se necessario.
Inoltre, che il principe non si preoccupi di ricorrere a quei vizi,
se sono necessari per salvare lo stato.

% Dover giustificare presuppone che i mezzi siano sbagliati,
ma nella prospettiva dell'autore, se le azioni sono necessarie
sono corrette e non c'è nulla da giustificare.
Questa prospettiva non è più morale ma sulla base di un criterio funzionale
e politico.
La morale ragione su dei principi assoluti, ma Machiavelli porta i valori a
dei principi relativi.
A questo punto lealtà e svelatà sono dei termini neutri:
possono essere positivi o negativi a seconda delle circostanze.

\pagebreak

\subsection{Capitolo XVIII}

% verità effettuale
Il tema di questo capitolo è la lealtà, slealtà e mantenere
la propria parola.
Viene continuata la trama principale:
nonostante un principe con tutte le virtù positive
sarebbe utopico, nella realtà \textit{effettuale},
i grandi principi sono stati quelli che hanno saputo essere sleali.

\snote{}{
    ciononostante \(\iff\) nondimanco
}

Un principe può combattere o con le leggi (morali)
o con la forza. Quel primo, è proprio dell'uomo, quel secondo
è delle bestie.
Pertanto ad un Principe è necessario saper ben usare la bestia e l'uomo.
Questo principio era già stato insegnato implicitamente dagli iscrittori antichi:
viene citato come esempio l'eroe greco Achille, che fu allevato
da un centauro (metà uomo, metà bestia).
Chiaramente, questa argomentazione è piuttosto debole,
soprattutto perché si tratta di un mito.

La forza bestiale che un principe deve possedere è, a sua volta,
suddivisa in astuzia (della volpe) e forza (del leone).
Un principe non deve mantenere la parola data, se ciò
gli si ritorce contro o se non valgono più i motivi per cui ha promesso.
Se tutti gli uomini fossero buoni, idealmente, questo non varrebbe,
ma dato che sono tristi (malvagi), e loro non manterebbero
a loro volta le loro promesse, allora è necessario.

È importante notare che il principe non deve farsi notare
quando inganna. Gli uomini sono semplici ed ingenui, e quindi
chi inganna troverà sempre chi può essere manipolato.
L'immagine che il principe deve dare
è quella di essere completamente sulla parte virtuosa.
L'uomo tende a giudicare sulla base delle apparenze.

Il principe è giudicato sulla base dei fini (dei risultati).
Tutto il mondo è popolo (e quindi manipolabile).

\subsection{Capitolo XXV}

% prudenza è virtù
Il capitolo 25 del Principe di Machiavelli affronta il tema della fortuna e del ruolo che essa gioca nella vita dei principi. Machiavelli discute come molte persone credano che il mondo sia governato dalla fortuna e da Dio, e che gli uomini non possano fare nulla per cambiarlo. Tuttavia, egli sostiene che mentre la fortuna può influenzare circa la metà delle azioni umane, l'altra metà è ancora soggetta al controllo umano.

Machivalli usa l'immagine di un fiume in piena per descrivere la fortuna: quando è arrabbiata, distrugge tutto ciò che trova sul suo cammino, ma durante i periodi di calma, gli uomini possono prendere precauzioni e proteggere se stessi con argini e ripari. Tuttavia, egli nota che l'Italia, che ha vissuto molte vicissitudini, manca di queste difese.
Il principe che ha successo è colui che \textit{adatta} il proprio comportamento agli eventi della fortuna.

Secondo Machiavelli, la fortuna dimostra la sua potenza dove non c'è virtù per resistere ad essa. Egli osserva che i principi possono essere felici o infelici a seconda di come il loro modo di agire si adatta ai tempi in cui vivono. Ad esempio, un principe che agisce con ferocia e impeto può avere successo se il momento è propizio, mentre un altro principe che agisce con prudenza e rispetto può fallire se non si adatta ai cambiamenti delle circostanze.

Machavelli conclude che è meglio essere impetuosi piuttosto che rispettivi, poiché la fortuna è simile a una donna e si lascia più facilmente vincere da coloro che la affrontano con audacia e ferocia. Infine, nota che la fortuna è amica dei giovani perché sono meno rispettosi e più audaci nel comandarla.

\pagebreak

\part{Ariosto}

\section{Biografia}

Ludovico Ariosto è stato un celebre poeta e drammaturgo italiano del Rinascimento.
Trascorse gran parte della sua vita a Ferrara, dove servì sotto il patronato
dei duchi d'Este. Inizialmente intraprese la carriera giuridica,
ma il suo vero amore era la poesia.
Fu attivo anche come diplomatico e funzionario di corte,
occupando diverse posizioni amministrative.

% Battaglia di Roncisvalle
% musulmani (oriente, africa e spagna (per gli influssi))
% contro l'europa

\section{Orlando furioso}

\subsection{Introduzione}

L'\textit{Orlando furioso} succede un altro libro,
\textit{Orlando Innamorato}.

\textbf{Testimonianza I:} Isabelle d'Este scrive una lettera a Ippolito d'Este,
ringraziandolo per avendole inviato l'Ariosto, il quale l'ha fatta
divertire per due giorni.

\textbf{Testimonianza II (1517):}  Machiavelli scrive una lettera all'amico
Lodovico Alamanni, dove dice di aver letto il libro e
che si tratta una bella opera, degna di ammirazione.
L'unica cosa di cui si lamenta Machiavelli è di non essere presente nella lista di
poeti apprezzati di Ariosto presente nel libro.

\epigraph{\quotes{Io ho letto a questi dì Orlando Furioso dello Ariosto, e veramente il poema è bello tutto, e in molti luoghi è mirabile. Se lo incontrate raccomandatemi a lui, e ditegli che io mi dolgo solo che, avendo egli ricordato tanti poeti, m'abbia lasciato indietro come un cazzo.}}
{\textit{Niccolò Machiavelli}}

% 1516      40 canti
% 1521      40 canti
% 1525      prose della volgar lingua
% 1532      46 canti

La metrica è sempre ABABABCC.
I canti 1-4 compongono il Proemio, quelli dal 5-9 compongono la \quotes{gionta}.

\pagebreak

\subsection{Proemio}

\subsubsection{Testo}

% https://letteritaliana.weebly.com/la-fuga-di-angelica1.html

\begin{center}
\begin{minipage}{0.5\textwidth}
\centering
\makecell[l]{
    \textit{Le donne, i cavallier, l'arme, gli amori,} \\
    \textit{le cortesie, l'audaci imprese io canto,} \\
    \textit{che furo al tempo che passaro i Mori} \\
    \textit{d'Africa il mare, e in Francia nocquer tanto,} \\
    \textit{seguendo l'ire e i giovenil furori} \\
    \textit{d'Agramante lor re, che si diè vanto} \\
    \textit{di vendicar la morte di Troiano} \\
    \textit{sopra re Carlo imperator romano.} \\
}
\end{minipage}
\end{center}

I due grandi temi del libro sono l'amore e la guerra.
I modelli letterali al quale si ispira sono al letteratura bretone (per l'amore),
mentre letteratura Carolingia (per l'amore).
Nella poesia è presente un doppio chiasmo, indicante.
Tradizionalmente il proemio è suddiviso in tre parti: protasi, invocazione e dedica.
La prima parte è la protasi cioè la dichiarazione della materia di
cui parlerà il libro, in questo caso va dall'ottava 1 all'ottava 2 verso 4.
Poi c'è l'invocazione che occupa soltanto la seconda parte della seconda ottava
(2.4 - 2.8). Infine, c'è la dedica che va dall'ottava 3 alla 4.
All'inizio c'è la spiegazione del tema, cioè amore e guerra.
Ariosto si basa su due cicli, quello carolingio e quello bretone.
Il ciclo carolingio, insieme di racconti francesi e ispirati a Carlo Magno,
che si impronta sulla guerra e la virtù religiosa.
Il ciclo bretone o arturiana si basa sulla corte di re Artù,
è improntato sulla fantasia e l'avventura. Le prime parole sono legate
tra loro tra un chiasmo di “donne, amori, cortesie” e “cavallier, arme,
audaci imprese”. Il chiasmo è presente per presagire la struttura del libro,
i due cicli saranno uniti lungo il libro, intrecciandosi come il chiasmo. 
La struttura dei primi due versi segue un ordine non “corretto”,
il soggetto che parla è all'ultimo.
Dunque, apparentemente l'Orlando viene scritto in maniera oggettiva,
il poeta si mette in secondo piano.
L'oggettività sta nel mettere la materia prima del narratore.

\begin{center}
\begin{minipage}{0.5\textwidth}
\centering
\makecell[l]{
    \textit{Dirò d'Orlando in un medesmo tratto} \\
    \textit{cosa non detta in prosa mai, né in rima:} \\
    \textit{che per amor venne in furore e matto,} \\
    \textit{d'uom che sì saggio era stimato prima;} \\
    \textit{se da colei che tal quasi m'ha fatto,} \\
    \textit{che 'l poco ingegno ad or ad or mi lima,} \\
    \textit{me ne sarà però tanto concesso,} \\
    \textit{che mi basti a finir quanto ho promesso.}
}
\end{minipage}
\end{center}

Vi dirò meglio una cosa che non ha mai scritto nessuno,
ossia l'impazzimento di Orlando a causa dell'amore,
lui che era così saggio.
Orlando è un personaggio storico, paladino di un nobile, sotto Carlo Magno,
ma le storie di pazzia dell'innamoramento sono inventate.
Questo innamoramento è straordinario perché Orlando era particolarmente saggio.
Anche se una persona così saggia come lui può impazzire per amore come lui,
da questa sorte non è al riparo nessuno.
%Vi è una autoironia dicendo: io l'ingegno ce l'ho già, speriamo
%non mi venga tolto.
Una lima, col suo agire costante, erode nel tempo.

\begin{center}
\begin{minipage}{0.5\textwidth}
\centering
\makecell[l]{
    \textit{Piacciavi, generosa Erculea prole,} \\
    \textit{ornamento e splendor del secol nostro,} \\
    \textit{Ippolito, aggradir questo che vuole} \\
    \textit{e darvi sol può l'umil servo vostro.} \\
    \textit{Quel ch'io vi debbo, posso di parole} \\
    \textit{pagare in parte e d'opera d'inchiostro;} \\
    \textit{pagare in parte e d'opera d'inchiostro;} \\
    \textit{che quanto io posso dar, tutto vi dono.}
}
\end{minipage}
\end{center}

Questi versi rappresentano un esempio di cortesia e
ammirazione nei confronti del destinatario del canto,
Ippolito d'Este, figlio di Ercole I d'Este.
Il poeta esprime la sua gratitudine e il suo rispetto
per Ippolito, riconoscendolo come un ornament
e uno splendore del suo tempo.

Il tono del canto è deferente e rispettoso,
e Ariosto si colloca in una posizione di umiltà di
fronte al destinatario.
Utilizza l'immagine di sé stesso come "umile servo" di Ippolito,
sottolineando il suo desiderio di compiacerlo con
il suo lavoro poetico.

Il verso finale, "che quanto io posso dar, tutto vi dono",
sottolinea l'impegno totale del poeta nel rendere omaggio
e onore a Ippolito, promettendo di dedicargli tutto
ciò che può offrire, sia in parole che in azione.

\begin{center}
\begin{minipage}{0.5\textwidth}
\centering
\makecell[l]{
    \textit{Voi sentirete fra i più degni eroi,} \\
    \textit{che nominar con laude m'apparecchio,} \\
    \textit{ricordar quel Ruggier, che fu di voi} \\
    \textit{e de' vostri avi illustri il ceppo vecchio.} \\
    \textit{L'alto valore e' chiari gesti suoi} \\
    \textit{vi farò udir, se voi mi date orecchio,} \\
    \textit{e vostri alti pensieri cedino un poco,} \\
    \textit{sì che tra lor miei versi abbiano loco.}
}
\end{minipage}
\end{center}

TODO

\begin{center} % 5
\begin{minipage}{0.5\textwidth}
\centering
\makecell[l]{
    \textit{Orlando, che gran tempo innamorato} \\
    \textit{fu de la bella Angelica, e per lei} \\
    \textit{in India, in Media, in Tartaria lasciato} \\
    \textit{avea infiniti ed immortal trofei,} \\
    \textit{in Ponente con essa era tornato,} \\
    \textit{dove sotto i gran monti Pirenei} \\
    \textit{con la gente di Francia e de Lamagna} \\
    \textit{re Carlo era attendato alla campagna,}
}
\end{minipage}
\end{center}

Orlando, che per molto tempo è stato innamorato de la bella Angelica (principessa musulmana del Catai, oggi Cina).
Riesce a portarla a casa sua in Ponente, dove trova una situazione di guerra,
con re Carlo si era accampato preparandosi per la guerra.

\begin{center} % 6
\begin{minipage}{0.5\textwidth}
\centering
\makecell[l]{
    \textit{per far al re Marsilio e al re Agramante} \\
    \textit{battersi ancor del folle ardir la guancia,}\\
    \textit{d'aver condotto, l'un, d'Africa quante}\\
    \textit{genti erano atte a portar spada e lancia;}\\
    \textit{l'altro, d'aver spinta la Spagna inante}\\
    \textit{a destruzion del bel regno di Francia.}\\
    \textit{E così Orlando arrivò quivi a punto:}\\
    \textit{ma tosto si pentì d'esservi giunto:}
}
\end{minipage}
\end{center}

Re Carlo si accampa per fare pentire al re Marsilio (re della spagna musulmana)
di aver condotto il proprio esercito contro la Francia e
al re Agramante (re dell'Africa, musulmano)
di aver portato le sue truppe in Europa.
Quando Orlando torna con la donna amata, si pente di essere tornato proprio in quel momento.

\begin{center} % 7
\begin{minipage}{0.5\textwidth}
\centering
\makecell[l]{
    \textit{Che vi fu tolta la sua donna poi:} \\
    \textit{ecco il giudicio uman come spesso erra!} \\
    \textit{Quella che dagli esperi ai liti eoi}\\
    \textit{avea difesa con sì lunga guerra,}\\
    \textit{or tolta gli è fra tanti amici suoi,}\\
    \textit{senza spada adoprar, ne la sua terra.}\\
    \textit{Il savio imperator, ch'estinguer volse}\\
    \textit{un grave incendio, fu che gli la tolse.}
}
\end{minipage}
\end{center}

Carlo Magno stesso, sottrae Angelica da Orlando.
Quella che aveva difeso gli viene portata via così dagli amici,
viene detto che Carlo Magno volesse \quotes{estinguere un incendio} (metaforicamente).

\begin{center} % 8
\begin{minipage}{0.5\textwidth}
\centering
\makecell[l]{
    \textit{Nata pochi dì inanzi era una gara} \\
    \textit{tra il conte Orlando e il suo cugin Rinaldo,} \\
    \textit{che entrambi avean per la bellezza rara}\\
    \textit{d'amoroso disio l'animo caldo.}\\
    \textit{Carlo, che non avea tal lite cara,}\\
    \textit{che gli rendea l'aiuto lor men saldo,}\\
    \textit{questa donzella, che la causa n'era,}\\
    \textit{tolse, e diè in mano al duca di Bavera;}
}
\end{minipage}
\end{center}

In questa ottava viene spiegato la metafora dell'incendio.
Il motivo è che il conte Orlando e suo cugino Rinaldo sono erano
innamorati di Angelica.
Allora, re Carlo, temendo una riduzione dell'efficienza dei due guerrieri più bravi,
sottrae la donna per non farli distrarre.
La donna viene data a Namo, il duca di Bavera, per custodirla.

\begin{center} % 9
\begin{minipage}{0.5\textwidth}
\centering
\makecell[l]{
    \textit{in premio promettendola a quel d'essi,} \\
    \textit{ch'in quel conflitto, in quella gran giornata,} \\
    \textit{degl'infideli più copia uccidessi,}\\
    \textit{e di sua man prestasse opra più grata.}\\
    \textit{Contrari ai voti poi furo i successi;}\\
    \textit{ch'in fuga andò la gente battezzata,}\\
    \textit{e con molti altri fu 'l duca prigione,}\\
    \textit{e restò abbandonato il padiglione.}
}
\end{minipage}
\end{center}

La donna viene promessa a chi fra Orlando e il cugino farà più morti in questa battaglia
- premio per chi fa più morte.
Questa ottava è bipartita perché, nella sua seconda parte,
in modo tutto imprevedibile, i cristiani (gente battezzata)
perdono la battaglia e si devono ritirare.
Namo viene imprigionato e non può più custodire Angelica, per cui rimane sola.

Qui termina l'Orlando innamorato.
Dalla prossima ottava la storia è tutta un'invenzione.

\begin{center} % 10
\begin{minipage}{0.5\textwidth}
\centering
\makecell[l]{
    \textit{Dove, poi che rimase la donzella} \\
    \textit{ch'esser dovea del vincitor mercede,}\\
    \textit{inanzi al caso era salita in sella,}\\
    \textit{e quando bisognò le spalle diede,}\\
    \textit{presaga che quel giorno esser rubella}\\
    \textit{dovea Fortuna alla cristiana fede:}\\
    \textit{entrò in un bosco, e ne la stretta via}\\
    \textit{rincontrò un cavallier ch'a piè venìa.}
}
\end{minipage}
\end{center}

Angelica, ancora prima dell'esito della battaglia, aveva intuito che sarebbe andata male per i cristiani,
e si era preparata a scappare.
Non perde tempo e scappa a cavallo. Incontra un cavaliere a piedi.
\snote{}{
    Un tipico elemento di Ariosto
    è il luogo di una selva, una stretta via labirintica
    dove, \textit{per caso}, incontra un cavaliere.
    Questo caso è il tema fondamentale di Ariosto. 
}


\begin{center} % 11
\begin{minipage}{0.5\textwidth}
\centering
\makecell[l]{
    \textit{Indosso la corazza, l'elmo in testa,} \\
    \textit{la spada al fianco, e in braccio avea lo scudo;}\\
    \textit{e più leggier correa per la foresta,}\\
    \textit{ch'al pallio rosso il villan mezzo ignudo.}\\
    \textit{Timida pastorella mai sì presta}\\
    \textit{non volse piede inanzi a serpe crudo,}\\
    \textit{come Angelica tosto il freno torse,}\\
    \textit{che del guerrier, ch'a piè venìa, s'accorse.}
}
\end{minipage}
\end{center}

L'ottava è bipartita perfettamente a metà.
Nella prima parte abbiamo la descrizione del cavaliere,
mentre la seconda è la reazione di Angelica.
La prima parte può ancora essere suddivisa a metà, perché i primi
due descrviono l'aspetto fisico, mentre gli altri due parlano di 
come il cavaliere si muovesse: più rapido di
chi un contadino che sta partecipando ad una gara
dove bisognasse inseguire un panno e prenderlo.
Nei primi due versioni abbiamo un chiasmo doppio fra la parte del corpo
e l'arma/oggetto.
\\ Appena Angelica vede il cavaliero, si ferma con una rapidità
maggiore di una timida pastorella che si scansa quando si trova un
serpente in mezzo ai piedi.
Angelica ha quindi una reazione terrorizzata.

\begin{center} % 12
\begin{minipage}{0.5\textwidth}
\centering
\makecell[l]{
    \textit{Era costui quel paladin gagliardo,} \\
    \textit{figliuol d'Amon, signor di Montalbano,}\\
    \textit{a cui pur dianzi il suo destrier Baiardo}\\
    \textit{per strano caso uscito era di mano.}\\
    \textit{Come alla donna egli drizzò lo sguardo,}\\
    \textit{riconobbe, quantunque di lontano,}\\
    \textit{l'\textbf{angelico sembiante} e quel bel volto}\\
    \textit{ch'all'amorose reti il tenea involto.}
}
\end{minipage}
\end{center}

Il cavaliere era il figlio del duca Amone, Rinaldo, uno dei cugini,
solo adesso viene svelato la sua identità.
Rinaldo sta inseguendo il cavallo Baiardo che era scappato per sbaglio.
Questa storia è stata raccontata nell'Orlando innamorato.
Abbiamo un'allusione al nome Angela mediante la sembianza angelica, e la visione
delle reti come catturato dall'amore.

\begin{center} % 13
\begin{minipage}{0.5\textwidth}
\centering
\makecell[l]{
    \textbf{La donna il palafreno a dietro volta,}\\
    \textbf{e per la selva a tutta briglia il caccia;}\\
    \textbf{né per la rara più che per la folta,}\\
    \textbf{la più sicura e miglior via procaccia:}\\
    \textbf{ma pallida, tremando, e di sé tolta,}\\
    \textbf{uderline{lascia cura al destrier che la via faccia.}}\\
    \textbf{\underline{Di sù di giù}, ne l'alta selva fiera}\\
    \textbf{\underline{tanto girò}, che venne a una riviera.}
}
\end{minipage}
\end{center}

Angelica non sceglie la strada, ma lascia che il cavallo la scelga al posto suo,
fino ad arrivare ad un fiume.
Il movimento casuale viene indicato da \underline{diverse espressioni}.

\begin{center} % 14
\begin{minipage}{0.5\textwidth}
\centering
\makecell[l]{
    \textbf{Su la riviera Ferraù trovosse}\\
    \textbf{di sudor pieno e tutto polveroso.}\\
    \textbf{Da la battaglia dianzi lo rimosse}\\
    \textbf{un gran disio di bere e di riposo;}\\
    \textbf{e poi, mal grado suo, quivi fermosse,}\\
    \textbf{perché, de l'acqua ingordo e frettoloso,}\\
    \textbf{l'elmo nel fiume si lasciò cadere,} \\
    \textbf{né l'avea potuto anco riavere}
}
\end{minipage}
\end{center}

Nell'Orlando innamorato, Angelica e Rinaldo avevano bevuto da delle fontane magiche:
Angelica da quella che fa innamorare, per cui si era innamorata di Rinaldo,
mentre Rinaldo da quella che fa odiare, per cui odiava Angelica.
Successivamente, i due bevettero dalle fontane opposte. Il motivo per cui
Angelica ha questa reazione è quindi perché odia e prova ribrezzo
per Rinaldo.

Sulla riviera trovò Ferraù, un guerriero musulmano.
Questo guerriera, pensava che si sarebbe fermato a bere dell'acqua e a riposare,
ma dalla sua sete il suo elmo era caduto nel fiume, e non l'aveva ancora
recuperato.
\snote{}{
    Tutti i personaggi stanno o scappando o sono alla ricerca di qualcosa.
}

\begin{center} % 15
\begin{minipage}{0.5\textwidth}
\centering
\makecell[l]{
    \textit{Quanto potea più forte, ne veniva} \\
    \textit{gridando la donzella ispaventata.}\\
    \textit{A quella voce salta in su la riva}\\
    \textit{il Saracino, e nel viso la guata;}\\
    \textit{e la conosce subito ch'arriva,}\\
    \textit{ben che di timor pallida e turbata,}\\
    \textit{e sien più dì che non n'udì novella,}\\
    \textit{che senza dubbio ell'è Angelica bella.}
}
\end{minipage}
\end{center}

Il guerrieri riconosce Angelica nonostante fosse pallida 
e terrorizzata e non avesse avuto notizie su di lei.
Anche il guerriero è innamorato di lei.

\begin{center} % 16
\begin{minipage}{0.5\textwidth}
\centering
\makecell[l]{
    \textit{E perché era cortese, e n'avea forse} \\
    \textit{non men de' dui cugini il petto caldo,}\\
    \textit{l'aiuto che potea tutto le porse,}\\
    \textit{pur come avesse l'elmo, ardito e baldo:}\\
    \textit{trasse la spada, e minacciando corse}\\
    \textit{dove poco di lui temea Rinaldo.}\\
    \textit{Più volte s'eran già non pur veduti,}\\
    \textit{m'al paragon de l'arme conosciuti.}
}
\end{minipage}
\end{center}

Il guerriere le porge tutto il suo aiuto, nonostante non avesse l'elmo.
Prende la spada per difenderla da chiunque stesse arrivando.
Più volte i due si erano già visti e si erano anche combattuti (Orlando innamorato).
Ferraù ha questa reazione istintiva per la sua cortesia (possiede i valori cavallereschi,
come difenderela donzella perseguitata).

\begin{center} % 17
\begin{minipage}{0.5\textwidth}
\centering
\makecell[l]{
    \textit{Cominciar quivi una crudel battaglia,} \\
    \textit{come a piè si trovar, coi brandi ignudi:} \\
    \textit{non che le piastre e la minuta maglia,} \\
    \textit{ma ai colpi lor non reggerian gl'incudi.} \\
    \textit{Or, mentre l'un con l'altro si travaglia,} \\
    \textit{bisogna al palafren che 'l passo studi;} \\
    \textit{che quanto può menar de le calcagna,} \\
    \textit{colei lo caccia al bosco e alla campagna.}
}
\end{minipage}
\end{center}

L'ottava è bipartita. La prima parte è dedicata al duello (ad armi pari).
i colpi che si davano, non solo sfondavano le piastre,
ma avrebbero spezzato anche le incudini (iperbole).
La seconda parte parla dell'esito del duello.
Mentre i due si stanno ammazzando, Angelica scappa con il suo cavallo.

\begin{center} % 18
\begin{minipage}{0.5\textwidth}
\centering
\makecell[l]{
    \textit{Poi che s'affaticar gran pezzo invano} \\
    \textit{i dui guerrier per por l'un l'altro sotto,} \\
    \textit{quando non meno era con l'arme in mano} \\
    \textit{questo di quel, né quel di questo dotto;} \\
    \textit{fu primiero il signor di Montalbano,} \\
    \textit{ch'al cavallier di Spagna fece motto,} \\
    \textit{sì come quel ch'ha nel cuor tanto fuoco,} \\
    \textit{che tutto n'arde e non ritrova loco.}
}
\end{minipage}
\end{center}

Entrambi sono abilissimi guerrieri, e nessuno dei due riesce a sopraffare l'altro.

\begin{center} % 19
\begin{minipage}{0.5\textwidth}
\centering
\makecell[l]{
    \textit{Disse al pagan: - Me sol creduto avrai,}
    \textit{e pur avrai te meco ancora offeso:}
    \textit{se questo avvien perché i fulgenti rai}
    \textit{del nuovo \textbf{sol} t'abbino il petto acceso,}
    \textit{di farmi qui tardar che guadagno hai?}
    \textit{che quando ancor tu m'abbi morto o preso,}
    \textit{non però tua la bella donna fia;}
    \textit{che, mentre noi tardiam, se ne va via.}
}
\end{minipage}
\end{center}

Rinaldo parla al guerriero musulmano alludendo all'inutilità del duello: entrambi si stanno danneggiando.
Angelica è scappata via, e anche se Ferraù uccide Rinaldo, la donna non sarà sua.
Angelica non viene nominata direttamente ma viene citata con
una descrizione stillnovistica, ossia quella della donna come raggi solari.

\begin{center} % 20
\begin{minipage}{0.5\textwidth}
\centering
\makecell[l]{
    \textit{Quanto fia meglio, amandola tu ancora,} \\
    \textit{che tu le venga a traversar la strada,}\\
    \textit{a ritenerla e farle far dimora,}\\
    \textit{prima che più lontana se ne vada!}\\
    \textit{Come l'avremo in potestate, allora}\\
    \textit{di chi esser de' si provi con la spada:}\\
    \textit{non so altrimenti, dopo un lungo affanno,}\\
    \textit{che possa riuscirci altro che danno. -}
}
\end{minipage}
\end{center}

Per terminare il discorso viene fatta una proposta pragmatica:
quella di risparmiare energia, raggiungere Angelica, bloccandola (farle far dimora),
e poi risumendo il duello.

\begin{center} % 21
\begin{minipage}{0.5\textwidth}
\centering
\makecell[l]{
    \textit{Al pagan la proposta non dispiacque:} \\
    \textit{così fu differita la tenzone;} \\
    \textit{e tal tregua tra lor subito nacque,} \\
    \textit{sì l'odio e l'ira va in oblivione,} \\
    \textit{che 'l pagano al partir da le fresche acque} \\
    \textit{non lasciò a piedi il buon figliuol d'Amone:} \\
    \textit{con preghi invita, ed al fin toglie in groppa,} \\
    \textit{e per l'orme d'Angelica galoppa.}
}
\end{minipage}
\end{center}

I due si dimenticano completamente dell'odio e dell'ira di pochi
minuti prima, e Ferraù, dopo aver accetato, offre un passaggio a cavallo.
Questo passaggio è dato dal fatto che entrambi posseggano i valori cavallereschi,
in particolare quello di avere armi pari. In questo caso,
il passaggio viene offerto affinché uno dei due non sia svantaggiato
rimandendo a piedi.

\begin{center} % 22
\begin{minipage}{0.5\textwidth}
\centering
\makecell[l]{
    \textit{Oh gran bontà de' cavallieri antiqui!}
    \textit{Eran rivali, eran di fé diversi,}
    \textit{e si sentian degli aspri colpi iniqui}
    \textit{per tutta la persona anco dolersi;}
    \textit{e pur per selve oscure e calli obliqui}
    \textit{insieme van senza sospetto aversi.}
    \textit{Da quattro sproni il destrier punto arriva}
    \textit{ove una strada in due si dipartiva.}
}
\end{minipage}
\end{center}

I due cavallieri sono rivali in amore, di diversa fede, e sono ancora
feriti dai colpi appena subiti. Nonostante ciò, i due si trovano sul medesimo cavallo.
I valori cavallereschi sono quindi più forti dei motivi per i quali
potrebbero continuare a duellare. Nessuno dei due teme di essere colpito alle spalle.

\begin{center} % 23
\begin{minipage}{0.5\textwidth}
\centering
\makecell[l]{
    \textit{E come quei che non sapean se l'una} \\
    \textit{o l'altra via facesse la donzella}\\
    \textit{(però che senza differenza alcuna}\\
    \textit{apparia in amendue l'orma novella),}\\
    \textit{si messero ad arbitrio di fortuna,}\\
    \textit{Rinaldo a questa, il Saracino a quella.}\\
    \textit{Pel bosco Ferraù molto s'avvolse,}\\
    \textit{e ritrovossi al fine onde si tolse.}
}
\end{minipage}
\end{center}

I due trovano un bivio con orme fresche da ambo el parti,
di conseguenza i due si separano.
Ferraù si avvolge nel bosco fino a ritorna al fiume di partenza. 

\snote{}{
    La ricerca che i personaggi affrontano è casuale,
    priva di indizi, senza tracce, basata sulle fortuna,
    senza fine e quindi senza una direzione precisa.
}

\begin{center} % 24
\begin{minipage}{0.5\textwidth}
\centering
\makecell[l]{
    \textit{Pur si ritrova ancor su la rivera,}\\
    \textit{là dove l'elmo gli cascò ne l'onde.}\\
    \textit{Poi che la donna ritrovar non spera,}\\
    \textit{per aver l'elmo che 'l fiume gli asconde,}\\
    \textit{in quella parte onde caduto gli era}\\
    \textit{discende ne l'estreme umide sponde:}\\
    \textit{ma quello era sì fitto ne la sabbia,}\\
    \textit{che molto avrà da far prima che l'abbia.}\\
}
\end{minipage}
\end{center}

Ferraù, immediatamente, torna a ricercare il suo elmo, come se tutta la vicenda
centrale di Angelica fosse sparita.

\begin{center} % 25
\begin{minipage}{0.5\textwidth}
\centering
\makecell[l]{
    \textit{Con un gran ramo d'albero rimondo,} \\
    \textit{di ch'avea fatto una pertica lunga,} \\
    \textit{tenta il fiume e ricerca sino al fondo,} \\
    \textit{né loco lascia ove non batta e punga.} \\
    \textit{Mentre con la maggior stizza del mondo} \\
    \textit{tanto l'indugio suo quivi prolunga,} \\
    \textit{vede di mezzo il fiume un cavalliero} \\
    \textit{insino al petto uscir, d'aspetto fiero.} \\
}
\end{minipage}
\end{center}

Prendendo un ramo, crea un bastone e lo usa per tastare il fondo del fiume.
Mentre svolge questo lavoro in maniera metodica e ossessiva,
sbuca un cavaliere che emerge fino al petto dal fiume, con un aspetto arrabiato.

\sdefinition{Stizza}{
    Viva irritazione, per lo più momentanea, provocata da un senso di fastidio o di molestia.
}

\begin{center} % 26
\begin{minipage}{0.5\textwidth}
\centering
\makecell[l]{
    \textit{Era, fuor che la testa, tutto armato,} \\
    \textit{ed avea un elmo ne la destra mano:} \\
    \textit{avea il medesimo elmo che cercato} \\
    \textit{da Ferraù fu lungamente invano.} \\
    \textit{A Ferraù parlò come adirato,} \\
    \textit{e disse: - Ah \textbf{mancator di fé}, marano!} \\
    \textit{perché di lasciar l'elmo anche t'aggrevi,} \\
    \textit{che render già gran tempo mi dovevi?}
}
\end{minipage}
\end{center}

L'ottava è bipartita in quattro versi di descrizione e quattro di dialogo.
Dalle parole del cavaliere si intuisce che i due si conoscono, e Ferraù
viene disprezzato per essere stato sleato.

\begin{center} % 27
\begin{minipage}{0.5\textwidth}
\centering
\makecell[l]{
    \textit{Ricordati, pagan, quando uccidesti} \\
    \textit{d'Angelica il fratel (che son quell'io),} \\
    \textit{dietro all'altr'arme tu mi promettesti} \\
    \textit{gittar fra pochi dì l'elmo nel rio.} \\
    \textit{Or se Fortuna (quel che non volesti} \\
    \textit{far tu) pone ad effetto il voler mio,} \\
    \textit{non ti \textbf{turbare}; e se \textbf{turbar} ti déi,} \\
    \textit{\textbf{turbati} che di fé mancato sei.}
}
\end{minipage}
\end{center}

Il cavaliere che sta parlando è un fantasma, ossia il fratello morte di Angelica,
ucciso da Ferraù.

\begin{center} % 28
\begin{minipage}{0.5\textwidth}
\centering
\makecell[l]{
    \textit{Ma se desir pur hai d'un elmo fino,} \\
    \textit{trovane un altro, ed abbil con più onore;} \\
    \textit{un tal ne porta Orlando paladino,} \\
    \textit{un tal Rinaldo, e forse anco migliore:} \\
    \textit{l'un fu d'Almonte, e l'altro di Mambrino:} \\
    \textit{acquista un di quei dui col tuo valore;} \\
    \textit{e questo, ch'hai già di lasciarmi detto,} \\
    \textit{farai bene a lasciarmi con effetto. -} \\
}
\end{minipage}
\end{center}

Il cavaliere continua suggerendo di trovarsi un altro elmo, con più orgoglio,
piuttosto che fare il vigliacco, e suggerisce anche alcune persone dalle quali prenderlo.

\begin{center} % 29
\begin{minipage}{0.5\textwidth}
\centering
\makecell[l]{
    \textit{All'apparir che fece all'improvviso} \\
    \textit{de l'acqua l'ombra, ogni pelo arricciossi,} \\
    \textit{e scolorossi al Saracino il viso;} \\
    \textit{la voce, ch'era per uscir, fermossi.} \\
    \textit{Udendo poi da l'Argalia, ch'ucciso} \\
    \textit{quivi avea già (che l'Argalia nomossi)} \\
    \textit{la rotta fede così improverarse,} \\
    \textit{di scorno e d'ira dentro e di fuor arse.}
}
\end{minipage}
\end{center}

Dopo le parole, viene data la reazione di Ferraù, il quale rimane pietrificato di paura.
Ferraù sa di essere stato sleale, e se ne vergogna.

\begin{center} % 30
\begin{minipage}{0.5\textwidth}
\centering
\makecell[l]{
    \textit{Né tempo avendo a pensar altra scusa,} \\
    \textit{e conoscendo ben che 'l ver gli disse,} \\
    \textit{restò senza risposta a bocca chiusa;} \\
    \textit{ma la vergogna il cor sì gli trafisse,} \\
    \textit{che giurò per la vita di Lanfusa} \\
    \textit{non voler mai ch'altro elmo lo coprisse,} \\
    \textit{se non quel buono che già in Aspramonte} \\
    \textit{trasse dal capo Orlando al fiero Almonte.}
}
\end{minipage}
\end{center}

Ferraù giura sulla propria madre di non volere altro elmo oltre quello di Orlando,
e parte per la ricerca del suo elmo.

\begin{center} % 31
\begin{minipage}{0.5\textwidth}
\centering
\makecell[l]{
    \textit{E servò meglio questo giuramento,} \\
    \textit{che non avea quell'altro fatto prima.} \\
    \textit{Quindi si parte tanto malcontento,} \\
    \textit{che molti giorni poi si rode e lima.} \\
    \textit{Sol di cercare è il paladino intento} \\
    \textit{di qua di là, dove trovarlo stima.} \\
    \textit{Altra ventura al buon Rinaldo accade,} \\
    \textit{che da costui tenea diverse strade.}
}
\end{minipage}
\end{center}

La \textit{ricerca}, casuale e consumante, va avanti per molti giorni.
La vicenda di Ferraù si chiude nei primi sei versi.
Attraverso la tecnica dell'entrelacement si torna alla vicenda di Ronaldo.

\begin{center} % 32
\begin{minipage}{0.5\textwidth}
\centering
\makecell[l]{
    \textit{Non molto va Rinaldo, che si vede} \\
    \textit{saltare inanzi il suo destrier feroce:} \\
    \textit{- Ferma, Baiardo mio, deh, ferma il piede!} \\
    \textit{che l'esser senza te troppo mi nuoce. -} \\
    \textit{Per questo il destrier sordo, a lui non riede} \\
    \textit{anzi più se ne va sempre veloce.} \\
    \textit{Segue Rinaldo, e d'ira si distrugge:} \\
    \textit{ma seguitiamo Angelica che fugge.}
}
\end{minipage}
\end{center}

Anche qui, con la tecnica dell'entrelacement, l'ottava viene suddivisa
cambiando la storia.
Rinaldo, per caso, ritrova il suo cavallo che stava cercando all'inizio
della sua prima apparizione.
Rinaldo ricomincia ad inseguirlo pieno di rabbia.
I verbi sono tutti legati all'idea di movimento.

\begin{center} % 33
\begin{minipage}{0.5\textwidth}
\centering
\makecell[l]{
    \textit{Fugge tra selve spaventose e scure,}\\
    \textit{per lochi inabitati, ermi e selvaggi.}\\
    \textit{Il mover de le frondi e di verzure,}\\
    \textit{che di cerri sentia, d'olmi e di faggi,}\\
    \textit{fatto le avea con subite paure}\\
    \textit{trovar di qua di là strani viaggi;} \\
    \textit{ch'ad ogni ombra veduta o in monte o in valle,} \\
    \textit{temea Rinaldo aver sempre alle spalle.}
}
\end{minipage}
\end{center}

La descrizione della selva è come quella Dantesca.
Ogni votla che Angelica sente un rumore, cambia strada data la sua paranoia.

\begin{center} % 34
\begin{minipage}{0.5\textwidth}
\centering
\makecell[l]{
    \textit{Qual pargoletta o damma o capriuola,}\\
    \textit{che tra le fronde del natio boschetto}\\
    \textit{alla madre veduta abbia la gola}\\
    \textit{stringer dal pardo, o aprirle 'l fianco o 'l petto,}\\
    \textit{di selva in selva dal crudel s'invola,}\\
    \textit{e di paura trema e di sospetto:}\\
    \textit{ad ogni sterpo che passando tocca,}\\
    \textit{esser si crede all'empia fera in bocca.}
}
\end{minipage}
\end{center}

Il sentimento di Angelica è esattamente come quello di una damma (femmina del daino) o una capriola,
che si ritrova da sola, perché ha assistito all'omicidio della madre,
e che quindi fugge temendo di far la stessa fine.

\begin{center} % 35
\begin{minipage}{0.5\textwidth}
\centering
\makecell[l]{
    \textit{Quel dì e la notte a mezzo l'altro giorno}\\
    \textit{s'andò aggirando, e non sapeva dove.}\\
    \textit{Trovossi al fin in un boschetto adorno,}\\
    \textit{che lievemente la fresca aura muove.}\\
    \textit{Duo chiari rivi, mormorando intorno,}\\
    \textit{sempre l'erbe vi fan tenere e nuove;}\\
    \textit{e rendea ad ascoltar dolce concento,}\\
    \textit{rotto tra picciol sassi, il correr lento.}
}
\end{minipage}
\end{center}

I primi due versi indicano la fuga frenetica, mentre i restanti sei indica il luogo.
Il luogo descritto è un locus amoenus. 

\begin{center} % 36
\begin{minipage}{0.5\textwidth}
\centering
\makecell[l]{
    \textit{Quivi parendo a lei d'esser sicura}\\
    \textit{e lontana a Rinaldo mille miglia,}\\
    \textit{da la via stanca e da l'estiva arsura,}\\
    \textit{di riposare alquanto si consiglia:}\\
    \textit{tra' fiori smonta, e lascia alla pastura}\\
    \textit{andare il palafren senza la briglia;}\\
    \textit{e quel va errando intorno alle chiare onde,}\\
    \textit{che di fresca erba avean piene le sponde.}
}
\end{minipage}
\end{center}

Angelica si calma, pensando di aver seminato Rinaldo, si riposa e lascia pascolare
il proprio cavallo.

\begin{center} % 37
\begin{minipage}{0.5\textwidth}
\centering
\makecell[l]{
    \textit{Ecco non lungi un bel cespuglio vede}\\
    \textit{di prun fioriti e di vermiglie rose,}\\
    \textit{che de le liquide onde al specchio siede,}\\
    \textit{chiuso dal sol fra l'alte querce ombrose;}\\
    \textit{così voto nel mezzo, che concede}\\
    \textit{fresca stanza fra l'ombre più nascose:}\\
    \textit{e la foglia coi rami in modo è mista,}\\
    \textit{che 'l sol non v'entra, non che minor vista.}
}
\end{minipage}
\end{center}

La natura ha creato come un cespuglio vuoto al suo interno, al suo quale 
si può entrare ma dove non entra quasi del tutto la luce, per cui un rifugio naturale.

\begin{center} % 38
\begin{minipage}{0.5\textwidth}
\centering
\makecell[l]{
    \textit{Dentro letto vi fan tenere erbette,}\\
    \textit{ch'invitano a posar chi s'appresenta.}\\
    \textit{La bella donna in mezzo a quel si mette,}\\
    \textit{ivi si corca ed ivi s'addormenta.}\\
    \textit{Ma non per lungo spazio così stette,}\\
    \textit{che un calpestio le par che venir senta:}\\
    \textit{cheta si leva e appresso alla riviera}\\
    \textit{vede ch'armato un cavallier giunt'era.}
}
\end{minipage}
\end{center}

Angelica si addormenta in mezzo alla natura.
Possiamo misurare una specie di anti-climax dall'ottava 33 (terrore assoluto),
36 (la calma del locus amoenus) e 38 (si addormenta).

\begin{center} % 39
\begin{minipage}{0.5\textwidth}
\centering
\makecell[l]{
    \textit{Se gli è \textbf{amico} o \textbf{nemico} non comprende:}\\
    \textit{\textbf{tema} e \textbf{speranza} il dubbio cor le scuote;}\\
    \textit{e di quella aventura il fine attende,}\\
    \textit{né pur d'un sol sospir l'aria percuote.}\\
    \textit{Il cavalliero in riva al fiume scende}\\
    \textit{sopra l'un braccio a riposar le gote;}\\
    \textit{e in un suo gran pensier tanto penètra,}\\
    \textit{che par cangiato in insensibil pietra.}
}
\end{minipage}
\end{center}

I primi due versi sono pervase da un chiasmo.
Angelica non viene vista dal cavaliere, ma riesce ad intravvederlo.
Il cavaliere si blocca nel suo pensiero.

\begin{center} % 40
\begin{minipage}{0.5\textwidth}
\centering
\makecell[l]{
    \textit{Pensoso più d'un'ora a capo basso}\\
    \textit{stette, Signore, il cavallier dolente;}\\
    \textit{poi cominciò con suono afflitto e lasso}\\
    \textit{a lamentarsi sì soavemente,}\\
    \textit{ch'avrebbe di pietà spezzato un sasso,}\\
    \textit{una tigre crudel fatta clemente.}\\
    \textit{Sospirante piangea, tal ch'un ruscello}\\
    \textit{parean le guance, e 'l petto un Mongibello.}
}
\end{minipage}
\end{center}

Il cavalliere si lamenta così soavemente che, dalla compassione, avrebbe
pure spaccato un sasso, o reso una tigre clemente (quattro iperboloi).
Il Mongibello è un altro nome dell'Etna.
Anche qui è presente un chiasmo (sospiri, pianto, ruscello, Mongibello).

\begin{center} % 41
\begin{minipage}{0.5\textwidth}
\centering
\makecell[l]{
    \textit{- Pensier (dicea) che 'l cor m'agghiacci ed ardi,}\\
    \textit{e causi il duol che sempre il rode e lima,}\\
    \textit{che debbo far, poi ch'io son giunto tardi,}\\
    \textit{e ch'altri a corre il frutto è andato prima?}\\
    \textit{a pena avuto io n'ho parole e sguardi,}\\
    \textit{ed altri n'ha tutta la spoglia opima.}\\
    \textit{Se non ne tocca a me frutto né fiore,}\\
    \textit{perché affligger per lei mi vuo' più il core?}
}
\end{minipage}
\end{center}

Il cavaliere parla, all'insaputa della presenza di Angelica.
Quello del verso primo è un ossimoro. Il cuore ghiacciato è una tipica immagine di Petrarca.
La disperazione del cavaliere è quello di amare una donna sapendo che si sia
precedentemente concessa ad un altro uomo (significato erotico sessuale esplicito).

\begin{center} % 42
\begin{minipage}{0.5\textwidth}
\centering
\makecell[l]{
    \textit{La verginella è simile alla rosa,}\\
    \textit{ch'in bel giardin su la nativa spina}\\
    \textit{mentre sola e sicura si riposa,}\\
    \textit{né gregge né pastor se le avvicina;}\\
    \textit{l'aura soave e l'alba rugiadosa,}\\
    \textit{l'acqua, la terra al suo favor s'inchina:}\\
    \textit{gioveni vaghi e donne inamorate}\\
    \textit{amano averne e seni e tempie ornate.}
}
\end{minipage}
\end{center}

\begin{center} % 43
\begin{minipage}{0.5\textwidth}
\centering
\makecell[l]{
    \textit{Ma non sì tosto dal materno stelo}\\
    \textit{rimossa viene e dal suo ceppo verde,}\\
    \textit{che quanto avea dagli uomini e dal cielo}\\
    \textit{favor, grazia e bellezza, tutto perde.}\\
    \textit{La vergine che 'l fior, di che più zelo}\\
    \textit{che de' begli occhi e de la vita aver de',}\\
    \textit{lascia altrui corre, il pregio ch'avea inanti}\\
    \textit{perde nel cor di tutti gli altri amanti.}
}
\end{minipage}
\end{center}

Ma non appena la rosa viene colta, staccata dal suo stelo materno,
perde tutta la sua bellezza poiché sfiorisce.
Quando una donna vergine lascia cogliere a qualcuno il fiore di cui deve avere più cura,
il pregio che aveva prima perde nel cuore di tutti gli altri amanti.
Nonostante dovrebbe perdere interesse, il cavaliere continua ad essere
tormentato dal pensiero di questa donna.

\begin{center} % 44
\begin{minipage}{0.5\textwidth}
\centering
\makecell[l]{
    \textit{Sia Vile agli altri, e da quel solo amata}\\
    \textit{a cui di sé fece sì larga copia.}\\
    \textit{Ah, Fortuna crudel, Fortuna ingrata!}\\
    \textit{trionfan gli altri, e ne moro io d'inopia.}\\
    \textit{Dunque esser può che non mi sia più grata?}\\
    \textit{dunque io posso lasciar mia vita propia?}\\
    \textit{Ah più tosto oggi manchino i dì miei,}\\
    \textit{ch'io viva più, s'amar non debbo lei! -}
}
\end{minipage}
\end{center}

Il paradosso è quello di non riuscire a smettere di amare
perché si desidera qualcosa che non è ottenibile.
Lui è convinto che Angelica abbia concesso la sua verginità a qualcun'altro.

%\textbf{Dal 49:}
% Il resto della storia:
%Angelica lo sdegna ma lo necessita per aiuto, e decise di cercare di farsi aiutare ma senza concedersi.
%Angelica esce dal cespuglio e gli dice di non avere opinioni false.
%Tuttavia, solo nel canto 19 verrà svelato la veridicità di questa proposizione.
%Il cavaliere comincia a spogliarsi, ma proprio in quel momento arriva un altro cavaliere,
%tutto vestito di bianco, con il quale comincia il duello.
%Sacripante viene sconfitto con tutto il peso del suo cavallo addosso,
%ma il cavaliere bianco non lo uccide e lo lascia così.

\subsubsection{Analisi}

L'ottava 22 è emblematica del fatto che Ariosto mostri, da una parte una
certa nostalgia di un vecchio mondo (di sette secondi prima) con dei valori cavallereschi,
dall'altra una grossa carica ironica.
La ricerca è una ricerca casuale, senza punti di riferimento, senza senso e senza indici
ma soprattutto senza risultato.
I personaggi si muovono senza sapere di per certo dove stanno andando
e alla fine vi è sempre la frustazione del non ottenere ciò che si desidera (stizza).

La visione medievale è quella del peccato rappresentato dalla selva, una volta caduti
nel peccato è difficile uscirne. La selva non è nè positiva nè negativa, è il posto
in cui capitano le cose più splendide e meravigliose e quelle più terribili.

Il percorso di Dante nella Commedia è una linea dritta attraverso il pianeta terra.
Invece, i peronsaggi di Ariosto si muovono in un modo completamente casuale (mondo Rinascimentale).
Questa differenza rappresenta il lento cambiamento che porta ad un mondo meno assoluto e più relativo,
con sempre più radici nell'immanenza piuttosto che nel trascendente (verso il Rinascimento).

Vi sono una serie di interventi da parte del narratore il quale
esprime il proprio giudizio (7.2, 22.2, 29.6, 32.8, 81.7-8).
Gli interventi diretti al lettore sono per far sì che il lettore mantenga vivo il suo senso critico, ricordandogli
della natura fittizia della storia.

\subsection{Canto XIX}

\subsubsection{Testo}

\begin{center} % 31
\begin{minipage}{0.5\textwidth}
\centering
\makecell[l]{
    \textit{O conte Orlando, o re di Circassia,} \\
    \textit{vostra inclita virtù, dite, che giova?} \\
    \textit{Vostro alto onor dite in che prezzo sia,} \\
    \textit{o che mercé vostro servir ritruova.} \\
    \textit{Mostratemi una sola cortesia} \\
    \textit{che mai costei v'usasse, o vecchia o nuova,} \\
    \textit{per ricompensa e guidardone e merto} \\
    \textit{di quanto avete già per lei sofferto.}
}
\end{minipage}
\end{center}

Si rivolge a Orlando, si rivolge a Sacripante con una domanda retorica,
ma il vostro onore, servizio e cortesia, a che cosa servono? (A niente).

\begin{center} % 32
\begin{minipage}{0.5\textwidth}
\centering
\makecell[l]{
    \textit{Oh se potessi ritornar mai vivo,} \\
    \textit{quanto ti parria duro, o re Agricane!} \\
    \textit{che già mostrò costei sì averti a schivo} \\
    \textit{con repulse crudeli ed inumane.} \\
    \textit{O Ferraù, o mille altri ch'io non scrivo,} \\
    \textit{ch'avete fatto mille pruove vane} \\
    \textit{per questa ingrata, quanto aspro vi fôra,} \\
    \textit{s'a costu' in braccio voi la vedesse ora!}
}
\end{minipage}
\end{center}

Il campo semantico che attraversa questa e l'ottava precedente fanno
parte delle parole cavalleresche, il codice dell'onore, il codice cortese
della generosità e del servizio verso qualcuno.

A differenza del lettore, i personaggi del testo non conoscono il carattere di Angelica,
ossia un'ingrata manipolatrice e sfruttatrice.
Angelica si vergogna si sè stessa per essere abbassata a tal punto di essersi fatta accompagnare
da personaggi come Orlando.
Invece, per i personaggi, Angelica è l'ideale di principessa, per la quale fare tutto,
perché prima o poi la si potrà ottenere.
Queste ultime due ottave delineano proprio questo concetto dell'idealizzazione di Angelica
e della differenza di cosa pensano i cavaliere in confronto alla realtà effettiva.
In questi cavaliere vi è forse qualcosa di folle, ossia questo accanirsi a seguire qualcosa
che vogliono che sia come loro vogliono, ma così non è.

Tutti sono soggetti alla lima dell'amore, indipendentemente da quanto tu possa resistere.
In questo canto viene delineato come anche Angelica possa essere vittima di questa cosa.

Nonostante Ariosto segua la tradizione letteraria si diverse a rovesciare gli schemi.
Questa è fondamentalmente la base sulle quali si fondano le parodie.

\begin{center} % 37
\begin{minipage}{0.5\textwidth}
\centering
\makecell[l]{
    \textit{Poi che le parve aver fatto soggiorno} \\
    \textit{quivi più ch'a bastanza, fe' disegno}\\
    \textit{di fare in India del Catai ritorno,}\\
    \textit{e Medor coronar del suo bel regno.}\\
    \textit{Portava al braccio un cerchio d'oro, adorno}\\
    \textit{di ricche gemme, in testimonio e segno}\\
    \textit{del ben che 'l conte Orlando le volea;}\\
    \textit{e portato gran tempo ve l'avea.}
}
\end{minipage}
\end{center}

In quella parte dell'India chiamata Catai (oggi Cina, Ariosto per India intendeva tutta l'asia).

% 38 è una analessi sulla storia di quel braccialetto

\begin{center} % 39
\begin{minipage}{0.5\textwidth}
\centering
\makecell[l]{
    \textit{Non per amor del paladino, quanto} \\
    \textit{perch'era ricco e d'artificio egregio,} \\
    \textit{caro avuto l'avea la donna tanto,} \\
    \textit{che più non si può aver cosa di pregio.} \\
    \textit{Se lo serbò ne l'Isola del pianto,} \\
    \textit{non so già dirvi con che privilegio,} \\
    \textit{là dove esposta al marin mostro nuda} \\
    \textit{fu da la gente inospitale e cruda.}
}
\end{minipage}
\end{center}

Angelica non ha tenuto il braccialetto con molto riguardo perché fosse dono di Orlando, ma perché
fosse estremamente prezioso.
Qui viene mostrato il tratto di indifferenza e di avidità di Angelica.

\begin{center} % 40
\begin{minipage}{0.5\textwidth}
\centering
\makecell[l]{
    \textit{Quivi non si trovando altra mercede} \\
    \textit{ch'al buon pastor ed alla moglie dessi,} \\
    \textit{che serviti gli avea con sì gran fede} \\
    \textit{dal dì che nel suo albergo si fur messi,} \\
    \textit{levò dal braccio il cerchio e gli lo diede,} \\
    \textit{e volse per suo amor che lo tenessi.} \\
    \textit{Indi saliron verso la montagna} \\
    \textit{che divide la Francia da la Spagna.}
}
\end{minipage}
\end{center}

Non si separa a malincuore del braccialetto, per lei è solamente un valore economico alto.
Angelica ha comunque un minimo di senso di gratitudine nei confronti del pastore.

\begin{center} % 41
\begin{minipage}{0.5\textwidth}
\centering
\makecell[l]{
    \textit{Dentro a Valenza o dentro a Barcellona} \\
    \textit{per qualche giorno avea pensato porsi,} \\
    \textit{fin che accadesse alcuna nave buona} \\
    \textit{che per Levante apparecchiasse a sciorsi.} \\
    \textit{Videro il mar scoprir sotto a Girona} \\
    \textit{ne lo smontar giù dei montani dorsi;} \\
    \textit{e costeggiando a man sinistra il lito,} \\
    \textit{a Barcellona andar pel camin trito.}
}
\end{minipage}
\end{center}

Il progetto è quello di scendere dai pirenei e aspettare per qualche giorno una nave.

\begin{center} % 42
\begin{minipage}{0.5\textwidth}
\centering
\makecell[l]{
    \textit{Ma non vi giunser prima, ch'un uom pazzo}\\
    \textit{giacer trovato in su l'estreme arene,}\\
    \textit{che, come porco, di loto e di guazzo}\\
    \textit{tutto era brutto e volto e petto e schene.}\\
    \textit{Costui si scagliò lor come cagnazzo}\\
    \textit{ch'assalir forestier subito viene;}\\
    \textit{e diè lor noia, e fu per far lor scorno.}\\
    \textit{Ma di Marfisa a ricontarvi torno.}
}
\end{minipage}
\end{center}

In questa ottava avviene un colpo di scena; trovano un pazzo lì sulla spiaggia,
completamente imbrattato di fango e di terra (come un porco - simulitudine animalesca).
Questo pazzo si scaglia alla coppia aggredendoli
(come un cagnazzo salta addosso ad una persona che non conosce).

L'ottava è divisa in sette versi più uno, dove nell'ultimo verso
il narratore sospende la storia sul più bello.
Non si sa chi sia il pazzo nè se farà effettivamente del male.
Verrà svelato che l'uomo è in realtà Orlando.

\subsubsection{Analisi}

Tutto l'episodio è diviso in due,
abbiamo la storia di Cloridano e Medoro (episodio epico e di guerra)
ed un episodio amoroso (Medoro e Angelica).

All'ottava 17 si nota il cambiamento e passaggio tra i due episodi.
Si passa in modo armonioso tra l'uno e l'altro.
Cambia da un argomento all'altro per mantenere vivo l'interesse.

Viene richiamata una delel storie d'amore più famose di tutta la lettera mondiale,
ossia quella dell'Eneide (35-37).
La vicenda di Angelica viene descritta come l'amore di Petrarca, riprendendo sintagmi
e le rime in maniera verbatim.
L'elemento anormale è che i ruoli sono invertiti (rovesciamento); la donna
fa ciò che farebbe l'uomo. Inoltre, il punto di prospettiva è quello della donna,
lei è la protagonista.

\pagebreak

\subsection{Canto XIX}

\subsubsection{Testo}

\begin{center} % 100
\begin{minipage}{0.5\textwidth}
\centering
\makecell[l]{
    \textit{Lo strano corso che tenne il cavallo} \\
    \textit{del Saracin pel bosco senza via,} \\
    \textit{fece ch'Orlando andò duo giorni in fallo,} \\
    \textit{né lo trovò, né poté averne spia.} \\
    \textit{Giunse ad un rivo che parea cristallo,} \\
    \textit{ne le cui sponde un bel pratel fioria,} \\
    \textit{di nativo color vago e dipinto,} \\
    \textit{e di molti e belli arbori distinto.}
}
\end{minipage}
\end{center}

La prima parte dell'ottava è narrativa, mentre la seconda descrittiva.
Nella prima parte viene ripresa una vicenda di un tale Saracin.
Ad ogni verso vi è unêspressione che indica il movimento labirintico.
Il posto dove Orlando giunse è un locus amoenus.

\begin{center} % 101
\begin{minipage}{0.5\textwidth}
\centering
\makecell[l]{
    \textit{Il merigge facea grato l'orezzo} \\
    \textit{al duro armento ed al pastore ignudo;} \\
    \textit{sì che né Orlando sentia alcun ribrezzo,} \\
    \textit{che la corazza avea, l'elmo e lo scudo.} \\
    \textit{Quivi egli entrò per riposarvi in mezzo;} \\
    \textit{e v'ebbe \textbf{travaglioso} albergo e \textbf{crudo},} \\
    \textit{e più che dir si possa \textbf{empio} soggiorno,} \\
    \textit{quell'\textbf{infelice} e \textbf{sfortunato} giorno.}
}
\end{minipage}
\end{center}

Il merigge sono le ore più calde della giornata.
Vi sono quindi le condizioni perfette, l'aria fresca si bilancia perfettamente con il sole.
Grazie a queste condizioni, Orlando non senti nessun brivido nonostante sia barbato.
Da questo punto Ariosto comincia a infliggere la sventura a Orlando.
La seconda parte dell'ottava rovescia tutti gli aggettivi positivi della prima parte.
% travaglioso albero crudo, mette Agg. Nome. Agg al posto che NAA oppure AAN

\begin{center} % 102
\begin{minipage}{0.5\textwidth}
\centering
\makecell[l]{
    \textit{Volgendosi ivi intorno, vide scritti} \\
    \textit{molti arbuscelli in su l'ombrosa riva.} \\
    \textit{Tosto che fermi v'ebbe gli occhi e fitti,} \\
    \textit{fu certo esser di man de la sua \textbf{diva}.} \\
    \textit{Questo era un di quei lochi già descritti,} \\
    \textit{ove sovente con Medor veniva} \\
    \textit{da casa del pastore indi vicina} \\
    \textit{la bella donna del Catai regina.} \\
}
\end{minipage}
\end{center}

Fra tutte le strade che avrebbe potuto prendere, Orlando prende proprio
quella dove Medoro e Angelica hanno inciso il loro nome, di cui si parla quattro canti prima.
Dopo aver visto le scritte, ferna lo sguardo e le fissa.
Orlando defiinsce Angelica come diva, a questo punto del testo la sta quindi idealizzando.
È ancora convinta che Angelicia sia la donna raggiungibile che Orlando vuole.

\begin{center} % 103
\begin{minipage}{0.5\textwidth}
\centering
\makecell[l]{
    \textit{Angelica e Medor con cento nodi} \\
    \textit{legati insieme, e in cento lochi vede.} \\
    \textit{Quante lettere son, tanti son chiodi} \\
    \textit{coi quali Amore il cor gli punge e fiede.} \\
    \textit{Va col pensier cercando in mille modi} \\
    \textit{non \colorbox{BurntOrange}{creder} quel ch'al suo dispetto \colorbox{BurntOrange}{crede}:} \\
    \textit{ch'altra Angelica sia, \colorbox{BurntOrange}{creder} si sforza,} \\
    \textit{ch'abbia scritto il suo nome in quella scorza.}
}
\end{minipage}
\end{center}

Orlando spera che questo sia un caso di omonimia, e quindi non la sua Angelica ad aver inciso il
suo nome.
In realtà, Orlando sa che si tratta della regina del Catai, ma si sforza di credere
diversamente. La realtà è talmente dolorosa, per amore e gelosia, che non vuole accettarla.

\fcolorbox{black}{BurntOrange}{\rule{0pt}{5pt}\rule{5pt}{0pt}}
La ripetizione di questa parola indica il suo pensiero ossessivo.

\begin{center} % 104
\begin{minipage}{0.5\textwidth}
\centering
\makecell[l]{
    \textit{Poi dice: - Conosco io pur queste note:} \\
    \textit{di tal'io n'ho tante vedute e lette.} \\
    \textit{Finger questo Medoro ella si puote:} \\
    \textit{forse ch'a me questo cognome mette. -} \\
    \textit{Con tali opinion dal ver remote} \\
    \textit{usando fraude a sé medesmo, stette} \\
    \textit{ne la speranza il malcontento Orlando,} \\
    \textit{che si seppe a se stesso ir procacciando.}
}
\end{minipage}
\end{center}

Orlando riconosce la scrittura di Angelica.
Forse il nome Medoro è un riferimento ad Orlando.
Questo è il secondo tentantivo della delusione e autoinganno Orlando.

\begin{center} % 105
\begin{minipage}{0.5\textwidth}
\centering
\makecell[l]{
    \textit{Ma sempre più raccende e più rinuova,} \\
    \textit{quanto spenger più cerca, il rio sospetto:} \\
    \textit{come l'incauto augel che si ritrova} \\
    \textit{in ragna o in visco aver dato di petto,} \\
    \textit{quanto più batte l'ale e più si prova} \\
    \textit{di disbrigar, più vi si lega stretto.} \\
    \textit{Orlando viene ove s'incurva il monte} \\
    \textit{a guisa d'arco in su la chiara fonte.}
}
\end{minipage}
\end{center}

Questa ottava è divisa in tre parti.
Più Orlando cerca di smettere di pensarci, più il pensiero diventa forte.

\begin{center} % 106
\begin{minipage}{0.5\textwidth}
\centering
\makecell[l]{
    \textit{Aveano in su l'entrata il luogo adorno} \\
    \textit{coi piedi storti edere e viti erranti.} \\
    \textit{Quivi soleano al più cocente giorno} \\
    \textit{stare abbracciati i duo felici amanti.} \\
    \textit{V'aveano i nomi lor dentro e d'intorno,} \\
    \textit{più che in altro dei luoghi circostanti,} \\
    \textit{scritti, qual con carbone e qual con gesso,} \\
    \textit{e qual con punte di coltelli impresso}
}
\end{minipage}
\end{center}

Orlando trova la grotta dove i due si erano amati durante le ore più calde,
dove vi sono molte incisioni.

\begin{center} % 107
\begin{minipage}{0.5\textwidth}
\centering
\makecell[l]{
    \textit{Il mesto conte a piè quivi discese;} \\
    \textit{e vide in su l'entrata de la grotta} \\
    \textit{parole assai, che di sua man distese} \\
    \textit{Medoro avea, che parean scritte allotta.} \\
    \textit{Del gran piacer che ne la grotta prese,} \\
    \textit{questa sentenza in versi avea ridotta.} \\
    \textit{Che fosse culta in suo linguaggio io penso;} \\
    \textit{ed era ne la nostra tale il senso:}
}
\end{minipage}
\end{center}

L'informazione nuova è che Medoro ha lasciato una scritta, quasi fresca,
in cui descrive il piacere che ha avuto in quella grotta.
Evidentmeente, essendo musulmano, Medoro ha scritto in arabo.

\begin{center} % 108
\begin{minipage}{0.5\textwidth}
\centering
\makecell[l]{
    \textit{- Liete piante, verdi erbe, limpide acque,} \\
    \textit{spelunca opaca e di fredde ombre grata,} \\
    \textit{dove la bella Angelica che nacque} \\
    \textit{di Galafron, da molti invano amata,} \\
    \textit{spesso ne le mie braccia nuda giacque;} \\
    \textit{de la commodità che qui m'è data,} \\
    \textit{io povero Medor ricompensarvi} \\
    \textit{d'altro non posso, che d'ognor lodarvi:} \\
}
\end{minipage}
\end{center}

\begin{center} % 109
\begin{minipage}{0.5\textwidth}
\centering
\makecell[l]{
    \textit{e di pregare ogni signore amante,} \\
    \textit{e cavallieri e damigelle, e ognuna} \\
    \textit{persona, o paesana o viandante,} \\
    \textit{che qui sua volontà meni o Fortuna;} \\
    \textit{ch'all'erbe, all'ombre, all'antro, al rio, alle piante} \\
    \textit{dica: benigno abbiate e sole e luna,} \\
    \textit{e de le ninfe il coro, che proveggia} \\
    \textit{che non conduca a voi pastor mai greggia. -}
}
\end{minipage}
\end{center}

Il narratore dà la traduzione del testo scritto in arabo.
Medoro spera di lasciare un messaggio a chi viene dopo, in maniera tale
da non deturbare questi luoghi.
Spera che questi luoghi rimangano immacolati così come quando lui è stato con Angelica
quella volta.
Medoro è straordinariamente lucido, consapevole della sua fortuna e
di una grande altezza poetica.

\begin{center} % 110
\begin{minipage}{0.5\textwidth}
\centering
\makecell[l]{
    \textit{Era scritto in arabico, che 'l conte} \\
    \textit{intendea così ben come latino:} \\
    \textit{fra molte lingue e molte ch'avea pronte,} \\
    \textit{prontissima avea quella il paladino;} \\
    \textit{e gli schivò più volte e danni ed onte,} \\
    \textit{che si trovò tra il popul saracino:} \\
    \textit{ma non si vanti, se già n'ebbe frutto;} \\
    \textit{ch'un danno or n'ha, che può scontargli il tutto.} \\
}
\end{minipage}
\end{center}

Orlando conosce varie lingue, fra cui l'arabo molto bene.
Questa conoscenza gli è servita varie volte in territorio nemico.
Questa nuova consapevolezza contrasta il valore positivo di sapere la lingua, perché
Orlando capisce esattamente il testo di Medoro.
Meodo, in maniera completamente innocua, ha citato nel testo il fatto che
Angelica sia amata in vano da molti. Questa espressione provoca molto dolore a
Orlando.

\begin{center} % 111
\begin{minipage}{0.5\textwidth}
\centering
\makecell[l]{
    \textit{Tre volte e quattro e sei lesse lo scritto} \\
    \textit{quello infelice, e pur cercando invano} \\
    \textit{che non vi fosse quel che v'era scritto;} \\
    \textit{e sempre lo vedea più chiaro e piano:} \\
    \textit{ed ogni volta in mezzo il petto afflitto} \\
    \textit{stringersi il cor sentia con fredda mano.} \\
    \textit{Rimase al fin con gli occhi e con la mente} \\
    \textit{fissi nel sasso, al sasso indifferente.}
}
\end{minipage}
\end{center}

Orlando legge il testo svariate volte nella speranza di trovare un dettaglio
per alimentare la propria delusione.

\begin{center} % 112
\begin{minipage}{0.5\textwidth}
\centering
\makecell[l]{
    \textit{Fu allora per uscir del sentimento} \\
    \textit{sì tutto in preda del dolor si lassa.} \\
    \textit{Credete a chi n'ha fatto esperimento,} \\
    \textit{che questo è 'l duol che tutti gli altri passa.} \\
    \textit{Caduto gli era sopra il petto il mento,} \\
    \textit{la fronte priva di baldanza e bassa;} \\
    \textit{né poté aver (che 'l duol l'occupò tanto)} \\
    \textit{alle querele voce, o umore al pianto.}
}
\end{minipage}
\end{center}

Il narratore indica di aver provato per esperienza questo dolore, il dolore
più forte di tutti, da cui nessuno è al riparo.
Lo prova il narratore e lo prova pure il più grande dei cavallieri.

\begin{center} % 113
\begin{minipage}{0.5\textwidth}
\centering
\makecell[l]{
    \textit{L'impetuosa doglia entro rimase,} \\
    \textit{che volea tutta uscir con troppa fretta.} \\
    \textit{Così veggiàn restar l'acqua nel vase,} \\
    \textit{che largo il ventre e la bocca abbia stretta;} \\
    \textit{che nel voltar che si fa in su la base,} \\
    \textit{l'umor che vorria uscir, tanto s'affretta,} \\
    \textit{e ne l'angusta via tanto s'intrica,} \\
    \textit{ch'a goccia a goccia fuore esce a fatica.} \\
}
\end{minipage}
\end{center}

Il dolore viene descritto con la similitudine di questa ottava.

\begin{center} % 114
\begin{minipage}{0.5\textwidth}
\centering
\makecell[l]{
    \textit{Poi ritorna in sé alquanto, e pensa come} \\
    \textit{possa esser che non sia la cosa vera:} \\
    \textit{che voglia alcun così infamare il nome} \\
    \textit{de la sua donna e \colorbox{BurntOrange}{crede} e \colorbox{BurntOrange}{brama} e \colorbox{BurntOrange}{spera},} \\
    \textit{o gravar lui d'insopportabil some} \\
    \textit{tanto di gelosia, che se ne pera;} \\
    \textit{ed abbia quel, sia chi si voglia stato,} \\
    \textit{molto la man di lei bene imitato.}
}
\end{minipage}
\end{center}

Come ultimo tentativo di delusione, Orlando crede che qualcosa possa avere imitato
la grafia di Angelica, o per infangarle il nome, o per farlo impazzire.

\begin{center} % 115
\begin{minipage}{0.5\textwidth}
\centering
\makecell[l]{
    \textit{In così poca, in così debol speme} \\
    \textit{sveglia gli spiriti e gli rifranca un poco;} \\
    \textit{indi al suo Brigliadoro il dosso preme,} \\
    \textit{dando già il sole alla sorella loco.} \\
    \textit{Non molto va, che da le vie supreme} \\
    \textit{dei tetti uscir vede il vapor del fuoco,} \\
    \textit{sente cani abbaiar, muggiare armento:} \\
    \textit{viene alla villa, e piglia alloggiamento.}
}
\end{minipage}
\end{center}

\begin{center} % 116
\begin{minipage}{0.5\textwidth}
\centering
\makecell[l]{
    \textit{Languido smonta, e lascia Brigliadoro} \\
    \textit{a un discreto garzon che n'abbia cura;} \\
    \textit{altri il disarma, altri gli sproni d'oro} \\
    \textit{gli leva, altri a forbir va l'armatura.} \\
    \textit{Era questa la casa ove Medoro} \\
    \textit{giacque ferito, e v'ebbe alta avventura.} \\
    \textit{Corcarsi Orlando e non cenar domanda,} \\
    \textit{di dolor sazio e non d'altra vivanda.}
}
\end{minipage}
\end{center}

Orlando è talmente distrutto, senza il cavallo, armatura, speroni d'oro, e senza Angelica,
che non è più un cavaliere (simbolicamente); torna un uomo, tanto che qualcuno deve aiutarlo.
Il grande eroe, di fronte a questo dolore, regredisce. 

\begin{center} % 117
\begin{minipage}{0.5\textwidth}
\centering
\makecell[l]{
    \textit{Quanto più cerca ritrovar quiete,} \\
    \textit{tanto ritrova più travaglio e pena;} \\
    \textit{che de l'odiato scritto ogni parete,} \\
    \textit{ogni uscio, ogni finestra vede piena.} \\
    \textit{Chieder ne vuol: poi tien le labra chete;} \\
    \textit{che teme non si far troppo serena,} \\
    \textit{troppo chiara la cosa che di nebbia} \\
    \textit{cerca offuscar, perché men nuocer debbia.}
}
\end{minipage}
\end{center}

Orlando vorrebbe chiedere al pastore chi siano i due che hanno fatto le incisioni,
ma si frena dal farlo in quanto teme che la nebbia che lo distacca dalla realtà svanisca.

\begin{center} % 118
\begin{minipage}{0.5\textwidth}
\centering
\makecell[l]{
    \textit{Poco gli giova usar fraude a se stesso;} \\
    \textit{che senza domandarne, è chi ne parla.} \\
    \textit{Il pastor che lo vede così oppresso} \\
    \textit{da sua tristizia, e che voria levarla,} \\
    \textit{l'istoria nota a sé, che dicea spesso} \\
    \textit{di quei duo amanti a chi volea ascoltarla,} \\
    \textit{ch'a molti dilettevole fu a udire,} \\
    \textit{gl'incominciò senza rispetto a dire:}
}
\end{minipage}
\end{center}

Il pastore, vedendolo triste, prova a rincuorargli la bella storia di Angelica e Medoro.

\begin{center} % 119
\begin{minipage}{0.5\textwidth}
\centering
\makecell[l]{
    XXX
}
\end{minipage}
\end{center}

\begin{center} % 120
\begin{minipage}{0.5\textwidth}
\centering
\makecell[l]{
    XXX
}
\end{minipage}
\end{center}

Il narratore riferisce cosa dice il pastore a Orlando.
% il braccialetto di pegno al suo amore. 

% orlando muore per il dolore, etc.

% fra 131-132 è come quasi se non si rispettassero più i confini dell'ottava.

\pagebreak

\part{Analisi dei testi}

% versi tronchi piani e sfruccioli | accenti tonici etc.
% le rime si considerano tali dalle medesime lettere dopo l'ulima vocale con accento tonico

L'analisi di un testo viene separata nell'\textit{analisi metrica} e nell'\textit{analisi}.

\paragraph{Analisi metrica}

\phantom{ }\vspace{0.1cm}
\sexample{Analisi metrica}{
   Sonetto con schema ABBA, ABBA, CDE, CDE.
}

\paragraph{Analisi}

Perifrasi = giro di parole

\section{Analisi delle novelle}

\begin{enumerate}
    \item Struttura
    \item Personaggi
    \item Tempo e spazio
    \item Novella e realtà storica
    \item Interpretauione storica-ideologica
\end{enumerate}

\end{document}

% espe 2
% non vi è cornice perché neifile comincia subito a parlare
% Dal momento che qualcuno prende la parola, è già novella.

% le forze di natura non possono essere contrastate:
% se non fosse stato per soldi, ma per amore, forse potrebbe essere stata perdonata