\documentclass[a4paper]{article}

\usepackage{amsmath}
\usepackage{amssymb}
\usepackage{parskip}
\usepackage{fullpage}
\usepackage{hyperref}
\usepackage{stellar}

\hypersetup{
    colorlinks=true,
    linkcolor=black,
    urlcolor=blue,
    pdftitle={Fisica},
    pdfpagemode=FullScreen,
}

\title{Fisica}
\author{Paolo Bettelini}
\date{}

\begin{document}

\maketitle
\tableofcontents

\pagebreak

\section{Forze}

\sdefinition{Costante di Coulomb}{
    La \textit{costante di Coulomb} è data da
    \[
        k = 9 \cdot 10^9 \frac{N \cdot m^2}{C^2} 
    \]
    dove \(C\) è l'unità di misura della carica elettrictà.
}

\sdefinition{Forza di Coulomb}{
    La \textit{forza di Coulomb} è la forze con la quale due cariche elettriche ferme,
    \(q_1\) e \(q_2\), a distanza \(r\), si attraggono
    \[
        F_Q = k \frac{q_1 q_2}{r^2}
    \]
    dove \(k\) è la costante di Coulomb.
}

\pagebreak

\section{Molle}

Due molle in parallelo hanno il medesimo allungamento,
mentre due molle in serie hanno la stessa forza.

% Spinta di Archimede
% Volume immerso = V liquido spostato
% Massa liquido spostato = Massa totale oggetto

\pagebreak

\section{Scontri fra oggetti}

\sdefinition{Urto elastico e anelastico}{
    Quando due oggetti si scontrano, se essi rimangono
    attaccati viene detto \textit{anelastico}, mentre se i due oggetti
    si dividono l'urto viene detto \textit{elastico}.
}

\sdefinition{Quantità di moto}{
    La \textit{quantità di moto} è una grandezza fisica definita come il prodotto fra
    massa e velocità
    \[
        p = mv
    \]
    dove \(p\) è la quantità di moto, \(v\) la velocità e \(m\) la massa. 
}

In un urto la quantità di moto viene conservata.

\begin{align*}
    p_1^i + p_2^i &= p_1^f + p_2^f \\
    m_1^i v_1^i + mp_2^i  v_2^i &= m_1^f v_1^f + m_2^f v_2^f \\
    m_1 (v_1^i - v_1^f) &= m_2 (v_2^i - v_2^f) \\
    m_1 &= m_2 \left(
        \frac{v_2^i - v_2^f}{v_1^i - v_1^f}
    \right)
\end{align*}

\sdefinition{Teorema dell'impulso}{
    Il \textit{teorema dell'impulso}
    dice che il cambiamento della quantità di modo in un impulso
    è pari alla forza applicata per il tempo passato
    \[
        \Delta \vec{p} \triangleq \int_{t_0}^{t_1} \vec{F} dt    
    \]
}

Nel caso in cui la forza è costante abbiamo
\[
    \Delta \vec{p} = \vec{F}\Delta t
\]

\end{document}
